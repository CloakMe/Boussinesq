\documentclass{article} 

%\usepackage{amsmath,amsthm,amsfonts}
\usepackage[english]{babel}
\usepackage[numbers]{natbib}
\usepackage{rotating}
\usepackage{amsfonts,amsmath}
\usepackage{graphicx}
\usepackage{color} 
\usepackage[notref,notcite]{showkeys}
\usepackage{multirow}
\usepackage[utf8]{inputenc}
\usepackage[T2A]{fontenc}

\newcommand{\be}{\begin{equation}}
\newcommand{\ee}{\end{equation}}
\newcommand{\rf}[1]{(\ref{#1})}
\newcommand{\RR}{\mathbb{R}}
\newtheorem{thm}{Theorem}
\newtheorem{lm}{Lemma}

\begin{document}
\section{Увод}
Парадигматично уравнение на Бузинеск
\be\label{problemCh}
...
\ee
\section{Смяна на променливите}

За удобство правим следната смяна на променливите (виж \cite{ref25}):

\begin{align}
x = \sqrt{\beta_1} \bar{x}, \quad y = \sqrt{\beta_1} \bar{y}, \quad t = \sqrt{\beta_1} \bar{t} \nonumber
\end{align}
която променя основното уравнение \rf{problemCh} в
\be\label{problemVC}
 \beta (I-\Delta) \frac{\partial^2}{\partial t^2}u= 
(I-\Delta)\Delta u +\Delta( (\beta - 1 )u - \alpha \beta u^2 )
\ee
където $\beta = \beta_1/\beta_2$. 

Неограниченият домейн на уравнението е заменен с достатъчно голяма дискретна област $\Omega$, така че стойностите на неизвестната функция $u$ са достатъчно малки близо до границата $\partial \Omega$. Използвана е равномерна мрежа  $\Omega_h$ дефинирана по следния начин:

$$
\Omega_h = \{(x_i,y_j): x_i = (i-\frac{N_x-1}{2})h, y_j = (j-\frac{N_y-1}{2})h, i = 0,\cdots, N_x, j = 0 ,\cdots , N_y \},
$$

където $N_x$ и $N_y$ описват броя на точките по осите $x$ и $y$, а стъпката по пространството $h$ удовлетворява $h =2 L_x/(N_x-1) =2 L_y/(N_y-1)$.
С $2 L_x$ и $2 L_y$ са означени размерите на $\Omega_h$ по осите $x$ и $y$. Дискретния времеви интервал е дефиниран аналогично чрез
$$
T_{\tau} = \{(t_k): t_k = k\tau, k = 0,\cdots ,N_t \},
$$
където с $\tau = T/N_t$ сме означили стъпката по времето. Стойността на неизвестната функция $u$ в точка от мрежата $(x_i,y_j,t_k)$ е означена с $u_{i,j}^k$,т.е. долния индекс се използва за пространствената дискретизация, а горния за времева дискретизация.

\section{Запазване на дискретната енергия}
Нека да дефинираме оператора $A$, така че да удовлетворява $Av=-\Delta_h v=-v_{\bar{x}x} - v_{\bar{y}y}$. А сега да разгледаме диференчната схема

\be\label{FDS1}
\beta (E+A)v_{\bar{t}t}^k +\beta Av^k+A^2 v^k -\alpha \beta A\left(\frac{(v^{k+1})^3-(v^{k-1})^3}{3(v^{k+1}-v^{k-1})} \right)=0
\ee
Умножаваме \rf{FDS1} с $A^{-1}$ и получаваме
\be\label{FDS2}
\beta (E+A^{-1})v_{\bar{t}t}^k +\beta v^k+A v^k -\alpha \beta \frac{(v^{k+1})^3-(v^{k-1})^3}{3(v^{k+1}-v^{k-1})} = 0
\ee
Ако заместим $v^{k}=0.5(v^{k+1}+v^{k-1})-\frac{\tau^2}{2}v_{\bar{t}t}^k$ в \rf{FDS2} получаваме
\begin{align*}
&\left( \beta (E+A^{-1})- \frac{\tau^2}{2}(\beta E+A ) \right)v_{\bar{t}t}^k  + \frac{1}{2} (\beta E +A )(v^{k+1}+v^{k-1}) \\
&~~~~~-\alpha \beta \frac{(v^{k+1})^3-(v^{k-1})^3}{3(v^{k+1}-v^{k-1})} =0
\end{align*}
Умножаваме последното уравнение с $(v^{k+1}-v^{k-1})=\tau (v_{\bar{t}}^k + v_{t}^k)$, което важи за всяка точка от пространствената мрежа $(x_i,y_j) \in \Omega_h$, и сумираме получените уравнения в спрямо всички точки от мрежата. Така, след допълнителна реорганизация на членовете в уравнението получаваме, че
\be \label{num_en}
E_h(v^k) =E_h(v^{k-1}),
\ee
където
\begin{align*}
E_h(v^k)=\left( \left( \beta (E+A^{-1})- \frac{\tau^2}{4}(\beta E+A ) \right)v_{t}^k ,v_{t}^k \right)+\frac{1}{4} \beta \left(  v^{k+1}+v^{k}, v^{k+1}+v^{k} \right) \\
+\frac{1}{4}  \left(  A(v^{k+1}+v^{k}), v^{k+1}+v^{k} \right)
-\alpha \beta \frac{((v^{k+1})^3,1)+((v^{k})^3,1)}{3}.
\end{align*}
Така доказваме следната теорема
\begin{thm}
Решението получено от диференчната схема \rf{FDS1} запазва дискретната енергия $E_h(v^0)$, т.е.  $E_h(v^k) =E_h(v^{0})$ за всяко $k=1,2,...N_t$.
\end{thm}

\begin{thm}
Линейната диференчна схема съответстваща на \rf{FDS1} е условно устойчива, когато е изпълнено
$\tau^2 < \frac{\beta}{2}(1-\frac{\tau^2}{4}) h^2$.

\end{thm}
Доказателството е следствие от изследване на устойчивост при диференчни схеми в книгата на 
Самарский \cite{samarski}.

\section{Квадратурни формули за дискретните маса и енергия}

Масата при непрекъснатата задача \rf{problemVC} се дефинира като
\begin{equation}\label{intM}
D(u(x,y,t))=\int_{\RR^2} u(x,y,t)dx dy
\end{equation}
а енергията 
\begin{align}\label{ex-en}
E(u(x,y,t)):=&\int_{\RR^2} u_t(x,y,t) \left((A^{-1}+E)u_t(x,y,t)\right) dxdy+
\beta \int_{\RR^2} u^2(x,y,t) dxdy \nonumber\\
+& \int_{\RR^2}u(x,y,t) \left(A u(x,y,t)\right) dxdy
-\frac{2 \alpha \beta}{3} \int_{\RR^2} u^3(x,y,t) dxdy,
\end{align}
където $Au=-\Delta u$, а $D(u(x,y,t)) = D(u(x,y,0))$ and $E(u(x,y,t)) = E(u(x,y,0))$ са непрекъснатите маса и енергия за задачата \rf{problemVC} (виж \cite{ref1}). Нека да заменим оператора $Au=-\Delta u$ с неговата дискретна версия $-\Delta_h u$ като използваме апроксимаций от различен ред - $O(|h|^2)$, $O(|h|^4)$, $O(|h|^6)$. Ако имаме дискретната производната $u_t$ изчислена с апроксимации също от втори, четвърти и шести ред - $O(\tau^2)$, $O(\tau^4)$, $O(\tau^6)$ - то тогава може да приложим квадратурни формули за численото пресмятане на енергията \rf{ex-en}.

%When the numerical solution is found with $O(h^2+\tau^2)$ error, we apply \rf{quadr2} with $O(h^2)$ error;

%When the numerical solution is found with $O(h^4+\tau^4)$ error, we apply 2D Simpson's rule \rf{quadr4} with $O(h^4)$ error;

%When the numerical solution is found with $O(h^6+\tau^6)$ error, we apply 2D Boole's rule \rf{quadr6-2D} with $O(h^6)$ error;

The following integral 
Следният интеграл
\begin{equation}\label{int}
D(u(x,y))=\int_{a_1}^{b_1} \int_{a_2}^{b_2} u(x,y)dx dy
\end{equation}

$x_i, ~i=0,1,...,N_x$; $x_0=a_1,~x_{N_x}=b_1$, $h_1=(b_1-a_1)/(N_x-1)$, 


$y_j, ~j=0,1,...,N_y$; $y_0=a_2,~y_{N_y}=b_2$,  $h_2=(b_2-a_2)/(N_y-1)$

може да се изчисли посредством двумерни квадратурни формули както е описано по-долу.

\subsection{ 2Д формула на трапците с глобална грешка $O(|h_1|^2+|h_2|^2)$ }

Апроксимацията на интеграла \eqref{intM} с глобална грешка $O(|h_1|^2+|h_2|^2)$ е

\begin{align}\label{quadr2}
D_h(u_{i,j}) =& \sum_{i=1}^{N_x-1} \sum_{j=1}^{N_y-1} h_1 h_2 u_{i,j}
+\frac{h_1}{2}\sum_{i=0} \sum_{j=1}^{N_y-1} h_2 u_{i,j}
+\frac{h_1}{2}\sum_{i=N_x} \sum_{j=1}^{N_y-1} h_2 u_{i,j} \nonumber\\
+&\frac{h_2}{2}\sum_{j=0} \sum_{i=1}^{N_x-1} h_1 u_{i,j}
+\frac{h_2}{2}\sum_{j=N_y} \sum_{i=1}^{N_x-1} h_1 u_{i,j}
\nonumber\\
+&\frac{1}{4}h_1 h_2 \left(u_{0,0}+u_{N_x,0}+u_{N_x,N_y}+u_{0,N_y}
\right).
\end{align}

\subsection{ 2Д правило на Симпсън с глобална грешка $O(|h_1|^4+|h_2|^4)$ error}

Нека да приемем, че $N_x=2k$, $N_y=2 l$. Така, за всяко $m=0,1,2,\cdots N_y$ пресмятаме, че

$$D_m= \frac{h_1 }{3} 
\left\{ u_{0,m}+u_{N_x,m}+ 4 \sum_{i=1}^{\frac{N_x}{2}}   u_{2i-1,m}
 +2 \sum_{i=1}^{\frac{N_x}{2}-1} u_{2i,m} \right\}.$$


От последната формула получаваме

\begin{equation}\label{quadr4}
D_h(u_{i,j}) =\frac{h_2 }{3} 
\left\{ D_{0}+D_{N_y}+ 4 \sum_{j=1}^{\frac{N_y}{2}}   D_{2j-1}
 +2 \sum_{j=1}^{{\frac{N_y}{2}}-1} D_{2j} \right\}
\end{equation}
апроксимацията на интеграла \eqref{intM} с глобална грешка $O(|h_1|^4+|h_2|^4)$.


\subsection{ 2D правило на Буул с глобална грешка $O(|h_1|^6+|h_2|^6)$}

Нека да приемем, че $N_x=4k$, $N_y=4 l$. Така, за всяко $m=0,1,2,\cdots N_y$ пресмятаме, че

\begin{align*}
D_m =& \frac{2h_1}{45} 
\left\{
7u_{0,m}+7u_{N_x,m}+32 \sum_{i=1}^{\frac{N_x}{2}}u_{2i-1,m}
+12\sum_{i=1}^{\frac{N_x}{4}}u_{4i-2,m}
+14 \sum_{i=1}^{\frac{N_x}{4}-1}u_{4i,m}
\right\}.
\end{align*}

Тогава формула \eqref{quadr6-2D} е апроксимацията на интеграла \eqref{intM} с глобална грешка $O(|h_1|^6+|h_2|^6)$.

\begin{align}\label{quadr6-2D}
&D_h(u_{i,j})  =
\frac{2h_2}{45} 
\left\{
7D_{0}+7D_{N_y}+32 \sum_{j=1}^{\frac{N_y}{2}}D_{2j-1}
+12\sum_{j=1}^{\frac{N_y}{4}}D_{4j-2}
+14 \sum_{j=1}^{\frac{N_y}{4}-1}D_{4j}
\right\}.
\end{align}

\section{Numerical Methods}

За всеки от разработените методи са изчислени масата и енергията върху полученото решение. То, заедно с неговите свойства като маса, енергия и форма са сравнени. Изчисленията са направени върху три вложени мрежи, за да се изследва сходимостта на методите. Показано е, че решенията получени с метода на Тейлор и Консервативната схема са доста близки. Една от целите на статията е да покаже, че метода на Тейлор приложен към уравнението \rf{problemVC} също води до достатъчно добри резултати, а също така може да се приложи с по-висок ред на апроксимация, което води и до по-фино решение.

\subsection{ Консервативна схема с крайни разлики }

Консервативната схема използва крайни разлики с втори ред на апроксимация, затова се предполага, че третите производни на непрекъснатото решение по времето и пространстово съществуват.  Тези апроксимациите са дефинирани както следва:

\be\label{difft}
\frac{\partial^2 u}{\partial t^2}(x_i, y_j, t_k ) = \frac{ u^{(k+1)}_{i, j} - 2u^{(k)}_{i,j} + u^{(k-1)}_{i,j} }{\tau^2} + O(\tau^2) 
\ee

\be\label{diffD}
\Delta u(x_i, y_j, t_k )  = \frac{ u^{(k)}_{i+1, j} - 2u^{(k)}_{i,j} + u^{(k)}_{i-1,j} }{h_1^2} + \frac{ u^{(k)}_{i, j+1} - 2u^{(k)}_{i,j} + u^{(k)}_{i,j-1} }{h_2^2} + O(h_1^2 + h_2^2) 
\ee
Като заместим дискретните диференциални оператори \rf{difft} и \rf{diffD} в \rf{problemVC} получаваме следното диференчно уравнение
\be\label{consFDS}
\beta (I-\Delta_h)\frac{ u^{(k+1)}_{i, j} - 2u^{(k)}_{i,j} + u^{(k-1)}_{i,j} }{\tau^2} = (\Delta_h - \Delta_h^2)u^{(k)}_{i,j} + \Delta_h(g(u^{(k)}_{i,j})),
\ee
%
където нелинейният член $g$ е дефиниран както следва:
\begin{align}
g(u^{(k)}_{i,j})=& -\frac{\alpha \beta} { 3 } \left( (u^{(k+1)}_{i,j})^2 + (u^{(k-1)}_{i,j})(u^{(k+1)}_{i,j}) + (u^{(k-1)}_{i,j})^2 \right) + \nonumber\\
+&\frac{ (\beta - 1 )}{ 2 }\left( u^{(k+1)}_{i,j} + u^{(k-1)}_{i,j} \right).
\end{align}

Това не е тривиална апроксимация на $g$, а такава с която дискретната енергия се запазва, т.е. е константна функция по отношение на времевата променлива $t$ (виж \cite{ref20}). Да отбележим, че Консервативната схема \rf{consFDS} е имплицитна тъй като нелинейният член зависи от горния слой по времето. Така на всеки слой по времето се правят итераций на Пикард, за да разрешим неизвестната дискретна функция $u^{(k+1)}_{i,j}$.

\subsection{ Taylor Series Approach with Method of Lines }

Методът на Тейлор използва развитие в ред спрямо времевата променлива на търсената функция $u(x,y,t)$. Но за извеждането на крайните разлики за пресмятане на производните по пространството също се използва развитие в ред на Тейлор. Затова се предполага, че решението е $p+1$ кратна гладка функция спрямо всички аргументи $x$, $y$ and $t$, т.е. $u \in C^{p+1,p+1,p+1}(\Omega \times T)$ като при следващите изчисления описани по долу са използвани стойностите $p=2,4,6$.
За дискретизацията на оператора на Лаплас са използвани централни крайни разлики с различна степен на апроксимация:
\begin{equation}\label{fd}
u_{\widehat{xx},p}(x,y) :=  \frac{1}{h^2} \sum\limits_{i=-p/2}^{p/2} d_i u(x+ih, y_j).
\end{equation}
като теглата $d_i$ са взети от \cite{forn} и са описани в Таблица \ref{table:A00}. Производните по $y$ са дефинирани аналогично със същия шаблон.
\begin{table}[ht]
\centering
\small
		\begin{tabular}{||c|l|l|l|l|l|l|l||}
			\hline
			\hline
            $p=2$          &          &                                 &     1      &   -2   &    1    &    &        \\
   			\hline 
			\hline 
           $p=4$          &                            &   $-\frac{1}{12}$     &     $\frac{4}{3}$      &   $-\frac{5}{2} $     &    $\frac{4}{3}$    &  $-\frac{1}{12}$   &        \\
	   \hline
			\hline 
            $p=6$        &   $\frac{1}{90}$       &     $-\frac{3}{20}$     &    $\frac{3}{2}$      &    $-\frac{49}{18}$   &    $\frac{3}{2}$    & $-\frac{3}{20}$    &    $\frac{1}{90}$       \\
	   \hline
			\hline 
		\end{tabular}
	\caption{ Finite differences used for the approximation of the Laplace operator.}
	\label{table:A00}
\end{table}

Грешката от апроксимациите \rf{fd} са $O(h^p)$ в зависимост от избора на $p$. Нека да означим с $u_{i,j}(t)$ апроксимацията на неизвестната функция $u$ в точка от мрежата $(x_i, y_j)$ за произволно време $t$. Непрекъснатият оператор на Лаплас в уравнение \rf{problemVC} се заменя с дискретния $\Delta_{h,p} = u_{\widehat{xx},p} + u_{\widehat{yy},p}$, което води до следната система от ОДУ:
\be \label{DiscreteEq}
\beta (I-\Delta_{h,p}) \frac{\partial^2 u}{\partial t^2}(x_i, y_j, t)=
 (I - \Delta_{h,p})\Delta_{h,p} u_{i, j}(t) + \Delta_{h,p} ( g( u_{i, j}(t) ) )
\ee
за всяка точка от мрежата $(x_i, y_j)$, където $i = 0..N_x$, $j=0..N_y$. За всяко едно ОДУ от системата се прави развитие в ред на Тейлор спрямо времевата променлива:
\begin{align} \label{TSe}
u(x_i, y_j, t+\tau) = u(x_i, y_j, t) + \tau \frac{ \partial u }{ \partial t }(x_i, y_j, t)  + ... 
%\nonumber
%\\
\frac{ \tau^p }{ p! } \frac{ \partial^p u }{ \partial t^p }(x_i, y_j, t) + O(\tau^{p+1})
\end{align}
при $p \ge 2$. Реда на апроксимация по времето зависи от броя на членовете $p+1$, които участват в реда на Тейлор. По този начин реда на апроксимация по времето и пространството е еднакъв и зависи от избора на $p$, което по-късно спомага за тестването на сходимостта. Всяка точка от мрежата представлява начало на крива, чиято траектория се описва с развитието на Тейлор \rf{TSe}. Изчисляването на формула \rf{TSe} е рекурсивен процес като всеки член от сумата се пресмята отделно. Например, при $t=0$ първите два члена са известни от началното условие ($u_0$, $u_1$), а третият се получава чрез дискретното уравнение \rf{DiscreteEq}. За намиране на последното се използват Fast Poisson Solvers за обръщане на оператора $I-\Delta_{h,p}$ представен от дясната страна на уравнението. Чрез диференциране на уравнение \rf{DiscreteEq} по времето се получават зависимости за по-високи производни $\frac{\partial^3 u}{\partial t^3}$, $\frac{\partial^4 u}{\partial t^4}$, $\frac{\partial^5 u}{\partial t^5}$ и т.н.  Това е итеративен процес, където например петата производна зависи от третата, втората, първата и нулевата, третата зависи от първата и нулевата. След като всички необходими производни са пресметнати се заместват в уравнение \rf{TSe} и така се получават следните апроксимации: за $p=2$ имаме $O(|h|^2 + \tau^2)$, за $p=4$ - $O(|h|^4 + \tau^4)$ и за $p=6$ - $O(|h|^6 + \tau^6)$. Тук е важно да се отбележи, че е необходимо да се пресметне и развитието в ред на Тейлор за първата производна по времето $u_t(x_i, y_j, t+\tau)$, която е сходна с \rf{TSe} и има същия ред на апроксимация. Това е необходимо тъй като решението на следващия слой по времето зависи от двойката ($u$, $u_t$), която служи като базис и всички по-високи производни могат да се изразят единствено чрез нея.

Сложността на алгоритъма зависи от големината на областа в която се разглежда числената задача
$$ O( N_x N_y N_t ) $$,
където $N_x N_y$ е броя на точките в $\Omega_h$ и $N_t = T/\tau$. Повечето от изчислителното време се отделя за обръщане на оператора $I-\Delta_{h,p}$ в уравнение \rf{DiscreteEq}, което е необходимо за $N_x N_t$ обръщания на $p+1$ лентова матрица с размери ($N_y \times N_y$). Избора на $p$, който е направен тук увеличава сложността с константа, което води до линейна зависимост на времето за изчисление спрямо броя на точките в областта. 

\begin{thebibliography}{99}

\bibitem{samarski} Samarskii, A., The Theory of Difference Schemes, Marcel Dekker Inc., New York, 2001.

\bibitem{ref25} Kolkovska N., Two families of finite difference schemes for multidimensional Boussinesq paradigm equation, In:
{\it Applications of Mathematics in Technical and Natural Sciences,  Sozopol (Bulgaria)},
\emph{AIP Conference Proceedings}, \textbf{1301} (2010), 395.

\end{thebibliography}
\end{document}
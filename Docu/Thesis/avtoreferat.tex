\documentclass[a5paper]{article} 
%\usepackage{amsmath,amsthm,amsfonts}
\usepackage[bulgarian]{babel}
\usepackage[numbers]{natbib}
\usepackage{rotating}
\usepackage{amsfonts,amsmath,amsthm}
\usepackage{graphicx}
\usepackage{color} 
%\usepackage[notref,notcite]{showkeys}
\usepackage{multirow}
\usepackage[utf8]{inputenc}
\usepackage[T2A]{fontenc}
%\usepackage[outdir=./]{epstopdf}
\usepackage{epstopdf}
\usepackage{wrapfig}
\usepackage{placeins}
\usepackage[left=1.5cm,bottom=1.5cm,top=1cm,right=1cm]{geometry}
%\usepackage[colorlinks=false, urlcolor=blue]{hyperref}
%\usepackage[ sorting=none ]{biblatex}
%    natbib=true,
%    style=numeric,

\newcommand{\be}{\begin{equation}}
\newcommand{\ee}{\end{equation}}
\newcommand{\rf}[1]{(\ref{#1})}
\newcommand{\RR}{\mathbb{R}}
\newtheorem{thm}{Theorem}
\newtheorem{lm}{Lemma}

\newcommand{\eeth}{\rm}
\newcommand{\dO}{\partial\Omega_{h}}
\theoremstyle{remark}
\newtheorem*{remark}{Remark}

\begin{document}
\begin{large}

%titlepage
%\thispagestyle{empty}
\begin{center}
%University logo
\begin{figure}[!htb]
      \includegraphics[width=1\linewidth]{LogoThesis.png}
\end{figure}

\begin{minipage}{0.95\linewidth}
    \centering
    \vspace{0.5cm}
%Thesis title
    {\Large ЧИСЛЕНО ИЗСЛЕДВАНЕ НА\\ ДВУМЕРНОТО УРАВНЕНИЕ НА БУСИНЕСК\par}
    \vspace{2cm}
%Author's name
    {\Large Красимир Андреев Ангелов \par с научен ръководител \\проф. д-р Наталия Кольковска\par}
    \vspace{2cm}
    {\Large АВТОРЕФЕРАТ\par}
    \vspace{1cm}
%Degree
    {\large на дисертация за получаване на образователната и научна степен „доктор”\\в област на висше образование\\4. Природни науки, математика и информатика,\\професионално направление 4.5. Математика,\\докторска програма „Математическо моделиране и приложение на математиката”\par}
    \vspace{3cm}
%Date
    {\Large Януари 2023}
\end{minipage}
\end{center}
\clearpage

\shipout\null
%\stepcounter{page}


\tableofcontents
\newpage

%\begin{appendix}
%  \listoffigures
%  \listoftables
%\end{appendix}

\section{Въведение}\label{introduction}
Настоящият труд разглежда моделна задача, описваща поведението на самотна вълна в плитки води, движеща се в правоъгълен канал. Математическият модел е открит първоначално от Джоузеф Бусинеск, важи за дълги нелинейни вълни и е известен с това, че нелинейността и дисперсията изпадат в самобалансиращо се състояние \cite{ref01,ref02}. Проф. Христо Христов \cite{ref1} извежда клас от вълнови уравнения, базирани на предходните две работи и едно от тях е наречено ``Парадигматично уравнение на Бусинеск'' (ПУБ):
\begin{align}
&u_{tt} - \Delta u -\beta_1  \Delta u_{tt} +\beta_2 \Delta ^2 u + \Delta f(u)=0 \text{ при } (x,y) \in \RR^2, \, t\in\RR^+,\label{eq1}
\\ \nonumber &u(x,y,0)=u_0(x,y), \, u_t(x,y,0)=u_1(x,y) \text{ при } (x,y) \in \RR^2,
\\  &u(x,y) \rightarrow 0,  \Delta u(x,y) \rightarrow 0 ,  \text{ при }  \sqrt{x^2 + y^2} \rightarrow \infty, \label{eq11}
\end{align}
където $f(u)=\alpha u^2$, $\alpha>0$ е амплитудата, $\beta_1>0$, $\beta_2>0$ са дисперсионни параметри, а $\Delta$ е операторът на Лаплас. Показано е, че уравнението \rf{eq1} е частично интегруемо и са изведени три закона за запазване на енергията, масата и момента \cite{ref1, ref159}. Други инварианти на \rf{eq1} не са ми известни. 

Задачата \rf{eq1}-\rf{eq11} и численото ѝ решаване ще са основна цел на настоящата работа. Уравненията на Бусинеск се използват не само за изучаване на динамиката на така наречените дълги вълни в плитка водна среда (например канал). Те могат да моделират разпространението на вълни в еластичен прът или в непрекъснатия еквивалент на решетъчни структури на молекулно ниво.

Началото на тази работа се фокусира върху решения на \rf{eq1} от вида 
\be\label{sub123}
u(x,y,t)=U(x,y-ct),
\ee
които са стационарни солитонни вълни (ССВ), движещи се по $y$ оста със скорост $c$. След полагане на \rf{sub123} в \rf{eq1}, ССВ удовлетворяват следното нелинейно елиптично диференциално уравнение от четвърти ред
\begin{equation}\label{eq2}
c^2 (E_1-\beta_1 \Delta) U_{yy} = \Delta U -\beta_2 \Delta^2 U - \Delta f(U),
\end{equation}
където $E_1$ е тъждествения оператор, при условие че $c < c_{max}:=\min (\sqrt{\beta_2}/ \sqrt{\beta_1},1)$. В последствие тези вълни ще служат за начално условие на хиперболичното уравнение \rf{eq1} -- \rf{eq11}. 

Алгебричните изрази от \cite{ref15}, които апроксимират решението на елиптичното уравнение \rf{eq2}, са използвани като начално условие за числени симулации на ПУБ \rf{eq1} -- \rf{eq11}, 
(виж  \cite{ref21, ref20, ref23, ref22, ref24}). Първите четири статии показват, че при стойности на скоростта $c \le 0.3$, ССВ се разсейват във формата на разширяваща се пръстеновидна вълна или избухват след кратък период от време. В последния труд \cite{ref24} са направени експерименти при по високи скорости $c=0.5,0.6$, 
но отново резултатите за формата на вълната са сходни с вече документираното поведение. 
Вижда се, че балансът между дисперсията и нелинейността в ПУБ \rf{eq1} е много крехък, 
което изисква повече усилия при запазването му. 

Едномерното ПУБ
\begin{align}
&u_{tt} - u_{xx} -\beta_1  u_{ttxx} +\beta_2 u_{xxxx} + f(u)_{xx}=0   \quad \text{при} \,  x \in \RR, \, t\in\RR^+,\label{eq1D}
\\ \nonumber &u(x,0)=u_0(x), \, u_t(x,0)=u_1(x)   \quad\text{при} \, x \in \RR,
\\  &u(x) \rightarrow 0,  u(x)_{xx} \rightarrow 0 ,  \quad \text{при} \, x \rightarrow \infty, \label{eq1d1}
\end{align}
притежава точно решение от тип солитон при $u =\tilde u(x-ct)$:
\begin{align}
\tilde u(x,t:c) = \frac{3}{2} \frac{c^2-1}{\alpha}\text{sech}^2 \left( \frac{1}{2}  \sqrt{ \frac{\beta_1 (c^2-1)}{\beta_1 c^2-\beta_2}} (x-c t \sqrt{\frac{\beta_1}{\beta_2}} ) \right).
\end{align}
Солитон е уединена вълна, движеща се със скорост $c$, от вида $\phi(x-ct)$, която е локализирана в пространството и при движение запазва формата си. Структурата ѝ се запазва дори след взаимодействие с друг(и) солитон(и).

Също така е добре известно, 
че при числени симулации в едномерния случай, за уравнение \rf{eq1D}, при условие, че $\beta_1/\beta_2 \le 1$ солитонните вълни са нестабилни за малки скорости $c$ около нулата и стабилни за 
по-големи скорости, както е показано в \cite{ref10000}. В последствие, такива изследвания са направени и за многомерното ПУБ, при $\beta_1/\beta_2 = 1$ (виж \cite{ref1c0}), където е изведено необходимо условие за скоростта, спрямо критичната енергия, при което вълната не избухва.

Тези наблюдения изместват фокуса към изчисляване на ССВ от \rf{eq1} с по-голяма точност и при по-големи скорости $c \approx c_{max}$, тъй като това е от съществено значение за конструирането на началните данни за ПУБ \rf{eq1}-\rf{eq11}. Важно е да получим числено решение на \rf{eq2} с висока точност, чрез гъвкав и устойчив процес, който ще позволи тестването на повече и различни сценарии за развитие на решенията на хиперболичната задача \rf{eq1}-\rf{eq11}.

\subsection{Литературен обзор}
Предисторията на изчисляването на ССВ за уравнение \rf{eq1} е свързана с няколко числени техники, както следва - спектрален метод на Гальоркин е изполван в \cite{ref14,ref13}; метод на простата итерация и шаблони с крайни разлики от втори ред върху неравномерна мрежа са приложени в \cite{ref117,ref116}; пертурбационно решение с развитие в ред около малък параметър (скоростта $c$) е представено в \cite{ref15}, където са изведени така наречените ``best-fit'' апроксимационни формули. Освен това в \cite{ref159} са изведени аналогични формули с тези от \cite{ref15}, но за нелинейност от вида $\alpha(u^3 - \sigma u^5)$.

Следните резултати се отнасят за двумерното ПУБ.
Христов, Кольковска и Василева \cite{ref20} разработват метод с неявна консервативна схема и неравномерна мрежа. Василева и Кольковска \cite{ref200} представят числен метод с движеща се координатна система. Черток, Христов и Курганов \cite{ref21} трансформират ПУБ в система от хиперболично и елиптично уравнения. При първото е използвана консервативна схема на Годунов, а за второто е приложен числен метод за елиптични уравнения базиран на бързи преобразувания на Фурие върху равномерна мрежа. В статията \cite{ref241} на Димова и Кольковска е изведена консервативна схема, където пространственият оператор пред втората производна по времето е разложен на произведение от три дискретни такива, което води до решаване на пет-диагонални системи от линейни уравнения. В \cite{ref23} Димова и Василева сравняват числения метод от \cite{ref241} с друга консервативна схема приложена върху системата показана в \cite{ref21}, при равномерна и неравномерна мрежи, като решенията от двата подхода си съответстват едно на друго. Кольковска \cite{ref25, ref251, ref252} е извела класове от трислойни и четирислойни консервативни диференчни схеми както с, така и без вътрешни итерации (на Пикард). Схемите без вътрешни итерации се изпълняват по-бързо. Блинков, Гердт, Панкратов и Коткова \cite{ref253} представят нова неявна консервативна схема, базирана на комбинацията от метод на крайните обеми и числено интегриране. Всички тези статии \cite{ref20, ref21, ref241, ref23, ref25, ref251, ref252, ref253} апроксимират вторите производни в уравнение \rf{eq1} с крайни разлики от втори ред - $O(|h|^2 + \tau^2)$. Кольковска и Ангелов \cite{ref22} използват векторни схеми на Абрашин върху равномерна мрежа. На всеки слой по времето се получават по две числени апроксимации за решението като втората ``изглажда'' първата, а грешката от апроксимациите е $O(|h|^2 + \tau)$. В \cite{ref24}, Юю Хе и Хонгтао Чен, извеждат неявна компактна консервативна схема с грешки $O(|h|^4 + \tau^2)$. Числените резултати са направени върху равномерна мрежа. Вучева и Кольковска \cite{ref254} прилагат вариация на Рунге-Кута схеми за получаване на симплектични числени методи с грешки $O(|h|^2 + \tau^4)$. 

Всичките статии, изброени в тази част, които представят числени решения на двумерното Парадигматично уравнение на Бусинеск (някои от статиите представят метод за многомерното ПУБ, но резултати само за едномерната задача), имат поне един пример с начално условие от \cite{ref15}, т.е. ``best-fit'' формулите. Разбира се са използвани и други, но нито една от тези работи не представя резултати с начално условие получено чрез итерационен подход за решаване на съответното елиптично уравнение \rf{eq2} (напр. ``Метод на Якоби'', ``Метод на простата итерация" и т.н.). Входни данни, (т.е. $u_0(x,y)$ и $u_1(x,y)$ от \rf{eq1}-\rf{eq11}) получени с такъв алгоритъм (наречен още ``False Transients''), са разработени от Христов и Чаудури в \cite{ref117,ref116}. Въпреки това все още липсват диференчни схеми от четвърти и по-висок ред $O(|h|^p)$, $p \ge 4$ за решаване на елиптичното стационарно уравнение на Бусинеск. В допълнение, все още няма разработени диференчни схеми за двумерното Парадигматично уравнение на Бусинеск с четвърти и по-висок ред на апроксимация на вторите производни, едновременно по пространството и времето $O(|h|^p + \tau^p)$, $p \ge 4$.
%\iffalse

\subsection{Цели на дисертацията}
Основната цел на този труд е да покаже, че двумерното ПУБ притежава числени решения със свойства, които са сходни с тези на солитоните. Построяването на подходящо начално условие заема съществена част от изследването на ПУБ и води до решаване на друга задача - стационарното (елиптично) уравнение на Бусинеск. Подробното изследване на тези две уравнения поставят следните цели:
\begin{itemize}
  \item построяване на диференчни схеми с висок ред на апроксимация (втори, четвърти и шести) върху равномерна мрежа за решаването на двумерните стационарно и хиперболично уравнения на Бусинеск;
  \item изследване на измененията в свойствата (енергията, масата, максимума и формата) на численото решение в зависимост от реда на апроксимация;
  \item сравнение между численото решение, получено за елиптичната задача, с шести ред на апроксимация, и ``best-fit'' формулите от Пертурбационното решение на проф. Христов \cite{ref15};
  \item построяване на ново асимптотично гранично условие за решаването на елиптичната задача;
  \item изследване на числените решения на хиперболичното уравнение \rf{eq1} при по-високи скорости $c$ близки до допустимия максимум $c \approx \min (\sqrt{\beta_2}/ \sqrt{\beta_1},1)$, $c < \min (\sqrt{\beta_2}/ \sqrt{\beta_1},1)$ с начално условие, получено от численото решение на елиптичната задача \rf{eq2} (построено с метода на простата итерация);
  \item сравнение между числените резултати, получени с метода на Тейлор и други известни в литературата решения, използващи консервативни схеми (виж \cite{ref20, ref23}) с втори ред на апроксимация на вторите производни за хиперболичната задача.
\end{itemize}

\section{Съдържание на дисертацията}
 
\textbf{Глава 2. Основни числени инструменти}

В тази глава са дефинирани помощни средства и алгоритми, които се използват при численото решение на двумерните стационарно и хиперболично уравнения на Бусинеск. Дискретизацията на пространствената област $\Omega_h$ е дефинирана чрез:
\begin{align}\label{Omega}
\Omega_h = \{(x_i,y_j):& x_i = (i-\frac{N_x-1}{2})h, \; y_j = (j-\frac{N_y-1}{2})h, \nonumber\\
                                   & i = 0,\cdots, N_x-1, j = 0 ,\cdots , N_y-1 \},
\end{align}
където $N_x$ и $N_y$ описват броят точки по осите $x$ и $y$. Стъпката по пространството $h$ удовлетворява $h =2 L_x/(N_x-1) =2 L_y/(N_y-1)$, където $2 L_x$ и $2 L_y$ са размерите на областта $\Omega$. Дискретният времеви интервал е дефиниран аналогично чрез
\be
T_{\tau} = \{(t_k): t_k = k\tau, k = 0,\cdots ,N_t-1 \},
\ee
където $N_t$ е броят точки по оста $t$, а $\tau = T/(N_t-1)$ е стъпката по времето.
\begin{table}[ht]
\centering
\small
		\begin{tabular}{||c|l|l|l|l|l|l|l||}
			\hline
			\hline
            $p=2$          &          &                                 &     1      &   -2   &    1    &    &        \\
   			\hline 
			\hline 
           $p=4$          &                            &   $-\frac{1}{12}$     &     $\frac{4}{3}$      &   $-\frac{5}{2} $     &    $\frac{4}{3}$    &  $-\frac{1}{12}$   &        \\
	   \hline
			\hline 
            $p=6$        &   $\frac{1}{90}$       &     $-\frac{3}{20}$     &    $\frac{3}{2}$      &    $-\frac{49}{18}$   &    $\frac{3}{2}$    & $-\frac{3}{20}$    &    $\frac{1}{90}$       \\
	   \hline
			\hline 
		\end{tabular}
	\caption{Централни крайни разлики, използвани при апроксимацията на оператора на Лаплас.}
	\label{table:A00}
\end{table}
За дискретизация на оператора на Лаплас, във вътрешността на $\Omega_h$, са използвани централни крайни разлики с различна степен на апроксимация:
\begin{equation}\label{fdx}
u_{\widehat{xx},p}(x,y) :=  \frac{1}{h^2} \sum\limits_{i=-p/2}^{p/2} d_i u(x+ih, y_j), \; p=2,4,6.
\end{equation}
като теглата $d_i$ са взети от \cite{forn} и са описани в Таблица \ref{table:A00}. Производните по $y$ се дефинират аналогично. С помощта на Таблица \ref{table:A00} и използвайки несиметрични крайни разлики напред и крайни разлики назад върху границата на областта $\partial \Omega_h$, са изведени следните диференчни оператори, в матричен вид, с големина $N_x \times N_x$, за вторите производни по $x$:
\[
\Delta_{h,2,x} = \frac{1}{h^2}
\begin{bmatrix}
    -2	       & 1        &     \dots   &   \dots        & 0   \\
    1               & -2            &   1           &   0               & \vdots    \\
        0           & \ddots        &    \ddots    &   \ddots       &  0 \\ 
    \vdots       &     0            &  1     	& -2    	   & 1 \\
    0               & \dots          &  \dots         & 1  	   & -2 \\
\end{bmatrix}
\]
\[
\Delta_{h,4,x} = \frac{1}{h^2}
\begin{bmatrix}
     -\frac{38}{15}	& \frac{29}{20}       & -\frac{2}{15}	&    \frac{1}{120}     &    \dots      	   &   0           & 0    \\
    \frac{4}{3}          &-\frac{5}{2}    	& \frac{4}{3}	&   -\frac{1}{12}	  &   \dots      	  &   0	           & \vdots  \\
    -\frac{1}{12}	& \frac{4}{3}         	& -\frac{5}{2}	&  \frac{4}{3}    	 &   -\frac{1}{12}	  &      0           &\vdots    \\
        0           		& \ddots        	&    \ddots   		 &   \ddots      	 &     \ddots      	  &  \ddots        &    0 \\	
\\
   \vdots      		 & 0           		 &  -\frac{1}{12}	& \frac{4}{3}    	& -\frac{5}{2}	&  \frac{4}{3}   &   -\frac{1}{12} \\
    0      		 &  \dots           	 &   0     		& -\frac{1}{12} 	 & \frac{4}{3} 	 & -\frac{5}{2}  &  \frac{4}{3}\\
    0              		 & \dots          	&  0              		 &\frac{1}{120} 	 &  -\frac{2}{15} 	& \frac{29}{20} & -\frac{38}{15}\\
\end{bmatrix}
\]
Матрицата $\Delta_{h,6,x}$ не е показана, защото е прекалено голяма, но се извежда аналогично като случаят при $p=4$. Матриците $\Delta_{h,p,y}, p=2,4,6$, които апроксимират вторите производни по $y$, имат същата структура, но различна големина ($N_y \times N_y$).  Дискретният оператор на Лаплас в $\Omega_h$  се дефинира чрез:
\be
\Delta_{h,p} = \Delta_{h,p,x} + \Delta_{h,p,y}, \; p=2,4,6,
\ee
където грешката от дискретните апроксимации е $O(h^p)$ в зависимост от избора на $p$.

Редът на сходимост на изследваните крайни разлики и развития в ред на Тейлор са получени посредством правилото на Рунге:
\begin{equation}\label{Runge}
\xi = ln  \frac{\Vert u_{h,\tau} - u_{(h/2,\tau/2)} \Vert_\kappa } {\Vert  u_{(h/2,\tau/2)} - u_{(h/4,\tau/4)} \Vert_\kappa  } / ln(2),
\end{equation}
тъй като не е известно аналитично решение на уравнението, а нормата $\kappa$ е подходящо подбрана. Тук $u_{h,\tau}$ е решението получено при стъпки $h$ и $\tau$. В настоящата работа за $\kappa$ са използвани $L_2$ и $L_\infty$ норми:
\begin{align*}
\Vert u_{h,\tau} \Vert_{L_2} = \sqrt{ h^2 \sum_{i=1}^{N_x-1} \sum_{j=1}^{N_y-1}  (u_{i,j})^2 } \\
\Vert u_{h,\tau} \Vert_{\infty} = \max_{i,j}(|u_{i,j}|), \; i=1,..,N_x, j=1,..,N_y
\end{align*}

Извели сме три различни квадратурни формули за пресмятане на интеграли, които са подробно описани в \cite{refHyp}. Тези формули са използвани за пресмятане на следните инварианти: маса и енергия за хиперболичната задача, съответно с апроксимационни грешки $O(h^2)$, $O(h^4)$ и $O(h^6)$. 

Използван е бърз директен метод за обръщане на двумерния дискретен оператор на Лаплас (Fast Poisson Solvers), който се описва чрез следната моделна задача:
\be\label{PoissonEq}
-\Delta v = f
\ee
дефинирана в областта $\Omega$ с хомогенни гранични условия $v \big|_{\partial\Omega} = 0$. Дискретната версия на \rf{PoissonEq} има вида:
\be\label{PsnDiscret}
-\Delta_{h,p,x}  V_h - V_h (\Delta_{h,p,y})^T = F_h,
\ee
където $V_h, F_h$ са двумерни матрици с елементи съответно $v(x_i,y_j)$ и  $f(x_i,y_j)$, и една и съща големина $(N_x-2)\times(N_y-2)$ (виж Фигура \ref{fig:FPSexplained}). Дискретните функции $V_h, F_h$ представят непрекъснатите такива $v, f$ ограничени върху мрежата $\Omega_h$, т.е. $V_h = v(\Omega_h)$ и $F_h = f(\Omega_h)$.
\begin{figure}[ht]
     \includegraphics[width=\linewidth]{FPSExplained.png}
	\caption{Дискретното уравнение на Лаплас \rf{PsnDiscret} в матричен вид}
	\label{fig:FPSexplained}
\end{figure}
\FloatBarrier
Този тип запис на задачата \rf{PoissonEq} е разгледан от Том Личи в \cite{ref34}, където областта е квадратна $L_x = L_y$ и са използвани централни крайни разлики от втори ред ($p=2$) с нулево гранично условие в $\partial \Omega_h$. При $p=2$ операторите $\Delta_{h,2,x}$ и $\Delta_{h,2,y}$ са симетрични, което допълнително улеснява търсенето на решение на поставената задача. Настоящата глава се явява разширение на горепосочения труд при произволна правоъгълна област с размери $2L_x \times 2L_y$ и несиметрични крайни разлики по границата на областта, където се прилага нулево гранично условие. Т.е. в общия случай матриците $\Delta_{h,p,x}$ и $\Delta_{h,p,y}$ не са симетрични.

\iffalse
Нека за матриците $(\Delta_{h,p,x})^T$ и $(\Delta_{h,p,y})^T$ имаме следните собствени стойности $\lambda_{x,i}$, $\lambda_{y,j}$ и собствени вектори $s_{x,i}$, $s_{y,j}$:
\begin{align}
S_x:=[s_{x,1},..,s_{x,N_x-2}],\\
D_x:= diag(\lambda_{x,1},..,\lambda_{x,N_x-2}),\\
S_y:=[s_{y,1},..,s_{y,N_y-2}],\\
D_y:= diag(\lambda_{x,1},..,\lambda_{x,N_y-2}).
\end{align}
Така се получава, че
\begin{align}\label{eigIdentity}
(\Delta_{h,p,x})^T  S_x = S_x  D_x\nonumber\\
(\Delta_{h,p,y})^T   S_y = S_y  D_y,
\end{align}
където $S_x, S_y$ се явяват матриците със собствени вектори и $D_x, D_y$ са диагоналните матрици със собствени стойности съответно на $(\Delta_{h,p,x})^T, (\Delta_{h,p,y})^T$. Последните са пресметнати, използвайки специализиран софтуер и библиотеки за работа с големи матрици. Ако положим
\be\label{subst}
X_h := ( S_x^T  V_h  S_y ) \quad \text{с размери} (N_x-2)\times(N_y-2),
\ee
и заместим в \rf{PsnDiscret}, се получава:
\be
-\Delta_{h,p,x}  (S_x^T)^{-1} X_h  S_y^{-1}  -(S_x^T)^{-1} X_h  S_y^{-1}  (\Delta_{h,p,y})^T = F_h.
\ee
След извършване на серия от прости математически операции, последният израз добива вида:
\be\label{fpp2}
- D_x  X_h -X_h  D_y = S_x^T  F_h  S_y.
\ee
Уравнение \rf{fpp2} е еквивалентно на
\be\label{fpp3}
-X_{i,j} \lambda_{x,i} - \lambda_{y,j} X_{i,j} = b_{i,j}.
\ee
разписано по компоненти и така се получава, че
\be\label{fpp4}
X_{i,j} = - b_{i,j}/(\lambda_{x,i} + \lambda_{y,j} ),
\ee
където $X_h = (X_{i,j})$ и $S_y^T  F_h   S_x = (b_{i,j})$ и това важи за всяко $(i,j)$:
$$i = 1,..,N_x-2, \quad j = 1,..,N_y-2 $$
с уточнението че $i = 0,N_x-1$, $j = 0,N_y-1$ са граничните стойности които не присъстват експлицитно в уравненията, но са взети в предвид. Имайки $X_h$ лесно може да получим $V_h$  с обратната субституция на \rf{subst}, която е 
\be\label{substInv}
V_h = (S_x^T)^{-1}  X_h  S_y^{-1}.
\ee

Изчислителната сложност при обръщането на оператора на Лаплас по метода в предходната част \ref{FPS} се разделя на две части. \textbf{ Първата част (I)} е свързана с изчисляването на собствените стойности $D_x, D_y$ и вектори $S_x, S_y$ на двете матрици $(\Delta_{h,p,x})^T$ и $(\Delta_{h,p,y})^T$. Също така се включват и пресмятанията, необходими за получаването на обратните матрици $(S_x^T)^{-1}$ и $S_y^{-1}$. Сложността на тези две процедури се оценя на $O(N_x^3+N_y^3)$ (\cite{ref260}) и съответно на $O(N_x^{2.37}+N_y^{2.37})$ (\cite{ref27}). \textbf{Втората част (II)} е свързана предимно с умножение на неразредени матрици в дясната част на \rf{fpp2} и обратната субституция \rf{substInv}, като и в двата случая произведението се формира от три матрици  с големини както следва: първата - $(N_x-2) \times (N_x-2)$, втората (средната) - $(N_x-2) \times (N_y-2)$ и третата - $(N_y-2) \times (N_y-2)$. \textbf{Това води} до изчислителна сложност от
\be\label{fpsComplex}
O(N_x N_y \bar{\epsilon}),
\ee
\textbf{където $\bar{\epsilon} \in (0, max(N_x, N_y))$ (\cite{ref26, ref27})}. В случая, когато областта е квадратна, $N_x = N_y$ се получава, че $\bar{\epsilon} = 0.37 N_x$, т.е. алгоритмичната сложност е $O(N_x^{2.37})$. При метода на Тейлор изчисленията, описани в (I), се правят еднократно и получените резултати се използват в частта (II), която се прави за всеки слой по времето, т.е. (II) и \rf{fpsComplex} са с по-голяма тежест. 
\fi

\vspace{0.5cm}
\textbf{Глава 3. Формулировка и числено решаване на двумерното стационарно уравнение на Бусинеск - Метод на простата итерация}

За удобство правим следната смяна на променливите: $x = \sqrt{\beta_1} \bar{x}, \; y = \sqrt{\beta_1} \bar{y}$, която трансформира уравнение \rf{eq2} в 
\begin{align}\label{eq3}
&c^2 \beta (E_1- \Delta) v_{{\overline y}{\overline y}} = \beta \Delta v - \Delta^2 v - \beta \Delta f(v), \quad \beta = \beta_1/\beta_2
\end{align}
За улеснение навсякъде по-надолу в текста ще се използват отново старите означения $x,y$ вместо ${\overline x},{\overline y}$.

\textbf{Част 3.1} Метод на простата итерация за числено решаване на стационарното Парадигматично уравнение на Бусинеск \rf{eq45}

Тук е описана числената имплементация на метода на простата итерация за решаване на системата от две елиптични уравнения \begin{equation}\label{eq45}
\begin{split}
 &- (1 - c^2 \beta) \widehat{v}_{yy} -\widehat{v}_{xx} + \beta (1-c^2) \widehat{v} - \alpha \beta \theta \widehat{v}^2 = \widehat{w}, \; \; \beta = \beta_1/\beta_2, \\
 &- \Delta \widehat{w} =  c^2 \beta \widehat{v}_{xx},
\end{split}
\end{equation}
която е еквивалентна на задачата \rf{eq3} и $\theta$ е стойността на максимума на търсената функция $v$ в нулата (виж \cite{ref16}). Дефинирана е явна диференчна схема за частните диференциални уравнения:
\begin{align}\label{eq5}
\begin{split}
 &\frac {\partial \widehat{v}}{\partial t} - (1 - c^2 \beta) \widehat{v}_{yy} -\widehat{v}_{xx} + \beta (1-c^2) \widehat{v} - \alpha \beta \theta \widehat{v}^2 = \widehat{w}, \\
 &\frac {\partial \widehat{w}}{\partial t} - \Delta \widehat{w} =  c^2 \beta \widehat{v}_{xx},
\end{split}
\end{align}
получени чрез параболизиране на системата \rf{eq45}, а решението им схожда към численото такова на \rf{eq45}. Задачата \rf{eq5} е решена числено в първи квадрант в мрежата $\omega_h \subset \Omega_h$, поради симетрията на решението спрямо абцисата и ординатата. Използвани са два варианта за начални стойности, с които започва първата итерация: ``best-fit'' апроксимационните формули от \cite{ref15} и числено решение на задачата \rf{eq45} при $c=0$, известно като ``Основно състояние'' (Ground state) и описано в \cite{ref1c0, ref2c0}. С първия тип начални данни решението схожда между $5-10\%$ по-бързо за случаите, които са разгледани в тази работа. Добавен е адаптивен метод за определяне на стъпката по фиктивното време, което ускорява процеса на схождане. Разписани са ясни критерии за спиране на итерационния алгоритъм. В последствие са дефинирани алгоритмичните стъпки, с които може да се възпроизведе целият итерационен процес и е изведена алгоритмичната сложност. В Таблица \ref{tab:a} е представен редът на сходимост за метода на простата итерация.

\begin{table}[ht]
\begin{small}
\centering
		\begin{tabular}{||c|l|ll|ll||}
			\hline
			\hline
      \multirow{2  }{*}{ }        & \multirow{2  }{*}{$h$}  &  	\multirow{2  }{*}{ $\Vert \bar{ E_i} \Vert_{L_2}$ }	&Ред на	& \multirow{2  }{*}{ $\Vert \bar{ E_i} \Vert_{L_\infty}$ } 		&Ред на   \\
	                                        &                                                & 							 					&  сход. 	& 								       					& сход. \\
   					\hline 
					\hline 

$\beta = 3$ 	&0.2    										&            &            &           &   \\
      c=0.45 	&0.1    & 0.014232  						&            & 0.016732 			&   \\
   $O(h^2)$     &0.05   & 0.003238  						&2.14  & 0.003997					& 2.07 \\
\hline 
$\beta = 3$   	&0.2   &            &            &             &    \\
      $c=0.45 $ &0.1   &   0.001758   &           &  0.002499  &   \\
       $O(h^4)$	&0.05  &  0.000114 & 3.95    & 0.000168  & 3.90  \\
\hline
$\beta = 3$   	&0.2   &            &        &                  &      \\
   $c=0.45$   	&0.1   &  0.005038 &           & 0.012462       &       \\
     $O(h^6)$	&0.05  &  0.000094  & 5.74  &  0.000323 & 5.27         \\
			\hline
			\hline 	
$\beta = 1$   	&0.4   &             &           &                & \\
     $c=0.9$     &0.2   &  0.043898  &             & 0.017906      &    \\
     $O(h^2)$	&0.1  & 0.009999 & 2.13       & 0.004348      & 2.04  \\
\hline 	
 $\beta = 1$   	&0.4  &            &               &               &     \\
     $c=0.9$  	&0.2   & 0.006309  &              & 0.002965      &        \\
     $O(h^4)$	&0.1  &  0.000432 &3.87        & 0.000200 &  3.89        \\
    \hline
 $\beta = 1$	&0.4   &             &        &               &        \\
   $ c=0.9$  	&0.2   &  0.000088  &        & 0.000115      &       \\
       $O(h^6)$	&0.1  &   0.000002 &5.35  & 0.000003 &   5.04       \\
	   \hline
			\hline 
		\end{tabular}
		\caption{\normalsize Ред на сходимост при метода на простата итерация с апроксимации $O(h^{2})$, $O(h^{4})$ и $O(h^{6})$ на уравнението за Тест  1 и Тест 2. Грешките на численото решение $E_i$ са пресметнати в $L_2$ и $L_\infty$ норми.}
\label{tab:a}
\end{small}
\end{table}
\FloatBarrier

\textbf{Част 3.2} Числени резултати от метода на простата итерация за решаването на стационарното Парадигматично уравнение на Бусинеск

В този подраздел са направени два числени теста при параметри $\beta = \beta_1/\beta_2 = 3, c=0.45$ и $\beta = 1, c=0.9$. За тези примери са показани следните резултати. 

Първо, в Таблица \ref{tab:a} е обърнато внимание на скоростта на сходимост, върху три вложени мрежи, по правилото на Рунге \rf{Runge}, която отговаря на реда на избраната апроксимация на вторите производни. При $p=6$ резултатите за сходимостта в $L_2$ и $L_\infty$ норми са по-малки от очакваното, но са по-големи от 5, което не може да се постигне с крайни разлики от четвърти ред. 

Второ резидуумът
\begin{equation*}\label{residual}
R_{i,j}^{(k)} := 
c^2\beta (\widehat{v}^{(k)}_{i,j})_{\widehat{yy},p} + \Delta_{h,p}(-\beta \widehat{v}^{(k)}_{i,j} - c^2\beta (\widehat{v}^{(k)}_{i,j})_{\widehat{yy},p} + \Delta_{h,p} \widehat{v}^{(k)}_{i,j} 
+ \alpha \beta \theta (\widehat{v}^{(k)}_{i,j})^2  ),
\end{equation*}
който е сума от всички членове в ПУБ, намалява от стойности близки до $1e-1$ в първата итерация до стойности под $1e-9$ на последната итерация. 

Трето, показано е, че четвъртите производни на решението, които са пресметнати числено, са сходящи по правилото на Рунге \rf{Runge}. 

Четвърто, получени са решения за различни стойности на параметрите $\beta$ и $c$. Това параметрично изследване потвърждава получените в \cite{ref15, ref116} резултати като например зависимостта на формата на вълната от скоростта $c$ и дисперсията $\beta$ и асимптотичното поведение на решението при $(x^2 + y^2) \rightarrow \infty$. 

Най-накрая, решението от последната итерация при $p=6$ е сравнено с ``best-fit'' апроксимационните формули от \cite{ref15}. 
\begin{table}[ht]
\begin{small}
\centering
\begin{tabular}{|l|c|l l|l l|}
\hline
\hline 
$\beta$	& c 	& $\|v^*-v \|_{L_2 }$ & $\|v^*-v \|_{L_\infty }$  	& $D^*_{L_2}$	& $D^*_{L_\infty }$	\\
\hline 
1& 		0.1	&	1.4772e-01 		& 	8.1024e-03 				& 2.7\%			& 1.7\%		\\
\hline 
1& 		0.3 	&	1.4310e-01 		& 	8.7770e-03				& 2.7\%			& 1.9\%		\\
\hline 
1& 		0.5 	&	1.6934e-01 		& 	1.3332e-02				& 3.5\%			& 3.3\%		\\
\hline 
1& 		0.7 	&	5.8673e-01		& 	5.1122e-02				& 14.8\%		& 16.2\%	\\
\hline 
1& 		0.9	&	2.1599e+0 		& 	1.6439e-01				& 93.1\%		& 121.2\%	\\
\hline 
\hline 
\end{tabular}
\caption{Разлики между числено решение $v$ на елиптичната задача с висок ред на апроксимация $O(h^6)$ и ``best-fit'' формулите $v^*$ от \cite{ref15} при $\beta=1$ и $c=0.1, 0.3, 0.5, 0.7, 0.9$.}
\label{tab:diff-beta1}
\end{small}
\end{table}
Резултатите от него са показани на Таблица \ref{tab:diff-beta1}, където 
$D^*_{\kappa}$ дефинира относителната грешка в проценти между двете решения: $v$ на елиптичната задача с висок ред на апроксимация $O(h^6)$ и ``best-fit'' формулите $v^*$, в ${L_2 }$ и ${L_\infty}$ норми, върху цялата област $\Omega_h$:
\be\label{diffvv}
D^*_{\kappa} := 100 \times \frac{\Vert v^*-v \Vert_{\kappa} }{ \Vert v \Vert_{\kappa} }, \; \kappa=2; \; \kappa=\infty.
\ee
Това се оказва съществен елемент и повратна точка в изследването на хиперболичната задача \rf{eq1}-\rf{eq11}. \\

\textbf{Част 3.3} Ново гранично условие за двумерното елиптично уравнение \rf{eq3}
Изведено е едно ново асимптотично гранично условие (виж \cite{bnd}) за решаването на стационарното елиптично уравнение на Бусинеск \rf{eq3}. В серия от числени експерименти при $\beta=1,3,5$, $c=0.17$ и при $\beta=1$, $c=0.1, 0.5, 0.9$ е разгледана асимптотиката на всеки един от членовете за достатъчно голям радиус $r=\sqrt{x^2 + y^2} \gg 0$ (виж Фигура \ref{fig:assympt_c017bt1}). Показано е, че асимптотиката на членовете с четвърти производни и нелинейните такива: 
$$- v_{xxxx}, \;  - (2-\beta c^2)v_{xxyy},  \;  - (1-\beta c^2)v_{yyyy}, \;  - \alpha \beta (v^2)_{xx}, \; - \alpha \beta (v^2)_{yy}$$
е от ред $O(r^{-6})$, а на членовете с втори производни 
$$\beta v_{xx}, \; \beta (1-c^2) v_{yy}$$
 е от ред $O(r^{-4})$. Използвайки тази зависимост, членовете с асимптотика $O(r^{-6})$ в елиптичното уравнение са пренебрегнати. Това води до извеждането на явната формула за гранично условие
\be\label{bndv}
\tilde v(x, y) = \mu_v \frac{ (1-c^2) x^2 - y^2 }{ ((1-c^2) x^2 + y^2)^2 }, \quad x^2+y^2 \gg 0.
\ee
Последната формула е валидирана с помощта на два числени теста. 

При първия експеримент разглеждаме поведението на решението по границата, когато размерите на дискретната област $\Omega_h$ нарастват - $L_x = L_y = 20, 40, 80, 160$. Показано е, че функцията $\widehat v(0,L_y)$ намалява с темп от вида $1/r^2$, а разликите между численото решение $\widehat v$ и граничната функция $\tilde v$ намаляват при по-големи размери $L_x, L_y$. Направено е сравнение между числените решения $v_{WithB}$ с гранично условие от \rf{bndv} и $v_{ZeroB}$ с нулево такова за елиптичната задача \rf{eq45}, при което се вижда, че $L_2$ нормата от разликата намалява два пъти, а $L_\infty$ нормата четири пъти, при двойно увеличение на областта. 

При втория тест разглеждаме поведението на функцията $\widehat v$ и нейната асимптотика по $x,y$ осите, където размерът на областта $\omega_h = [0, 50] \times [0, 50]$ е фиксиран. Показано е, че численото решение следва наклона на функцията $1/r^2$ за достатъчно големи $r \gg 1$, което е в съответствие с новото асимптотично гранично условие от \rf{bndv} и асимптотиката намерена в \cite{ref116}.\\
\begin{figure}[ht]
	\begin{minipage}[b]{0.85\linewidth}
		\includegraphics[width=\linewidth]{AssymptForEachTerm/c017_bt1_5/ChristovIC_AlongX_50_ZB2_bt1_c017_h020_O(h^6).png}
	\end{minipage}
	\begin{minipage}[b]{0.85\linewidth}
		\includegraphics[width=\linewidth]{AssymptForEachTerm/c017_bt1_5/ChristovIC_AlongY_50_ZB2_bt1_c017_h020_O(h^6).png}
	\end{minipage}
	\caption{Асимптотично поведение на членовете от елиптичното уравнение, представени в абсолютна стойност и логаритмични скали по $x,y$, получено с апроксимация от шести ред и нулево гранично условие. Скоростта и дисперсионният параметър са $\boldsymbol{c=0.17}$ и $\boldsymbol{\beta = 1}$. }
	\label{fig:assympt_c017bt1}
\end{figure}
\FloatBarrier

\newpage
\textbf{Глава 4 Формулировка и числено решаване на двумерното Парадигматично уравнение на Бусинеск} 

За удобство правим следната смяна на променливите (виж \cite{ref25}):

\be\label{vcHyp}
x = \sqrt{\beta_1} \bar{x}, \quad y = \sqrt{\beta_1} \bar{y}, \quad t = \sqrt{\beta_1} \bar{t} ,
\ee
която променя основното уравнение \rf{eq1} в
\be\label{problemVC}
 \beta (E_1-\Delta) \frac{\partial^2}{\partial t^2}u= 
(I-\Delta)\Delta u +\Delta( (\beta - 1 )u - \alpha \beta u^2 )
\ee
където $\beta = \beta_1/\beta_2$, а $E_1$ е тъждественият оператор. За улеснение навсякъде по-надолу в текста ще се използват отново старите означения $x,y,t$ вместо ${\overline x},{\overline y},{\overline t}$.

Непрекъснатата хиперболична задача \rf{problemVC} притежава три инварианти, които се запазват във времето, както е показано в \cite{ref1}. В настоящата работа са разгледани само две от тях: масата и енергията, дефинирани съответно както следва:
\begin{align}\label{intM}
M(u(x,y,t)):=\int_{\RR^2} u(x,y,t)dx dy,
\end{align}
\begin{align}\label{ex-en}
&E(u(x,y,t)):= \nonumber\\ 
&\beta \int_{\RR^2} u_t(x,y,t) \: \left((-\Delta )^{-1}+E_1\right)u_t(x,y,t) dxdy+
\beta \int_{\RR^2} u^2(x,y,t) dxdy \nonumber\\
&-\int_{\RR^2}u(x,y,t) \left(\Delta u(x,y,t)\right) dxdy - \frac{2 \alpha \beta}{3} \int_{\RR^2} u^3(x,y,t) dxdy.
\end{align}

\textbf{Част 4.1} Дискретен закон за запазване на енергията на Консервативната схема

Консервативната схема има вида:
\begin{align}\label{consFDS}
\beta (E_1-\Delta_{h,2})&\frac{ u^{(k+1)}_{i, j} - 2u^{(k)}_{i,j} + u^{(k-1)}_{i,j} }{\tau^2} =  \nonumber \\
&=(\Delta_{h,2} - \Delta_{h,2}^2)u^{(k)}_{i,j} + \Delta_{h,2}(g(u^{(k)}_{i,j})),
\end{align}
където:
\begin{align}
g(u^{(k)}_{i,j})=& -\frac{\alpha \beta} { 3 } \left( (u^{(k+1)}_{i,j})^2 + (u^{(k-1)}_{i,j})(u^{(k+1)}_{i,j}) + (u^{(k-1)}_{i,j})^2 \right) + \nonumber\\
+&\frac{ (\beta - 1 )}{ 2 }\left( u^{(k+1)}_{i,j} + u^{(k-1)}_{i,j} \right).
\end{align}
При извеждането на \rf{consFDS} са следвани стъпките описани в \cite{ref25, ref999, ref1000}, с уточнението че апроксимацията на нелинейния член е различна и схемата е направена за уравнението \rf{problemVC} след смяната на променливите.

За Консервативната схема е изведена дискретната енергията:
\begin{align}\label{en_norm}
E_h(&v^{(k)})=\left< \left( \beta (E_1+(-\Delta_h)^{-1})- \frac{\tau^2}{4}( E_1-\Delta_h ) \right)v_{t}^{(k)} ,v_{t}^{(k)} \right>+ \nonumber\\
&+\frac{1}{4}  \left<  ( E_1-\Delta_h)(v^{(k+1)}+v^{(k)}), v^{(k+1)}+v^{(k)} \right> - \nonumber\\
&- \frac{\alpha \beta}{3} \left< ((v^{(k+1)})^3,1)+((v^{(k)})^3,1) \right> + \nonumber\\
&+\frac{\beta - 1}{2} \left< \left( (v^{(k+1)})^2+(v^{(k)})^2 \right), 1 \right>.
\end{align}
и в Теорема 1. е показано, че тя е константа величина във всеки един момент от време $t=\tau k$. В Теорема 2. е изведено условието за устойчивост
\be
\tau^2 < \frac{ 4 \beta h^2 } { 8 + h^2}, \; \beta \ge 1,
\ee 
което важи само за линейната част съответстваща на консервативното диференчно уравнение.
\\

\textbf{Част 4.2} Свойства на Консервативна схема

Показана е сходимостта на численото решение по правилото на Рунге \rf{Runge} върху три вложени мрежи, която варира между $1.41$ и $2.14$ за различните параметри $\beta$ и $c$ и разгледаните норми $L_2$ и $L_\infty$. Развитие във времето за дискретната енергия \rf{en_norm} е показано на Фигура \rf{EnOnly}. Скоростта ѝ на сходимост по правилото на Рунге \rf{Runge} варира между $2.00$ и $2.52$ за различните параметри $\beta$ и $c$ и разгледаните норми $L_2$ и $L_\infty$.
\begin{figure}[ht]
	\begin{minipage}[b]{0.49\linewidth}
		\includegraphics[width=\linewidth]{../amitans/figures/Energy_EnergySave_bt3_c045_x3O.png}	
	\end{minipage}
	\begin{minipage}[b]{0.49\linewidth}
		 \includegraphics[width=\linewidth]{../amitans/figures/Energy_EnergySave_bt1_c090_x3O.png}
	\end{minipage}
\caption{Дискретната енергия на решението от Консервативната схема при Тест 1 (ляво) и Тест 2 (дясно) с апроксимационна грешка $O(|h|^2 + \tau^2)$ във времето $T_{\tau} = [0, 10]$.}
\label{EnOnly}
\end{figure}
За изчислението на члена $-\Delta_{h,2}^{-1}v_{t}^{(k)}$ от \rf{en_norm} се използват бързите директни методи за обръщане на дискретния оператор на Лаплас, дефинирани в Част 2.6.\\

\textbf{Част 4.3} Метод на Тейлор и метод на правите за хиперболичната задача

За хиперболичното частно диференциално уравнение \rf{problemVC} се прави дискретизация по пространството, което води до следната система от $N_x \times N_y$ обикновени диференциални уравнения, както сме показали в \cite{refHyp}:
\begin{align} \label{DiscreteEq}
\beta (E_1-\Delta_{h,p}) \frac{\partial^2 }{\partial t^2}u_{i, j}(t)= \nonumber \\
 (E_1 - \Delta_{h,p})\Delta_{h,p} u_{i, j}(t) + \Delta_{h,p} ( -\beta f( u_{i, j}(t) ) + (\beta-1) u_{i, j}(t) ), \nonumber \\
i = 0,\cdots, N_x-1, \: j = 0 ,\cdots , N_y-1,
\end{align}
където $u_{i,j}(t)$ е приближение на решението в точката $(x_i, y_j) \in \Omega_h$, а $\Delta_{h,p}, p=2,4,6$ е дискретният оператор на Лаплас, апроксимиран с крайни разлики от втори, четвърти и шести ред.
Броят на уравненията в системата $N_x \times N_y$ е равен на броят точки в мрежата $\Omega_h$ и варира между $15 625$ и $716 800$ за разгледаните числени примери в дисертацията. Получената система \rf{DiscreteEq} е решена числено, използвайки развитие в ред на Тейлор, спрямо времевата променлива $t$:
\begin{align} \label{TSe}
u_{i, j}(t+\tau) = u_{i, j}(t) + \tau \frac{ \partial }{ \partial t }u_{i, j}(t)  + ... 
\frac{ \tau^p }{ p! } \frac{ \partial^p}{ \partial t^p }u_{i, j}(t) + O(\tau^{p+1})
\end{align}
при $p \ge 2$. 

Всяка една от производните по времето в \rf{TSe} изчисляваме посредством диференциране на \rf{DiscreteEq} по времето, като допускаме, че решението е достатъчно гладко $u \in C^{p+1,p+1,p+1}(\Omega \times T)$. За да получим съответната производна в явен вид, трябва да обърнем оператора $(E_1-\Delta_{h,p})$ в \rf{DiscreteEq}. За целта използваме така наречените ``Fast Poisson Solvers'', които са описани в Част 2.6. Редът на апроксимация по времето зависи от броя на членовете $p+1$, които участват в реда на Тейлор. По този начин редът на апроксимация при метода на Тейлор по времето и пространството е еднакъв $O(|h|^p + \tau^p)$, и зависи от избора на $p=2,4,6$.

При $t=0$ първите два члена в \rf{TSe} са известни от началното условие ($u_0$, $u_1$), а третият се получава чрез дискретното уравнение \rf{DiscreteEq}.  Следващият член в реда на Тейлор \rf{TSe} - третата производна по времето изглежда по следния начин:
\begin{align*} \label{der3}
 &\frac{\partial^3}{\partial t^3}u_{i,j}(t) = \frac{1}{\beta}\Delta_{h,p} \frac{\partial}{\partial t}u_{i, j}(t) + \nonumber\\
&+ \frac{1}{\beta} (E_1-\Delta_{h,p})^{-1}\Delta_{h,p} \left( -2 \alpha \beta \: u_{i, j}(t) \frac{\partial}{\partial t}u_{i, j}(t) +  (\beta-1) \frac{\partial}{\partial t} u_{i, j}(t) \right).
\end{align*}
Пресмятането на производните е итерационен процес, като производната от ред $s$, където $s \le p$, зависи от вече изчислените производни от ред $0,..,s-2.$ \\

\textbf{Част 4.4} Числени резултати от Консервативната схема и метода на Тейлор

В тази част са представени резултати от имплементираните методи за хиперболичната задача.
\begin{table}[ht]
\begin{small}
\centering
\small
		\begin{tabular}{||c|l|ll|ll||}
			\hline
			\hline


      \multirow{2  }{*}{$p=2,4,6$}        & \multirow{2  }{*}{$h$, $\tau$}  &	\multirow{2  }{*}{  $\Vert \bar E_i \Vert_{L_2} $ } 	&Ред на & \multirow{2  }{*}{  $\Vert \bar E_i \Vert_{L_\infty}$ }	&Ред на   \\
	                                        &                                                &    										&  сход. & 										& сход. \\
\hline 
\hline 
  $\beta=3$               &0.2, 0.1      &              	&           &                	&      \\
   c=0.45                   &0.1, 0.05    &0.968044  	&           &1.034208   &       \\
 $O(h^2 + \tau^ 2)$ 	&0.05, 0.025	& 0.340955 	& 1.51    &0.351518  	&  1.56      \\
\hline 
  $\beta=3$               &0.2, 0.1      &              	&          	&                 &      \\
   c=0.45                   &0.1, 0.05    &0.191389 	&          	&0.194056   	&       \\
$O(h^4+ \tau^4)$	&0.05, 0.025	&0.013036 	& 3.88   	&0.013664   	& 3.83       \\
\hline 
  $\beta=3$               &0.2, 0.01    &                	&          	&                 &      \\
     c=0.45                 &0.1, 0.05    &0.032671 	&          	& 0.033625  	&       \\
  $O(h^6+ \tau^6)$ 	&0.05, 0.025	&0.000599 	&5.78    	& 0.000635  	& 5.73       \\
\hline
\hline 
       $\beta=1$        	&0.4, 0.2      &             	&            &           &   \\
           c=0.9    		&0.2, 0.1      &  0.148014 	&            &0.058905 &   \\
  $O(h^2+ \tau^2)$  	&0.1, 0.05   	& 0.030690  	&2.27  	 &0.014185   & 2.05 \\
\hline
      $\beta=1$           &0.4, 0.2    	&            	&               &             &    \\
       c=0.9                 &0.2, 0.1     & 0.028869   &        &  0.013791   &   \\
 $O(h^4+ \tau^4)$ 	&0.1, 0.05   	&0.001860 	& 3.96  & 0.000996  & 3.80  \\
\hline
  $\beta=1$     		&0.4, 0.2   	&            	&          	&                  &      \\
      c=0.9                  &0.2, 0.1   	&0.006732 	&            & 0.003334      &       \\
 $O(h^6+ \tau^6)$ 	&0.1, 0.05 	& 0.000249 	& 4.75 	& 0.000068  & 5.61        \\
\hline
\hline 
		\end{tabular}
		\caption{Скорост на сходимост на численото решение при метода на Тейлор с нулево гранично условие и грешки от апроксимацията $O(|h|^{2} + \tau^2 )$, $O(|h|^{4} + \tau^4 )$ и $O(|h|^{6} + \tau^6 )$. Грешките $\bar E_i$ са измерени в $L_2$ и $L_\infty$ норми.}
\label{tableA}
\end{small}
\end{table}
Редът на сходимост на дискретните решение и енергия, получени от метода на Тейлор, по правилото на Рунге \rf{Runge} са показани съответно на Таблици \ref{tableA} и \ref{tableB}. Редът на сходимост отговаря на приложения ред на апроксимация. Само при Тест 2 и $O(|h|^6 + \tau^6)$, сходимостта при енергията е по-ниска от очакваното. Този резултат е поради високия градиент на решението, близо до границата и по-ниската сходимост, получена още при началното условие от решаване на елиптичната задача (виж последният ред в Таблица \ref{tab:a}).
\begin{table}[ht]
\begin{small}
\centering
\small
		\begin{tabular}{||c|l|ll|ll||}
			\hline
			\hline
      \multirow{2  }{*}{FDS}        & \multirow{2  }{*}{$h$, $\tau$}  &  	\multirow{2  }{*}{ $\Vert \bar{\bar{ E_i}} \Vert_{L_2}$ }	&Ред на	& \multirow{2  }{*}{ $\Vert \bar{\bar{ E_i}} \Vert_{L_\infty}$ } 		&Ред на   \\
	                                        &                                                & 							 					&  сход. 	& 								       					& сход. \\
   			\hline 
					\hline 
  $\beta=3$             	&0.2, 0.1         	&              	&          	&                     &      \\
   c=0.45                 	&0.1, 0.05         	&0.379690  	&          	&0.446557 		&       \\
 $O(h^2 + \tau^ 2)$ 	&0.05, 0.025  	& 0.075257 	& 2.33  	& 0.111797     	& 2.00      \\
\hline 
  $\beta=3$               &0.2, 0.01       	&                	&          	&                     	&      \\
   c=0.45                   &0.1, 0.005      	&0.044788 	&         	&0.082344   		&       \\
 $O(h^4+ \tau^4)$ &0.05, 0.025  		&0.002459 	&4.18    	&0.003562   		&4.53      \\
\hline 
  $\beta=3$               &0.2, 0.01       	&                	&          	&                 	&            \\
     c=0.45                 &0.1, 0.05        	&0.063681 	&          	&  0.133517  	&           \\
 $O(h^6+ \tau^6)$ &0.05, 0.025 		&0.001086 	& 5.87   	&  0.002041		&6.03   \\
\hline
\hline 
       $\beta=1$       	&0.4, 0.2     		&             	&           &           		&   \\
                  c=0.9    	&0.2, 0.1     		& 0.404499 	&           &0.348428 		&   \\
$O(h^2+ \tau^2)$ 	&0.1, 0.05   		& 0.081033  	&2.32  	&0.087663  		& 2.00 \\
\hline
      $\beta=1$           &0.4, 0.2    		&            	&        	&             		&    \\
       c=0.9                	&0.2, 0.1     		& 0.064165  	&        	&  0.056574   	&   \\
$O(h^4+ \tau^4)$ 	&0.1, 0.05   		&0.005141 	& 3.64  	& 0.005173  		& 3.45  \\
\hline
  $\beta=1$     	 	&0.4, 0.2   		&            	&         	&                  	&      \\
      c=0.9                 	&0.2, 0.1   		&0.037448	&         	& 0.054236      	&       \\
   $O(h^6+ \tau^6)$ &0.1, 0.05  		& 0.003098 	& 3.60 	& 0.003293  		& 4.04        \\
\hline
\hline 
		\end{tabular}
		\caption{Скорост на сходимост на дискретната Енергия при метода на Тейлор с нулево гранично условие и грешки от апроксимацията $O(|h|^{2} + \tau^2 )$, $O(|h|^{4} + \tau^4 )$ и $O(|h|^{6} + \tau^6 )$. Грешките $\bar{\bar E_i}$ са измерени в $L_2$ и $L_\infty$ норми.}
\label{tableB}
\end{small}
\end{table}
\FloatBarrier

\begin{figure}[ht]\vspace{0.2cm}
	\begin{minipage}[b]{0.51\linewidth}
		%\includegraphics[width=\linewidth]{../amitans/figures/Energy_bt3_c045_h005_Taylor_Conservative.png}	
		\includegraphics[width=\linewidth]{Energy/Energy_bt3_c045_h005_Taylor_Conservative.png}	
	\end{minipage}
	\begin{minipage}[b]{0.51\linewidth}
		%\includegraphics[width=\linewidth]{../amitans/figures/Energy_bt1_c090_h010_Taylor_Conservative.png}		
		\includegraphics[width=\linewidth]{Energy/Energy_bt1_c090_h010_Taylor_Conservative.png}				
	\end{minipage}
\caption{Дискретнa eнергия на решението като функция на времето до $T=10$ при Тест 1 (ляво) и Тест 2 (дясно).}
\label{EnTest1Test2}
\FloatBarrier
\end{figure}
Направени са допълнителни изчисления върху области с по-големи размери и по-голяма стъпка $h$ ($h=0.2$ при Тест 1 и $h=0.4$ при Тест 2), за да се обясни нарастването при масата, като резултатите от изследванията са показани на Фигура \rf{Test1_2Mass}. Видно е, че стойността ѝ се запазва по-добре, когато задачата се пресмята в по-големи области $\Omega_h$. 
\begin{figure}[ht]\vspace{0.2cm}
	\begin{minipage}[b]{0.51\linewidth}
		\includegraphics[width=\linewidth]{Mass/MassTaylor_120_60_30_ZB1_bt3_c045_h020_O(h^6).png}
	\end{minipage}	
	\begin{minipage}[b]{0.51\linewidth}
		\includegraphics[width=\linewidth]{Mass/MassTaylor_512_256_128_ZB1_bt1_c090_h040_O(h^6).png}		
	\end{minipage}
\caption{Масата от решението от метода Тейлор с $O(|h|^6 + \tau^6)$ апроксимация като функция на времето до $T=10$ върху три различни по големина области. Левият панел е при $\beta =  3$, $c = 0.45$, а десният панел е при $\beta =  1$, $c = 0.9$.}
\label{Test1_2Mass}
\end{figure}
\FloatBarrier
Методът на Тейлор не запазва безусловно енергията, затова тя заедно с масата и формата на решението (при $O(|h|^2 + \tau^2)$) са сравнени с тези, получени при Консервативната схема. Представен е анализ на зависимостта на формата и максимума на решението от реда на апроксимация и големината на дискретните стъпки по пространството и времето - $h, \tau$. Резултатите показват, че с намаляване на $h$ и $\tau$ и увеличаване реда на апроксимация, разликата в профила на вълната между началния и крайния момент намалява. 

Най-накрая е разгледан случай при по-голям времеви интервал $T=30$, за който отново са описани поведението на формата и максимума, като е показано че тези две свойства на решението се запазват с пренебрежимо малки отклонения.\\

\textbf{Част 4.5} Числен тест за хиперболичната задача при параметри $\beta = 3$ и $c=0.3$

Настоящата част разглежда един добре познат в литературата случай за хиперболичната задача (виж \cite{ref21, ref20, ref23, ref22} ), при който за начално условие в споменатите трудове винаги са използвани ``best-fit'' апроксимационните формули от \cite{ref15}. В настоящата работа при $t=0$ ще бъде използвано решението от стационарното уравнение на Бусинеск (еквивалентно на системата \rf{eq45}) с апроксимации от висок ред и  ``Симетрично гранично условие по абцисата и ординатата при елиптичната задача'', както е дефинирано в Част 2.3. Големината на областта $\omega_h$ е $L_x = L_y = 50$, $\alpha = 1$, а използваните крайни разлики са с четвърти и шести ред на апроксимация (при $p=4, 6$). Поради симетрията на решението по абцисата и ординатата е лесно да се построи началната функция в цялата област $\Omega_h$. 
\begin{table}[ht]
\begin{small}
\centering
\small
		\begin{tabular}{||c|l|l|l||}
			\hline
			\hline
  $p=4,6$   &  $h, \tau$ &  $||u^{(0)} - u^{(N_t)}||_{L_2}$  & $||u^{(0)} - u^{(N_t)}||_{L_\infty}$   \\
   		      \hline 
			\hline
           				& $0.4, 0.2$   &  3.020072 & 2.571611     \\
			\hline 
  $O(|h|^4+\tau^4)$ & $0.2, 0.1$   & 0.365473 & 0.371433      \\
			\hline 
           				& $0.1, 0.05$ & 0.025631 & 0.026517      \\
	   \hline
          \hline
           				& $0.4, 0.2$   & 1.939208 & 1.673675      \\
			\hline
  $O(|h|^6+\tau^6)$ & $0.2, 0.1$   & 0.037638 & 0.038985      \\
    \hline
           				& $0.1, 0.05$  & 0.000655 & 0.000679       \\
	   \hline
		\hline 
		\end{tabular}
		\caption{Разлика $||u^{(0)} - u^{(N_t)}||_\kappa$ в $L_2$ и $L_\infty$ норми между числените решения в началото при $t=0$ и в края при $t=10$. За резултатите в таблицата е използван методът на Тейлор с четвърти и шести ред на апроксимация. }
\label{tableK}
\end{small}
\end{table}
\FloatBarrier
\begin{figure}
	\centering
	\includegraphics[width=0.70\linewidth]{Maximum_TaylorZeroBnd_50x50_bt3_c030.png}
\caption{Развитие на максимума на решението в интервала $[0, 10]$ при $\beta = 3$, $c=0.3$. За резултатите от графиката е използван метода на Тейлор с четвърти и шести ред на апроксимация и дискретните стъпки $h=0.4, 0.2, 0.1, \tau = h/2$.}
\label{MultiMaximum}
\end{figure}
\begin{figure}
	\centering
	\includegraphics[width=\linewidth]{SolutionView/TaylorZeroBnd_50_ZB2_bt3_c030_h005_TAll_O(h^6).png}
\caption{Развитие на формата на численото решение при $\beta = 3$, $c=0.3$. Използван е метод на Тейлор с апроксимация от шести ред и следните дискретни стъпки: $h=0.05, \tau = 0.025$. По вертикалата е $y$ оста, а по хоризонталата $x$ оста.}
\label{solShape3}
\end{figure}
При хиперболичната задача (разгледана в $\Omega_h$) е използван метод на Тейлор, отново с апроксимационни формули от четвърти и шести ред. Представен е анализ на зависимостта на формата и максимума на решението от реда на апроксимация и големината на дискретните стъпки по пространството и времето - $h, \tau$ - за метода на Тейлор (виж Фигура \ref{MultiMaximum} и Таблица \ref{tableK}). 

Основна характеристика на солитоните е запазването на формата им по времето. В Таблица \ref{tableK} сме изследвали формата на решението в интервала $[0, 10]$ и показваме, че с намаляване на стъпките по пространството $h$ и времето $\tau$, разликата в профила на вълната между началния и крайния момент намалява. 
В допълнение се вижда, че формата на вълната се запазва по-добре, когато степента на апроксимация на диференциалните оператори е по-висока. 

Последните разсъждения затвърждават очакването, че разликата във формата на численото решение между началния и крайния моменти $||u^{(0)} - u^{(N_t)}||_\kappa$, за произволно фиксирано време $T$, клони към нула, когато $h$ и $\tau$ са много малки.

При този тест отново е разгледан случай при по-голям времеви интервал $T=30$, за който отново са описани поведението на формата и максимума, като е показано че тези две свойства на решението се запазват с пренебрежимо малки отклонения. Развитието на вълната до време $t=20$ не търпи видими структурни промени, както е показано на Фигура \ref{solShape3}.
В оригиналните променливи това време е $34.64$ (преди смяната \rf{vcHyp}). В сравнение с резултатите получени в \cite{ref20}, където за начално условие са използвани ``best-fit'' формулите, на Фигура 1 е разгледано детайлно развитието на формата на решението и неговия максимум. Там ясно се вижда, че вълната претърпява макар и слаби ($\approx 0.15$), но все пак видими структурни промени още в началото за времевия интервал $[0, 4]$ (в оригиналните променливи). Тук измененията във формата са под $7.0e-6$ за всяка една точка от мрежата в интервала $[0, \sqrt{\beta_1} 5] \approx [0, 8.66]$ (отново в оригиналните променливи).\\
\FloatBarrier

\section{Заключение}

За решаване на стационарното уравнение на Бусинеск са използвани диференчни схеми с втори, четвърти и шести ред на апроксимация на вторите производни заедно с метода на простата итерация. В Таблица \ref{tab:a} е показана скоростта на сходимост по правилото на Рунге върху три вложени мрежи, която в повечето случаи отговаря на използвания ред на апроксимация. Полученото числено решение има сходно поведение с това, описано в \cite{ref117,ref116} от гледна точка на формата и зависимостта ѝ от скоростта $c$ и дисперсионния параметър $\beta$. Получените резултати се използват за начално условие в ПУБ.

Изведено е ново асимптотично гранично условие за елиптичната задача, след като са изследвани асимптотиките на всички членове поотделно в стационарното елиптично уравнение на Бусинеск. Получената явна формула е валидирана в серия от числени експерименти при различни размери на областта $L_x=L_y=20,40,80,160$. Показано е, че численото решение се апроксимира по-добре, когато $L_x$ и $L_y$ са по-големи. Направени са изчисления с нулево гранично условие и резултатите са сравнени с новото такова, където се вижда, че $L_2$ нормата от разликата намалява два пъти, а $L_\infty$ нормата - четири пъти, при двойно увеличение на областта.

За хиперболичната задача са използвани два различни подхода - Консервативната схема и комбинацията от метод на Тейлор и метод на правите с втори, четвърти и шести ред на апроксимация, който се прилага за пръв път за ПУБ. Числените решения, заедно с неговите свойства като маса, енергия и форма, получени с метода на Тейлор с $O(|h|^2 + \tau^2)$ и Консервативната схема, са сравнени и е показано, че са доста близки. Една от целите на работата е да покаже, че методът на Тейлор, приложен към уравнението \rf{problemVC}, също води до достатъчно добри резултати, а също така може да се приложи с по-висок ред на апроксимация, което води и до по-фино решение. Числената енергия е запазена и в двата случая. При масата е показано, че се запазва по-добре върху по големи области. 

Формата и максимумът на вълната при Тест 3 се запазват за по-дълго време до $T=34.64$ в оригиналните променливи, когато се изполват апроксимации от по-висок шести ред и начално условие получено числено чрез решение на стационарното уравнение на Бусинеск. В допълнение, формата на решението се запазва по-добре спрямо получените резултати в \cite{ref21, ref20, ref23, ref22, ref24} за дадено фиксирано време $T$ (виж Таблица \ref{tableG} и Таблица \ref{tableK}) с намаляване на дискретните съпки по времето и пространството $h, \tau$ и увеличаване реда на апроксимация на метода.

Числените алгоритми за решенията на стационарното и Парадигматичното уравнения на Бусинеск могат да бъдат намерени и клонирани от следното интернет хранилище (repository):\\
https://github.com/CloakMe/Boussinesq.git

\section{Научни приноси}

Основните научни приноси на дисертацията са:
\begin{itemize}
  \item Разработили сме числени методи с четвърти и шести ред на апроксимация за стационарното и Парадигматичното уравнения на Бусинеск;
  \item Изследвали сме подробно асимптотиката на всеки един от членовете в елиптичното уравнение и сме изведели ново асимптотично гранично условие; 
  \item Подробно са сравнени численото решение на елиптичната задача с ``best-fit'' приближенията от \cite{ref15}. Показали сме, че началното условие, получено чрез итерационен метод с висок ред на апроксимация за решаване на стационарното уравнение, превъзхожда ``best-fit'' апроксимационните формули, когато се изследва солитонният характер на вълната в ПУБ, защото нейните форма и максимум се запазват за по-дълъг период от време; 
  \item Числените експерименти демонстрират, че тези две свойства на решението се запазват по-добре (както при ниски $c=0.3$ така и при високи скорости $c=0.45,0.9$), когато се използват по-ситни дискретни стъпки и по-висок ред на апроксимация за фиксиран интервал от време $T$;
  \item Mетодът на Тейлор запазва енергията и масата толкова добре колкото и Консервативната схема при разгледаните числени тестове.
\end{itemize}

\section{Декларация}
Декларирам, че представената дисертация на тема: ``Числено изследване на двумерното уравнение на Бусинеск'' е мой труд. В нейното разработване не са ползвани разработки и чужди публикации в нарушение на авторските им права. 

Всички цитирания на източници на информация, текст и други са обозначени според стандартите.

Резултатите от дисертационното изследване са оригинални и не са взаимствани от източници, в които нямам участие.
\vspace{1cm}

Подпис: ...............

\newpage
\section{Благодарности и посвещение}

{\Large \it Искам сърдечно да благодаря на моя научен ръководител - проф. д-р Наталия Кольковска, за помощта и съветите, които ми даде. Продължителните дискусии, които водихме, оказаха съществено влияние върху работата по дисертацията.}

{\Large \it Благодаря на моето семейството и роднини за търпението и подкрепата, която ми оказваха през годините.}

{\Large \it Благодаря на колегите от секция ``Математическо моделиране и числен анализ'' за полезните съвети и насоки.}

{\Large \it Благодаря на Тодор Коларов, управител на Телеконт ЕООД и Димитър Димитров, управител на Диджитал Лайтс ЕООД, които ми позволиха да работя над този труд, чрез допълнителна платена отпуска.}

{\Large \it Благодаря на доц. д-р Иван Бажлеков, за предоставения достъп до неговия работен компютър, на който съм пускал изчисления в продължение на месеци.}

\vspace{4cm}

{\Large \it Въпреки че не познавам лично проф. Христо Христов, имах възможност да се запозная с  изследванията му върху Парадигматичното уравнение на Бусинеск. Посвещавам този труд на него. }

\newpage
\noindent\textbf{\Large Апробация на дисертационната работа}
\vspace{0.3cm}

Резултатите от дисертацията са докладвани на следните международни конференции и семинари:
\begin{itemize}
\item Scientific Seminars of IMI-BAS, Oct 2013;

\item Conference ``Mathematics days in Sofia'', 10 June 2014;

\item Conference ``BIOMATH'', 27 June 2014;

\item Conference ``Workshop on Approximation Theory, CAGD, Numerical Analysis and Symbolic Computation'',  Johannes Kepler University, Linz, September, 2015;

\item 13th International Conference for Promoting the Application of Mathematics in Technical and Natural Sciences (AMiTaNS’22), Albena, Bulgaria, 24–29 June 2021. 
\end{itemize}
\vspace{0.2cm}
\textbf{\Large Публикации}
\vspace{0.3cm}

Резултатите от дисертацията са публикувани както следва:
\begin{itemize}
\item N. Kolkovska, K. Angelow, Numerical computation of the critical energy constant for two dimensional Boussinesq equations, AIP Conference Proceedings, Appl. of Mathematics in Technical and Natural Sciences, 1684 (2015) , SJR = 0.18;

\item К. Angelow, N. Kolkovska, Numerical Study of Traveling Wave Solutions to 2D Boussinesq Equation, Serdica Journal of Computing, 13 (2019), 1-16;

\item K. Angelow, New Boundary Condition for the Two Dimensional Stationary Boussinesq Paradigm Equation, International Journal of Applied Mathematics, 32 (2019), 141-154, SJR = 0.27 (Q3, Scopus);

\item K. Angelow, Comparison Between Two Numerical Methods for Solution of 2D BPE, AIP Conference Proceedings, 2522 (2022), 1, SJR = 0.18.
\end{itemize}

\newpage
\begin{thebibliography}{99}

	\bibitem{ref01} Boussinesq, J., Theorie de l'intumescence liquide, applelee onnde solitaire ou de translation, se propageant dans un canal rectangulaire, {\it Comptes Rendus de l'Academie des Sciences}, \textbf{72} (1871), 755-759.

	\bibitem{ref02} Boussinesq, J., Theorie des ondes et des remous qui se propagent le long d'un canal rectangulaire horizontal, en communiquant au liquide contenu dans ce canal des vitesses sensiblement pareilles de la surface au fond.  {\it Journal de Mathematiques Pures et Appliquees}, \textbf{17} (1872), 55-108.

	\bibitem{ref1} Christov, C.I., An energy-consistent dispersive shallow-water model,  {\it Wave Motion}, \textbf{34} (2001), 161-174.

	\bibitem{ref159} Todorov, M. D., Nonlinear Waves: Theory, computer simulation, experiment, Two-dimensional Boussinesq equation. Boussinesq paradigm and soliton solutions, Morgan and Claypool Publishers, California, 2018

	\bibitem{ref15} Christov, C.I., Choudhury, J., Perturbation solution for the 2D Boussinesq equation, {\it Mech. Res. Commun.}, \textbf{38} (2011), 274-281.

	\bibitem{ref21} Chertok, A., Christov, C.I., Kurganov, A., Central-Upwind Schemes for the Boussinesq Paradigm Equations,
{\it Computational Science and High Performance Computing IV, Notes Numer. Fluid Mech.}, \textbf{113} (2011), 267-281.

	\bibitem{ref20} Christov, C.I., Kolkovska, N., Vasileva, D., On the Numerical Simulation of Un-steady Solutions for the 2D Boussinesq Paragigm Equation,
{\it In: I. Dimov, S. Dimova, N. Kolkovska (Eds.), Numerical Methods and Applications 2010},
\emph{Conference Proceedings}, \textbf{6046} (2010), 386–394.

	\bibitem{ref23} M.Dimova, D. Vasileva, Comparison of Two Numerical Approaches to Boussinesq Paradigm Equation, 
{\it Numerical Analysis and Its Applications. NAA 2012. Lecture Notes in Computer Science}, \textbf{8236}, (2013), 255-262.

	\bibitem{ref22} Kolkovska N., Angelow K., A Multicomponent Alternating Direction Method for Numerical Solving of Boussinesq Paradigm Equation,
In: {\it  I. Dimov, I., Farago, I., Vulkov, L. (eds.) NAA 2012},
\emph{Conference Proceedings}, \textbf{8236} (2013), 371–378.

	\bibitem{ref25} Kolkovska N., Two families of finite difference schemes for multidimensional Boussinesq paradigm equation, In:
{\it Applications of Mathematics in Technical and Natural Sciences,  Sozopol (Bulgaria)},
\emph{AIP Conference Proceedings}, \textbf{1301} (2010), 395.

	\bibitem{ref10000} Erbay H., Erbay S., Erkip A., Instability and stability properties of traveling waves for the double dispersion equation, {\it Nonlinear Analysis}, \textbf{133} (2016), 1-14.

	\bibitem{ref1c0} Kolkovska N. Angelow K., Numerical computation of the critical energy constant for two-dimensional Boussinesq equations,
{\it Application of Mathematics in Technical and Natural Sciences, Albena (Bulgaria)},
\emph{AIP Conference Proceedings}  \textbf{1684} (2015), 080007.

	\bibitem{ref14} Christou M. , Christov C.I.,
Fourier Galerkin method for 2D solitons of Boussinesq equation,
{\it Mathematics and Computers in Simulation} \textbf{74} (2007), 82-92.

	\bibitem{ref13} Christou M. , Christov C.I.,
Galerkin spectral method for the 2D solitary waves of Boussinesq paradigm equation,
In: {\it Applications of Mathematics in Technical and Natural Sciences, Sozopol (Bulgaria)},
\emph{AIP Conference Proceedings}, \textbf{1186}, Issue 1 (2009), 217-225.

	\bibitem{ref117} J. Choudhury, C.I. Christov,
2D solitary waves of Boussinesq equation, in ISIS International Symposium on Interdisciplinary Science, Natchitoches, October 6-8, 2004, {\it APS Conference Proceedings 755}, Washington, DC, 2005, pp 85-90.

	\bibitem{ref116} C.I. Christov,
Numerical implementation of the asymptotic boundary conditions for steadily propagating 2D solitons of Boussinesq type equations,
{\it Mathematics and Computers in Simulation}, \textbf{82} (2012), 1079-1092.

	\bibitem{ref200} Vasileva, D., Kolkovska, N., Investigation of two numerical schemes for the 2D Boussinesq paradigm equation in a moving frame coordinate system, {\it In: Mathematics in industry},
\emph{Cambridge Scholars}, (2014), 289–305.

	\bibitem{ref241} M.Dimova, N. Kolkovska, Comparison of some finite difference schemes for Boussinesq Paradigm Equation, 
{\it Int. Conf. Math. Model. Comput. Phys.}, \emph{Springer}, (2011), 215-220.

	\bibitem{ref251} Kolkovska N., Four-Level Conservative Finite-Difference Schemes for Boussinesq Paradigm Equation, {\it American Institute of Physics},
\emph{AIP Conference Proceedings}, \textbf{1561} (2013), 68-74.

	\bibitem{ref252} Kolkovska N., Error Estimates of Four Level Conservative Finite Difference Schemes for Multidimensional Boussinesq  Equation,  {\it Int. Conf. Finite Differ. Methods}, \emph{Springer} (2014), 266-273.

	\bibitem{ref253} Blinkov Y., Gerdt V., Pankratov I., Kotkova E., Construction of a New Implicit Difference Scheme for 2D Boussinesq Paradigm Equation, {\it Int. Workshop Comput. Algebra Sci. Comput.}, \emph{Springer}, (2019), 152-163.

	\bibitem{ref24} Yuyu He, Hongtao Chen, Efficient algorithm and convergence analysis of conservative SAV compact difference scheme for Boussinesq Paradigm equation, 
{\it Computers and Mathematics with Applications}, \textbf{125} (2022), 34-50.

	\bibitem{ref254} Vucheva V., Kolkovska N., High Order Symplectic Finite Difference Scheme for Double Dispersion Equations, {\it AIP Publishing LLC}, \emph{AIP Conference Proceedings}, \textbf{2321}, (2021), 030037

	\bibitem{bnd} Angelow K., New Boundary Condition for the Two Dimensional Stationary Boussinesq Paradigm Equation, 
{\it International Journal of Applied Mathematics}, \textbf{32,1}, (2019), 141-154.

	\bibitem{ref999} Kolkovksa, N., Convergence of FDS for a Multidimensional Boussinesq Equation, {\it Lecture Notes in Computer Science}, \textbf{6046} (2011), 469-476.

	\bibitem{ref1000} Kolkovksa, N., Dimova, M., A new conservative finite difference scheme for Boussinesq paradigm equation, {\it Central European Journal of Mathematics}, \textbf{10} (2012), 1159-1171.

	\bibitem{refHyp} Angelow K., Comparison between two numerical methods for solution of 2D Boussinesq paradigm equation, \emph{AIP Conference Proceedings}, \textbf{2522}, (2022), 090001

	\bibitem{ref34} T. Lyche, Fast Direct Solution of a Large Linear System,
{\it Numerical Linear Algebra and Matrix Factorizations, Texts in Computational Science and Engineering}, \textbf{22}, (2020), p 237-250

	\bibitem{forn} Fornberg, B., Generation of Finite Difference Formulas on Arbitrarily Spaced Grids, 
Math. Comput., 51(1988),  699 -- 706.

	%\bibitem{boole} Zucker, Ruth, "Chapter 25.4.14: Numerical Interpolation, Differentiation, and Integration - Integration - Numerical Analysis". In Abramowitz, Milton; Stegun, Irene Ann (eds.). Handbook of Mathematical Functions with Formulas, Graphs, and Mathematical Tables. {\it Applied Mathematics Series} \textbf{55}  (1983), ISBN 978-0-486-61272-0.

	%\bibitem{ref260} G. H. Golub and C. F. Van Loan, Matrix Computations, 3/e, Johns Hopkins University, Press, Baltimore, 1996.

	%\bibitem{ref27} Coppersmith Don, Winograd Shmuel., Matrix multiplication via arithmetic progressions, {\it  Journal of Symbolic Computation}, \textbf{9 (3)} (1990), 251, doi:10.1016/S0747-7171(08)80013-2.

	%\bibitem{ref26} Strassen, Volker, Gaussian Elimination is not Optimal, {\it Numer. Math.}, \textbf{13 (4)} (1969), 354–356, doi:10.1007/BF02165411. S2CID 121656251

	%\bibitem{samarski} Samarskii A., The Theory of Difference Schemes, Marcel Dekker Inc., New York, 2001.

	%\bibitem{ref1c00} McLeod, Kevin, James Serrin, Uniqueness of Solutions of Semilinear Poisson Equations, In {\it Proceedings of the National Academy of Sciences of the United States of America}, \textbf{78}, no. 11, (1981), 6592–95. JSTOR, http://www.jstor.org/stable/11173,

	\bibitem{ref2c0} Kolkovska N., Numerical Evaluation of 2D Ground States,
\emph{ EPJ Web of Conferences}, \textbf{108} (2016), 02032.

	%\bibitem{ref31} B. Engquist and A. Majda, Absorbing boundary conditions for the numerical simulation of waves, {\it Math. Comp.}, \textbf{31} (1977), 629–651.

	%\bibitem{ref32} C.I. Goldstein, A finite element method for solving Helmholtz type equations in waveguides and other unbounded domains,
{\it Math. Comp.}, \textbf{39} (1982), 309–324.

	%\bibitem{ref33} H. Han, W. Bao, Error estimates for the finite element approximation of problems in unbounded domains,
{\it SIAM J. Numer. Anal.} \textbf{37}, \textbf{4} (2000), 1101–1119

	\bibitem{ref16} Angelow K., Kolkovksa, N., Numercal Study of Traveling Wave Solutions to 2D Boussinesq Equation, {\it Serdica J. Computing}, \textbf{13} (2019), 1-16.

	%\bibitem{ref4} Christov, I., Christov, C.I., Physical dynamics of quasi-particles in nonlinear wave equations,
{\it Physics Letters A}, \textbf{372}, Issue 4 (2008),  841-848.

\end{thebibliography}

\end{large}
\end{document}
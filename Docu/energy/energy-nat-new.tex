\documentclass{article} 

%\usepackage{amsmath,amsthm,amsfonts}
\usepackage[english]{babel}
\usepackage[numbers]{natbib}
\usepackage{rotating}
\usepackage{amsfonts,amsmath}
\usepackage{graphicx}
\usepackage{color} 
\usepackage[notref,notcite]{showkeys}
\usepackage{multirow}

\newcommand{\be}{\begin{equation}}
\newcommand{\ee}{\end{equation}}
\newcommand{\rf}[1]{(\ref{#1})}
\newcommand{\RR}{\mathbb{R}}
\newtheorem{thm}{Theorem}
\newtheorem{lm}{Lemma}

\begin{document}
We study the problem

\be\label{problem}
\beta(E-\Delta) \frac{\partial^2 u}{\partial t^2}=
 \beta \Delta u -\Delta^2 u -\alpha \beta \Delta (u^2)
\ee


Define the operator $A$ as $Av=-\Delta_h v=-v_{\bar{x}x} - v_{\bar{y}y}$ and consider the FDS

\be\label{FDS1}
\beta (E+A)v_{\bar{t}t}^k +\beta Av^k+A^2 v^k -\alpha \beta A\left(\frac{(v^{k+1})^3-(v^{k-1})^3}{3(v^{k+1}-v^{k-1})} \right)=0
\ee

We multiply \rf{FDS1} by $A^{-1}$ and get 
\be\label{FDS2}
\beta (E+A^{-1})v_{\bar{t}t}^k +\beta v^k+A v^k -\alpha \beta \frac{(v^{k+1})^3-(v^{k-1})^3}{3(v^{k+1}-v^{k-1})} =0
\ee

We substitute $v^{k}=0.5(v^{k+1}+v^{k-1})-\frac{\tau^2}{2}v_{\bar{t}t}^k$ into \rf{FDS2}
and obtain
\begin{align*}
&\left( \beta (E+A^{-1})- \frac{\tau^2}{2}(\beta E+A ) \right)v_{\bar{t}t}^k  + \frac{1}{2} (\beta E +A )(v^{k+1}+v^{k-1}) \\
&~~~~~-\alpha \beta \frac{(v^{k+1})^3-(v^{k-1})^3}{3(v^{k+1}-v^{k-1})} =0
\end{align*}
We multiply the previous equation by $(v^{k+1}-v^{k-1})=\tau (v_{\bar{t}}^k + v_{t}^k)$ and sum over the spatial mesh points.
We reorganize the expressions and get the following formula 
\be\label{num_en}
E_h(v^k) =E_h(v^{k-1}),
\ee
where
\begin{align*}
E_h(v^k)=\left( \left( \beta (E+A^{-1})- \frac{\tau^2}{4}(\beta E+A ) \right)v_{t}^k ,v_{t}^k \right)+\frac{1}{4} \beta \left(  v^{k+1}+v^{k}, v^{k+1}+v^{k} \right) \\
+\frac{1}{4}  \left(  A(v^{k+1}+v^{k}), v^{k+1}+v^{k} \right)
-\alpha \beta \frac{((v^{k+1})^3,1)+((v^{k})^3,1)}{3}.
\end{align*}

In this way we prove the following theorem
\begin{thm}
The solution to the FDS \rf{FDS1} conserves the discrete energy
 $E_h(v^0)$, i.e.  $E_h(v^k) =E_h(v^{0})$ for each $k=1,2,...K$.
\end{thm}

\begin{thm}
The linear FDS corresponding to \rf{FDS1} is conditionally stable if
the following restriction is satisfied
$\tau^2 < \frac{\beta}{2}(1-\frac{\tau^2}{4}) h^2$.

\end{thm}
The proof is a consequence of the stability results from the book of
 Samarskii,  The theory of difference schemes, Marcel Dekker Inc., New York, 2001.

\newpage

the energy of the continuous problem \rf{problem}:

Define$Au=-\Delta u$
Then
\begin{align}\label{ex-en}
E(u):=&\int_{R^2} u_t \left((A^{-1}+E)u_t\right) dxdy+
\beta \int_{R^2} u^2 dxdy \nonumber\\
+& \int_{R^2}u \left(A u\right) dxdy
-\frac{2 \alpha \beta}{3} \int_{R^2} u^3 dxdy =const
\end{align}
Here $E(u)$ is the exact energy of problem \rf{problem}

===============================================


Let us replace the operator $Au=-
\Delta u$ with discrete operators $-\Delta_h u$ with different approximation errors - $O(h^2)$, $O(h^4)$, $O(h^6)$.

Suppose we know the discrete approximation of the derivative $u_t$
with approximation errors $O(\tau^2)$, $O(\tau^4)$, $O(\tau^6)$.

then one can apply quadrature formulas to evaluate numerically
the energy \rf{ex-en}!

When the numerical solution is found with $O(h^2+\tau^2)$ error,
we apply \rf{quadr2} with $O(h^2)$ error;


When the numerical solution is found with $O(h^4+\tau^4)$ error,
we apply 2D Simpson's rule \rf{quadr4} with $O(h^4)$ error;

When the numerical solution is found with $O(h^6+\tau^6)$ error,
we apply 2D Boole's rule \rf{quadr6-2D} with $O(h^6)$ error;

===========================================================

Quadrature formulas in 2D case for evaluation of 

\begin{equation}\label{int}
D(u)=\int_{a_1}^{b_1} \int_{a_2}^{b_2} u(x,y)dx dy
\end{equation}

$x_i, ~i=0,1,...,N_1$; $x_0=a_1,~x_{N_1}=b_1$, $h_1=(b_1-a_1)/N_1$


$y_j, ~j=0,1,...,N_2$; $y_0=a_2,~y_{N_2}=b_2$,  $h_2=(b_2-a_2)/N_2$

===========================================================================================

2D trapezoidal formula (with global $O(h^2)$ error):

\begin{align}\label{quadr2}
D_h(u_{i,j}) =& \sum_{i=1}^{N_1-1} \sum_{j=1}^{N_2-1} h_1 h_2 u_{i,j}
+\frac{h_1}{2}\sum_{i=0} \sum_{j=1}^{N_2-1} h_2 u_{i,j}
+\frac{h_1}{2}\sum_{i=N_1} \sum_{j=1}^{N_2-1} h_2 u_{i,j} \nonumber\\
+&\frac{h_2}{2}\sum_{j=0} \sum_{i=1}^{N_1-1} h_1 u_{i,j}
+\frac{h_2}{2}\sum_{j=N_2} \sum_{i=1}^{N_1-1} h_1 u_{i,j}
\nonumber\\
+&\frac{1}{4}h_1 h_2 \left(u_{0,0}+u_{N_1,0}+u_{N_1,N_2}+u_{0,N_2}
\right)
\end{align}
=================================================

2D Simpson's rule (with global $O(h_1^4+h_2^4)$ error), assuming that $N_1=2k$, $N_2=2 l$:

for every $m=0,1,2,\cdots N_2$ we compute 
$$D_m= \frac{h_1 }{3} 
\left\{ u_{0,m}+u_{N_1,m}+ 4 \sum_{i=1}^{\frac{N_1}{2}}   u_{2i-1,m}
 +2 \sum_{i=1}^{\frac{N_1}{2}-1} u_{2i,m} \right\}$$


Then 

\begin{equation}\label{quadr4}
D_h(u)=\frac{h_2 }{3} 
\left\{ D_{0}+D_{N_2}+ 4 \sum_{j=1}^{\frac{N_2}{2}}   D_{2j-1}
 +2 \sum_{j=1}^{{\frac{N_2}{2}}-1} D_{2j} \right\}
\end{equation}
is the approximation of the integral \eqref{int} with global $O(h_1^4+h_2^4)$ error

%
==============================================

Let $N_1=4k$, $x_i, ~i=0,1,...,N_1$, $h_1=(b_1-a_1)/N_1$ 

1D Boole's rule (with global $O(h_1^6)$ error):

\begin{align}\label{quadr6-1D}
D_h(u) =& \frac{2h_1}{45} 
\left\{
7u_0+7u_{N_1}+32 \sum_{k=1}^{{\frac{N_1}{2}}}u_{2k-1}
+12\sum_{k=1}^{{\frac{N_1}{4}}}u_{4k-2}
+14 \sum_{k=1}^{\frac{N_1}{4}-1}u_{4k}
\right\}
\end{align}

===========================================

2D Boole's rule (with global $O(h_1^6+h_2^6)$ error), where $N_1=4k$, $N_2=4 l$

for every $m=0,1,2,\cdots N_2$ we compute

\begin{align*}
D_m =& \frac{2h_1}{45} 
\left\{
7u_{0,m}+7u_{N_1,m}+32 \sum_{i=1}^{\frac{N_1}{2}}u_{2i-1,m}
+12\sum_{i=1}^{\frac{N_1}{4}}u_{4i-2,m}
+14 \sum_{i=1}^{\frac{N_1}{4}-1}u_{4i,m}
\right\}
\end{align*}


then \eqref{quadr6-2D} is the approximation to the integral \eqref{int} with global $O(h_1^6+h_2^6)$ error



\begin{align}\label{quadr6-2D}
&D_h(u) =
\frac{2h_2}{45} 
\left\{
7D_{0}+7D_{N_2}+32 \sum_{j=1}^{\frac{N_2}{2}}D_{2j-1}
+12\sum_{j=1}^{\frac{N_2}{4}}D_{4j-2}
+14 \sum_{j=1}^{\frac{N_2}{4}-1}D_{4j}
\right\}
\end{align}

At first convergence rates of the obtained Taylor solutions are calculated, followed by the convergence rate of the energy. Then the same type of results are displayed for the Energy Saving method. Some additional calculations are made for the integral to assure its proper behavior. At the end, the difference between Energy Saving and Taylor methods is presented. The energy and integral are vectors of size $[0:\tau:T]$, i.e. those are calculated at each iteration step. In the case of Taylor method, the energy is calculated using trapezoidal, Simpson's and Boole's rules for $O(h^{2} + \tau^2 )$, $O(h^{4} + \tau^4 )$ and $O(h^{6} + \tau^6 )$, respectively. In the case of Energy Saving method, the energy is calculated using trapezoidal rule. The main tool for testing the convergence rate $\xi$ of all examined finite difference schemes and TS expansions is the Runge's Method. 
\begin{equation}\label{Runge}
\xi = ln  \frac{\Vert u_{h,\tau} - u_{(h,\tau)/2} \Vert_\kappa } {\Vert  u_{(h,\tau)/2} - u_{(h,\tau)/4} \Vert_\kappa  } | / ln(2),
\end{equation}
when no exact solution to the problem is known. It is also used in order to calculate the convergence of the energy. For the last, one obtains three nested vectors of size $[0:\tau:T]$, $[0:\tau/2:T]$ and $[0:\tau/4:T]$. For all calculations and tables below it is given that $$T=10.$$ The calculations are done for two parameter sets $\beta = 3$, $c=0.52$ and $\beta = 1$, $c=0.9$. The domain size for the first case is $L_x = 40$, $L_y = 40$ and for the other is  $L_x = 128$, $L_y = 58$, respectively. There are also two cases on the boundary: zero and with some boundary function.
%A
\begin{table}[ht]
\centering
\small
		\begin{tabular}{||c|l|ll|ll||}
			\hline
			\hline
      \multirow{2  }{*}{FDS}        & \multirow{2  }{*}{$h$, $\tau$}  & \multirow{2  }{*}{errors $E_i$in$L_2$}  &Conv.& \multirow{2  }{*}{errors $E_i$in$L_\infty$}  &Conv.  \\
	         &                    &                               & Rate   &                                        & Rate \\
   			\hline 
					\hline 
  $\beta=3$                &0.2, 0.005          &              &              &                     &      \\
   c=0.52                     &0.1, 0.0025          & 0.918613 &            &0.833815    &       \\
     $O(h^2 + \tau^ 2)$ &0.05, 0.00125   & 0.279642 & 1.71    &0.324158    &  1.36       \\
			\hline 
  $\beta=3$               &0.2, 0.02       &              &            &                     &      \\
   c=0.52                    &0.1, 0.01      &0.305301 &            &0.371968    &       \\
     $O(h^4+ \tau^4)$ &0.05, 0.005&0.020067 & 3.92   &0.021740     &4.09       \\
			\hline 
  $\beta=3$               &0.2, 0.02       &                &            &                     &      \\
     c=0.52                 &0.1, 0.01        &0.094772 &            &  0.126178    &       \\
     $O(h^6+ \tau^6)$ &0.05, 0.005 &0.001757 &5.75     & 0.002006     & 5.97       \\
	   \hline
			\hline 
       $\beta=1$       &0.4, 0.002        &             &            &           &   \\
                  c=0.9    &0.2, 0.001       &  0.20366   &            &0.075854 &   \\
  $O(h^2+ \tau^2)$ &0.1, 0.0005   &0.048320   &2.07  &0.022307  & 1.77 \\
			\hline
      $\beta=1$               &0.4, 0.04    &            &               &             &    \\
       c=0.9                     &0.2, 0.02     & 0.028275   &        &  0.013518   &   \\
       $O(h^4+ \tau^4)$ &0.1, 0.01   &0.001812 & 3.96  & 0.000971  & 3.80  \\
    \hline
  $\beta=1$     &0.4, 0.04   &            &          &                  &      \\
      c=0.9                    &0.2, 0.02   &0.006734 &           & 0.003338      &       \\
     $O(h^6+ \tau^6)$ &0.1, 0.01 & 0.000232 &4.86 & 0.000069  & 5.60        \\
	   \hline
			\hline 
		\end{tabular}
		\caption{Convergence tests for Taylor method with zero boundary and different approximation errors $O(h^{2} + \tau^2 )$, $O(h^{4} + \tau^4 )$ and $O(h^{6} + \tau^6 )$. Errors $E_i$ are measured in $L_2$ and $L_\infty$ norms}
\label{table:A}
\end{table}

%B
\begin{table}[ht]
\centering
\small
		\begin{tabular}{||c|l|ll|ll||}
			\hline
			\hline
      \multirow{2  }{*}{FDS}        & \multirow{2  }{*}{$h$, $\tau$}  & \multirow{ 2 }{*}{errors $E_i$in$L_2$}  &Conv.& \multirow{2  }{*}{errors $E_i$in$L_\infty$}  &Conv.  \\
	                                        &                                                     &                                                                 &  Rate &                                                                       & Rate \\
   			\hline 
					\hline 
  $\beta=3$                &0.2, 0.005         &              &            &                     &      \\
   c=0.52                     &0.1, 0.0025         & 0.086596  &            &1.015191 &       \\
     $O(h^2 + \tau^ 2)$ &0.05, 0.00125  &0.014754 &2.55       &0.260121     & 1.96      \\
			\hline 
  $\beta=3$               &0.2, 0.02       &                &            &                     &      \\
   c=0.52                    &0.1, 0.01      &0.094290 &            &0.499355   &       \\
     $O(h^4+ \tau^4)$ &0.05, 0.005  &0.003936 &4.58    &0.012718   &5.29      \\
			\hline 
  $\beta=3$               &0.2, 0.02       &                &            &                      &            \\
     c=0.52                 &0.1, 0.01        &0.139543 &            &  0.681089    &           \\
     $O(h^6+ \tau^6)$ &0.05, 0.005 &0.003360 &5.37     & 0.018324     & 5.21   \\
	   \hline
			\hline 
       $\beta=1$       &0.4, 0.002        &             &            &           &   \\
                  c=0.9    &0.2, 0.001       &  0.046343   &            &0.352955 &   \\
  $O(h^2+ \tau^2)$ &0.1, 0.0005   &0.007430   &2.64  &0.086470  & 2.02 \\
			\hline
      $\beta=1$               &0.4, 0.04    &            &               &             &    \\
       c=0.9                     &0.2, 0.02     & 0.023067   &        &  0.040550   &   \\
       $O(h^4+ \tau^4)$ &0.1, 0.01   &0.001411 & 4.03   & 0.003203  & 3.66  \\
    \hline
  $\beta=1$     &0.4, 0.04   &            &          &                  &      \\
      c=0.9                    &0.2, 0.02   &0.010898 &           & 0.032597      &       \\
     $O(h^6+ \tau^6)$ &0.1, 0.01 & 0.000496 &4.45 & 0.001266  & 4.68        \\
	   \hline
			\hline 
		\end{tabular}
		\caption{Convergence tests for the energy of Taylor solution with zero boundary and different approximation errors $O(h^{2} + \tau^2 )$, $O(h^{4} + \tau^4 )$ and $O(h^{6} + \tau^6 )$. Errors $E_i$ are measured in $L_2$ and $L_\infty$ norms.}
\label{table:B}
\end{table}
 
The convergence results for the solution and energy when using the Taylor method are good and correspond to the approximation order that is used. For the energy case and $O(h^ + \tau^)$...
%C
\begin{table}[ht]
\centering
\small
		\begin{tabular}{||c|l|ll|ll||}
			\hline
			\hline
      \multirow{2  }{*}{FDS}        & \multirow{2  }{*}{$h$, $\tau$}  & \multirow{2  }{*}{errors $E_i$in$L_2$}  &Conv.& \multirow{2  }{*}{errors $E_i$in$L_\infty$}  &Conv.  \\
	                                        &                                                     &                                                                 &  Rate &                                                                       & Rate \\
   			\hline 
					\hline 
  $\beta=3$                &0.2, 0.005         &                    &                &                  &                   \\
   c=0.52                     &0.1, 0.0025         & 0.917455   &                & 0.834942  &                   \\
     $O(h^2 + \tau^ 2)$ &0.05, 125e-6  & 0.289621   & 1.66       & 0.333312   &  1.32    \\
	   \hline
			\hline 
       $\beta=1$           & 0.4, 0.002       &                   &           &                 &   \\
                  c=0.9       & 0.2, 0.001        & 0.200424   &          &0.072726  &   \\
  $O(h^2+ \tau^2)$  & 0.1, 0.0005       & 0.047899   & 2.06  &0.021451  & 1.76 \\
	   \hline
			\hline 
		\end{tabular}
		\caption{Convergence tests for the Energy Saving method with zero boundary with approximation errors $O(h^{2} + \tau^2 )$. Errors $E_i$ are measured in $L_2$ and $L_\infty$ norms}
\label{tableC}
\end{table}

%D
\begin{table}[ht]
\centering
\small
		\begin{tabular}{||c|l|ll|ll||}
			\hline
			\hline
      \multirow{2  }{*}{FDS}        & \multirow{2  }{*}{$h$, $\tau$}  & \multirow{2  }{*}{errors $E_i$in$L_2$}  &Conv.& \multirow{2  }{*}{errors $E_i$in$L_\infty$}  &Conv.  \\
	                                        &                                                     &                                                                 &  Rate &                                                                       & Rate \\
   			\hline 
					\hline 
  $\beta=3$                &0.2, 0.005         &                    &                &                  &                   \\
   c=0.52                     &0.1, 0.0025         & 0.101519   &                & 1.015192  &                   \\
     $O(h^2 + \tau^ 2)$ &0.05, 125e-6  & 0.018393   & 2.46       & 0.260121   & 1.96   \\
	   \hline
			\hline 
       $\beta=1$           & 0.4, 0.002       &                   &           &                 &   \\
                  c=0.9       & 0.2, 0.001        & 0.051409   &          &0.363515  &   \\
  $O(h^2+ \tau^2)$  & 0.1, 0.0005       & 0.008939   & 2.52  &0.089393  & 2.02  \\
	   \hline
			\hline 
		\end{tabular}
		\caption{ Convergence tests for the energy of the Energy Saving method with zero boundary and different approximation errors $O(h^{2} + \tau^2 )$. Errors $E_i$ are measured in $L_2$ and $L_\infty$ norms. }
\label{tableD}
\end{table}


%F
\begin{table}[ht]
\centering
\small
		\begin{tabular}{||c|l|l|l|l|l|l|l||}
			\hline
			\hline
      FDS                    & $h$, $\tau$      &   \multicolumn{3}{ |c| }{Energy Saving method}   &    \multicolumn{3}{ |c|| }{Taylor method}   \\
                                 &                         & min &  max   &  diff      & min &  max   &  diff  \\
   			\hline 
			\hline 
  $\beta=3$                   &0.2,  0.001       &28.586694 & 28.586695 & 0.000001 &  28.586695 & 29.273293 & 0.686598       \\
   c=0.52                        &0.1, 0.0005      & 29.601886 & 29.601886 & 0.000000 &  29.601886 & 29.828547 & 0.226660        \\
     Energy                     &0.05, 0.00025      &29.862007 & 29.862007 & 0.000000   & 29.862008 & 29.959565 & 0.097557        \\
			\hline 
  $\beta=3$                   &0.2, 0.001       &3.996038  & 4.009427  & 0.013389 &  3.996038  & 4.009427  & 0.013389      \\
   c=0.52                        &0.1, 0.0005      & 4.129162  & 4.143066  & 0.013904 & 4.129162  & 4.143066  & 0.013904        \\
     Integral                     &0.05, 0.00025      &  4.159081  & 4.173124  & 0.014043   &  4.159091  & 4.173133  & 0.014042       \\

	   \hline
			\hline 
       $\beta=1$          &0.4, 0.002        & 13.037144 & 13.037144 & 0.000000  &  13.037144 & 13.144258 & 0.107114  \\
                  c=0.9      &0.2, 0.001         & 13.390099 & 13.390099 & 0.000000 & 13.390099 & 13.440774 & 0.050675   \\
  Energy&0.1, 0.0005        & 13.476569 & 13.476569 & 0.000000 &  13.476569 & 13.501607 & 0.025037   \\
			\hline 
       $\beta=1$          &0.4, 0.002        & 15.803319     & 16.106299  & 0.302981  &  15.803319  & 16.106299  & 0.302981   \\
                  c=0.9      &0.2, 0.001         & 16.155400  & 16.466173  & 0.310772     & 16.155400  & 16.466173  & 0.310772   \\
                    Integral &0.1, 0.0005       & 16.247459  & 16.560524  & 0.313065  &  16.247459  & 16.560524  & 0.313065  \\
	   		\hline
			\hline 
		\end{tabular}
		\caption{The Energy/Integral vector is a map from $0:\tau:T \rightarrow D(u(0:\tau:T))$ using formula \rf{int}. The min, max is simply the minimum and maximum element in the vector; diff value is simply max - min. The calculations use $O(h^2 + \tau^ 2)$ approximation with zero boundary condition.  }
\label{tableF}
\end{table}
The third column with energy values shows that the conservative scheme in the Energy Saving method keeps the energy constant with very small and almost invisible deviations for both parameter sets. In the case of the Taylor method, deviations from the initial energy decrease with decreasing the spatial and time step sizes, as shown in the "diff" column (6th), and are very small compared to the total energy. 
The integral though increases (monotonically) with time where the "min" value column present is at time $t=0$ and the "max" is at time $t=10$. In the case of $\beta = 1$, $c=0.9$ the difference (again 3rd and 6th columns) is roughly $7\%$ for both solution approaches.
%E
\begin{table}[ht]
\centering
\small
		\begin{tabular}{||c|l|l|l|l||}
			\hline
			\hline
      \multirow{2  }{*}{FDS}        & \multirow{2  }{*}{$h$, $\tau$}  &   $uT_i - uE_i$  in $L_2$     &  $uT_i - uE_i$ in $L_\infty$ & \multirow{2  }{*}{$max|uE_i|$} \\
	                                        &                                                     &      difference                     &           difference                  &                                                       \\
   			\hline 
					\hline 
  $\beta=3$                   &0.2, 0.001         &  0.021823       & 0.024721  & 0.893215     \\
   c=0.52                        &0.1, 0.0005        &  0.025126       & 0.028328 &  1.723728     \\
     $O(h^2 + \tau^ 2)$ &0.05, 0.00025     & 0.015794         &0.017048  &   1.872523   \\
			\hline 
			\hline 
       $\beta=1$          &0.4, 0.002        & 0.009981     & 0.004560 & 0.656747   \\
                  c=0.9      &0.2, 0.001        & 0.005047      & 0.002373  & 0.673901   \\
  $O(h^2+ \tau^2)$ &0.1, 0.0005         & 0.002521      &0.001117 & 0.672231   \\
			\hline
	   \hline
			\hline 
		\end{tabular}
		\caption{Difference between obtained solutions with Energy Saving $uE$ and Taylor $uT$ methods using zero boundary condition and approximation errors $O(h^{2} + \tau^2 )$. Differences are measured in $L_2$ and $L_\infty$ norms.}
\label{tableE}
\end{table}

Notice that the final time $T=10$  is the same for all tables.
\end{document}
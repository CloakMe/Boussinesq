\documentclass{beamer}


%\beamertemplateshadingbackground{yellow!100}{white}
\usepackage[english]{babel}
\usepackage{amsfonts,amsmath}
\usepackage{graphicx}
\usepackage{sansmathaccent}
\pdfmapfile{+sansmathaccent.map}
\usepackage{multirow}
%\usetheme{Warsaw}
\usetheme[secheader]{Madrid}

\usepackage{multimedia}
\logo{\includegraphics[height=0.5cm]{imi.jpg}}

\newcommand{\be}{\begin{equation}}
\newcommand{\ee}{\end{equation}}
\newcommand{\rf}[1]{(\ref{#1})}
\newcommand{\RR}{\mathbb{R}}
\newtheorem{thm}{Theorem}
\newtheorem{lm}{Lemma}

%\def\ra#1\{\renewcommand{\arraystretch}{#1}
\begin{document}
 %[propagating wave solutions to the 2D BPE]
\title{Comparison between two numerical methods for solution of 2D Boussinesq paradigm equation}

\author[K. Angelow]{{\underline{Krassimir Angelow}}, Natalia Kolkovska}
\institute[IMI -- BAS]{Institute of Mathematics and Informatics\\ Bulgarian Academy of Sciences, Sofia, Bulgaria,\\ e-mail: angelow@math.bas.bg}
\date[2021]{AMITANS, June 20-25  2021,  Albena, Bulgaria}


%---------- frame 01 ----------------
\begin{frame}
 \titlepage
 \begin{center}
  Supported by Bulgarian National Science Fund Grant K$\Pi$-06-H22/2
  \end{center}
\end{frame}

%---------- frame 02 ----------------
\begin{frame}
\tableofcontents 
\setbeamertemplate{table of contents shaded}[default]
\section{Boussinesq Paradigm Equation}
\section{Quadrature Formulas}
\section{Numerical Methods}
\subsection{Energy Saving Method with Conservative Scheme}
\subsection{Taylor Series (TS) Approach}
%---------- frame 06 ----------------
\section{Convergence}

%---------- frame 07 ----------------
\section{Comparison Between TS and Conservative Scheme}
%\tableofcontents 
\end{frame}

%---------- frame 03 ----------------
\begin{frame}
\frametitle{Boussinesq Paradigm Equation}


The standard two dimensional BPE formulation is

\begin{align}\label{problem}
 \frac{\partial^2 u}{\partial t^2}= \Delta u + \beta_1 \Delta \frac{\partial^2}{\partial t^2} u -  \beta_2 \Delta^2 u +  \alpha \Delta (u^2)
\\
u(x,y,0) = u_0(x,y), \quad \frac{\partial u}{\partial t}(x,y,0)=u_1(x,y), \nonumber
\\
u(x,y,t) \rightarrow 0, \quad \Delta u(x,y, t) \rightarrow 0 \quad \text{for} \quad ||(x,y)|| \rightarrow \inf \nonumber
\end{align}
where $\alpha>0$, $\beta_1>0$, $\beta_2>0$  are dispersion parameters, and $\Delta$ is the Laplace operator. The domain of the unknown function $u$ is defined by:
\be
 u:\Omega \times [0, T] \rightarrow R
\ee
where $\Omega \in R^2$ and $T>0$.
\end{frame}

%---------- frame 04 ----------------

\begin{frame}
\frametitle{Boussinesq Paradigm Equation}


The stationary two dimensional solitary wave is defined by
$$u(x, y, t) = U(x, y-ct)$$
and evaluated in 
\begin{description}
 \item[1] Christou, Christov, Galerkin Spectral Method for the 2D Solitary Waves of Boussinesq Paradigm Equation, 
 \item[2] Christou, Christov, Fourier-Galerkin Method for 2D Solitons of Boussinesq Equation, 
  \item[3] Christov, Choudhury, The Perturbation Solution for the 2D Boussinesq Equation.
\end{description}

Remarks:
\begin{description}
 \item[-] [1] and [2] uses FDS with $O(h^2 + \tau^2)$ approximation, 
  \item[-] [3] - the" best fit" formulas for small speeds $c$.
\end{description}
%Numerical Realization of Unsteady Solutions for the 2D Boussiensq Paradigm Equation, Christov, Kolkovska, Vasileva
% represents another methodology based on "best fit" formulas which is applicable for small speeds $c$.
% using two numerical methods with $O(|h|^2 + \tau^2)$ error is
\end{frame}

%---------- frame 05 ----------------

\begin{frame}
\frametitle{Boussinesq Paradigm Equation}


Based on the stationary solutions, the 2D hyperbolic equation is solved numerically in 
\begin{description}
 \item[4] Christov, Vasileva, Kolkovska, Numerical Realization of Unsteady Solutions for the 2D BPE,
 \item[5] Chertock, Christov, Kurganov, Central-Upwind Schemes for the Boussinesq Paradigm Equations, 
  \item[6] Angelow, Kolkovska, A Multicomponent Alternating Direction Method for Numerical Solving of Boussinesq Paradigm Equation.
\end{description}
\end{frame}

%---------- frame 06 ----------------

\begin{frame}
\frametitle{Boussinesq Paradigm Equation}

The goal of this talk is to:
\begin{description}
 \item[-] propose high order numerical method for the solution of BPE,
 \item[-] investigate properties of the resulting wave for velocities $c ~ max(1, \sqrt{ \frac{\beta_2}{\beta_1} } )$, which are close to its maximal existence values,
 \item[-] the wave's maximum is constant
\end{description}
\end{frame}

%---------- frame 07 ----------------

\begin{frame}
\frametitle{Boussinesq Paradigm Equation}
The following variable change is applied:
\begin{align}
x = \sqrt{\beta_1} \bar{x}, \quad y = \sqrt{\beta_1} \bar{y}, \quad t = \sqrt{\beta_1} \bar{t}
\end{align}

We study the problem

\be\label{problem}
\beta(E-\Delta) \frac{\partial^2 u}{\partial t^2}=
 \beta \Delta u -\Delta^2 u -\alpha \beta \Delta (u^2)
\ee

\end{frame}



%---------- frame 05 ----------------
\begin{frame}
\frametitle{Quadrature Formulas for the Integral and Energy}
\framesubtitle{Integral and Energy}
The integral:

\begin{equation}\label{int}
D(u)=\int_{R^2} u(x,y)dx dy
\end{equation}

and energy:
\begin{align}\label{ex-en}
E(u):=&\int_{R^2} u_t \left((A^{-1}+E)u_t\right) dxdy+
\beta \int_{R^2} u^2 dxdy \nonumber\\
+& \int_{R^2}u \left(A u\right) dxdy
-\frac{2 \alpha \beta}{3} \int_{R^2} u^3 dxdy =const
\end{align}
of the continuous problem \rf{problem}. Here $E(u)$ is the exact energy of problem \rf{problem} and $Au=-\Delta u$.
\end{frame}

%-----------------------------------------------------------------

\begin{frame}
\frametitle{Energy Saving Method with Conservative Scheme}
\framesubtitle{The Grid}

\begin{center}\vspace{0.4cm}
	\begin{minipage}[b]{0.6\linewidth}
		\includegraphics[width=\linewidth]{Omega_dah.png}
	\end{minipage}
\end{center}

\end{frame}

%---------- frame ----------------

\begin{frame}
\frametitle{Quadrature Formulas for the Integral and Energy}
\framesubtitle{2D trapezoidal formula (with global $O(h_1^2+h_2^2)$ error)}

Let's assign the X,Y dimensions of the computational domain with $N_1$ and $N_2$.
Then the approximation of the integral \eqref{int} with global $O(h_1^2+h_2^2)$ error is

\begin{align}\label{quadr2}
D_h(u_{i,j}) =& \sum_{i=1}^{N_1-1} \sum_{j=1}^{N_2-1} h_1 h_2 u_{i,j}
+\frac{h_1}{2}\sum_{i=0} \sum_{j=1}^{N_2-1} h_2 u_{i,j}
+\frac{h_1}{2}\sum_{i=N_1} \sum_{j=1}^{N_2-1} h_2 u_{i,j} \nonumber\\
+&\frac{h_2}{2}\sum_{j=0} \sum_{i=1}^{N_1-1} h_1 u_{i,j}
+\frac{h_2}{2}\sum_{j=N_2} \sum_{i=1}^{N_1-1} h_1 u_{i,j}
\nonumber\\
+&\frac{1}{4}h_1 h_2 \left(u_{0,0}+u_{N_1,0}+u_{N_1,N_2}+u_{0,N_2}
\right).
\end{align}

\end{frame}

\begin{frame}
\frametitle{Quadrature Formulas for the Integral and Energy}
\framesubtitle{2D Simpson's rule (with global $O(h_1^4+h_2^4)$ error)}
Assume that $N_1=2k$, $N_2=2 l$.
For every $m=0,1,2,\cdots N_2$ we compute 

$$D_m= \frac{h_1 }{3} 
\left\{ u_{0,m}+u_{N_1,m}+ 4 \sum_{i=1}^{\frac{N_1}{2}}   u_{2i-1,m}
 +2 \sum_{i=1}^{\frac{N_1}{2}-1} u_{2i,m} \right\}$$


Then 

\begin{equation}\label{quadr4}
D_h(u)=\frac{h_2 }{3} 
\left\{ D_{0}+D_{N_2}+ 4 \sum_{j=1}^{\frac{N_2}{2}}   D_{2j-1}
 +2 \sum_{j=1}^{{\frac{N_2}{2}}-1} D_{2j} \right\}
\end{equation}
is the approximation of the integral \eqref{int} with global $O(h_1^4+h_2^4)$ error

\end{frame}

%------------------------------------------------------------------

\begin{frame}
\frametitle{Quadrature Formulas for the Integral and Energy}
\framesubtitle{2D Boole's rule (with global $O(h_1^6+h_2^6)$ error)}

Assume that $N_1=4k$, $N_2=4 l$.
For every $m=0,1,2,\cdots N_2$ we compute

\begin{align*}
D_m =& \frac{2h_1}{45} 
\left\{
7u_{0,m}+7u_{N_1,m}+32 \sum_{i=1}^{\frac{N_1}{2}}u_{2i-1,m}
+12\sum_{i=1}^{\frac{N_1}{4}}u_{4i-2,m}
+14 \sum_{i=1}^{\frac{N_1}{4}-1}u_{4i,m}
\right\}
\end{align*}

then \eqref{quadr6-2D} is the approximation to the integral \eqref{int} with global $O(h_1^6+h_2^6)$ error

\begin{align}\label{quadr6-2D}
&D_h(u) =
\frac{2h_2}{45} 
\left\{
7D_{0}+7D_{N_2}+32 \sum_{j=1}^{\frac{N_2}{2}}D_{2j-1}
+12\sum_{j=1}^{\frac{N_2}{4}}D_{4j-2}
+14 \sum_{j=1}^{\frac{N_2}{4}-1}D_{4j}
\right\}
\end{align}
\end{frame}

%------------------------------------------------------------------

\begin{frame}
\frametitle{Numerical Methods}
\framesubtitle{Energy Saving Method with Conservative Scheme}

The approximation of the differential operators is defined as:
\begin{equation}
\frac{\partial^2 u}{\partial t^2}(x_i, y_j, t_k ) = \frac{ u^{(k+1)}_{i, j} - 2u^{(k)}_{i,j} + u^{(k-1)}_{i,j} }{\tau^2} + O(\tau^2) 
\end{equation}

\begin{equation}
\frac{\partial^2 u}{\partial x^2}(x_i, y_j, t_k ) = \frac{ u^{(k)}_{i+1, j} - 2u^{(k)}_{i,j} + u^{(k)}_{i-1,j} }{h_1^2} + O(h_1^2) 
\end{equation}

\begin{equation}
\frac{\partial^2 u}{\partial y^2}(x_i, y_j, t_k ) = \frac{ u^{(k)}_{i, j+1} - 2u^{(k)}_{i,j} + u^{(k)}_{i,j-1} }{h_2^2} + O(h_2^2) 
\end{equation}


\begin{equation}
\Delta_h u(x_i, y_j, t_k )  = \frac{ u^{(k)}_{i+1, j} - 2u^{(k)}_{i,j} + u^{(k)}_{i-1,j} }{h_1^2} + \frac{ u^{(k)}_{i, j+1} - 2u^{(k)}_{i,j} + u^{(k)}_{i,j-1} }{h_2^2}
\end{equation}

\end{frame}

%------------------------------------------------------------------

\begin{frame}
\frametitle{Numerical Methods}
\framesubtitle{Energy Saving Method with Conservative Scheme}
The grid function is defined as:
\begin{equation}
(I-\Delta_h)\frac{ u^{(k+1)}_{i, j} - 2u^{(k)}_{i,j} + u^{(k-1)}_{i,j} }{\tau^2} = (\Delta_h - \Delta_h^2)u^{(k)}_{i,j} + \Delta_h(g(u^{(k)}_{i,j}))
\end{equation}
%
where the non-linear term $g$ is defined as:
\begin{align}
g(u^{(k)}_{i,j})=& -\frac{\alpha \beta} { 3 } \left( (u^{(k+1)}_{i,j})^2 + (u^{(k-1)}_{i,j})(u^{(k+1)}_{i,j}) + (u^{(k-1)}_{i,j})^2 \right) + \nonumber\\
+&\frac{ (\beta - 1 )}{ 2 }\left( u^{(k+1)}_{i,j} + u^{(k-1)}_{i,j} \right).
\end{align}


\end{frame}

%------------------------------------------------------------------

\begin{frame}
\frametitle{Numerical Methods}
\framesubtitle{Energy Saving Method with Conservative Scheme}
The last transforms into an implicit equation system $M v^{(k+1)} = b( v^{(k+1)} ,  v^{(k)} ,  v^{(k-1)}  )$.

Piccardi  Iterations:

Set
\begin{align}
 v^{(k+1)}_0 =  v^{(k)}
\end{align}
Do
\begin{align}
 v^{(k+1)}_{m+1} =  M^{-1}  b( v^{(k+1)}_{m} ,  v^{(k)} , v^{(k-1)}  ), \quad m=1,2, ...
\end{align}
Until $||  v^{(k+1)}_{m+1} -  v^{(k+1)}_{m}|| < \epsilon$. 
\\
For $\epsilon = 10^{-13}$ the number of iterations $m$ is around 6.
\end{frame}

%------------------------------------------------------------------

\begin{frame}
\frametitle{Numerical Methods}
\framesubtitle{Taylor Series}

Suppose we have functions $y$ and $f$ with the following properties
\begin{equation}
y:I \rightarrow \RR, \quad f:I \times \Omega \rightarrow \RR \nonumber
\end{equation}
where $y \in C^1(I,\RR)$ and $f$ is sufficiently smooth on $I \times \Omega$. Then, $f$ represent the phase space and $y$ is the unknown function to the differential equaiton:
\begin{equation}
\dot{y}(t) = f(t, y(t)).
\end{equation}
By repeated differentiation we can find each function

\begin{equation}
\frac{d^s}{dt^s}y(t) = \frac{d^{s-1}}{dt^{s-1}}f(t, y(t))
\end{equation}

and evaluate it at $t=t_0$ for each derivative.
\end{frame}

%------------------------------------------------------------------

\begin{frame}
\frametitle{Numerical Methods}
\framesubtitle{Taylor Series}

The order $s$ formula for computing $y(t_{n+1})=$ $y(t_{n}+h)$ using these functions, evaluated at $t=t_{n}$ and $y=y_{n}$, is
\begin{eqnarray}
y_{n+1} = y_{n} + h f(t_{n}, y_{n}) + \frac{h^2}{2!} \frac{d^2}{dt^2} f(t_{n}, y_{n}) + ... + \frac{h^s}{s!} \frac{d^s}{dt^s} f(t_{n}, y_{n}), &\nonumber\\ 
f_i(t_l, y_l), y_l(t_l), t_l \in \RR, 1 \leq l \leq n, 1 \leq i \leq s
\end{eqnarray}
where we already have the pair $(t_0, y_0)$ from the initial condition of the differential equation.

\end{frame}

%------------------------------------------------------------------

\begin{frame}
\frametitle{Convergence}
%A

\begin{table}[ht]
\centering
\small
\resizebox{0.6\linewidth}{!}{%
		\begin{tabular}{||c|l|ll|ll||}
			\hline
			\hline
      \multirow{2  }{*}{FDS}        & \multirow{2  }{*}{$h$, $\tau$}  & \multirow{2  }{*}{errors $E_i$in$L_2$}  &Conv.& \multirow{2  }{*}{errors $E_i$in$L_\infty$}  &Conv.  \\
	         &                    &                               & Rate   &                                        & Rate \\
   			\hline 
					\hline 
  $\beta=3$                &0.2, 0.001          &              &              &                     &      \\
   c=0.45                     &0.1, 0.0005          &0.989414 &            &1.043641    &       \\
     $O(h^2 + \tau^ 2)$ &0.05, 0.00025   & 0.344813 & 1.52    &0.355511    &  1.55      \\
			\hline 
  $\beta=3$               &0.2, 0.02       &              &            &                     &      \\
   c=0.45                    &0.1, 0.01      &0.191224 &            &0.193874    &       \\
     $O(h^4+ \tau^4)$ &0.05, 0.005&0.013029 & 3.87   &0.013656     &3.82       \\
			\hline 
  $\beta=3$               &0.2, 0.02       &                &            &                     &      \\
     c=0.45                 &0.1, 0.01        &0.032671 &            &  0.033626    &       \\
     $O(h^6+ \tau^6)$ &0.05, 0.005 &0.000598 &5.77     & 0.000635    & 5.72       \\
	   \hline
			\hline 
       $\beta=1$       &0.4, 0.002        &             &            &           &   \\
                  c=0.9    &0.2, 0.001       &  0.20366   &            &0.075854 &   \\
  $O(h^2+ \tau^2)$ &0.1, 0.0005   &0.048320   &2.07  &0.022307  & 1.77 \\
			\hline
      $\beta=1$               &0.4, 0.04    &            &               &             &    \\
       c=0.9                     &0.2, 0.02     & 0.028275   &        &  0.013518   &   \\
       $O(h^4+ \tau^4)$ &0.1, 0.01   &0.001812 & 3.96  & 0.000971  & 3.80  \\
    \hline
  $\beta=1$     &0.4, 0.04   &            &          &                  &      \\
      c=0.9                    &0.2, 0.02   &0.006734 &           & 0.003338      &       \\
     $O(h^6+ \tau^6)$ &0.1, 0.01 & 0.000232 &4.86 & 0.000069  & 5.60        \\
	   \hline
			\hline 
		\end{tabular}
		}%
		\caption{Convergence tests for Taylor method with zero boundary and different approximation errors $O(h^{2} + \tau^2 )$, $O(h^{4} + \tau^4 )$ and $O(h^{6} + \tau^6 )$. Errors $E_i$ are measured in $L_2$ and $L_\infty$ norms}
\label{table:A}
\end{table}

\end{frame}
%------------------------------------------------------------------


\begin{frame}
\frametitle{Results}
\framesubtitle{Integral and Energy}
\begin{center}\vspace{0.4cm}
	\begin{minipage}[b]{0.4\linewidth}
		 \includegraphics[width=\linewidth]{figures/Integral.png}
	\end{minipage}	
	\begin{minipage}[b]{0.4\linewidth}
		\includegraphics[width=\linewidth]{figures/Energy.png}
		
	\end{minipage}

\end{center}
The Integral  (left panel) and Energy (right panel) of the solution for Evolution of the maximum for  $\beta=1$ and $c = 0.9$.
\end{frame}

%------------------------------------------------------------------

\begin{frame}
\frametitle{Results}
\framesubtitle{Wave Evolution}
\begin{center}\vspace{0.4cm}
	\begin{minipage}[b]{0.30\linewidth}
		\includegraphics[width=\linewidth]{figures/Solution1_t=0.png}
	\end{minipage}	
	\begin{minipage}[b]{0.30\linewidth}
		\includegraphics[width=\linewidth]{figures/Solution1_t=8.png}
	\end{minipage}	
	\begin{minipage}[b]{0.30\linewidth}
		 \includegraphics[width=\linewidth]{figures/Solution1_t=16.png}
	\end{minipage}
	\begin{minipage}[b]{0.30\linewidth}
		\includegraphics[width=\linewidth]{figures/Solution1_t=24.png}
	\end{minipage}	
	\begin{minipage}[b]{0.30\linewidth}
		 \includegraphics[width=\linewidth]{figures/Solution1_t=32.png}
	\end{minipage}
	\begin{minipage}[b]{0.30\linewidth}
		 \includegraphics[width=\linewidth]{figures/Solution1_t=40.png}
	\end{minipage}
%\captionof{figure}{\color{Green} Numerical solution of single wave for $\beta=1$ and $c = 0.9$ at times $t=0,8,16,24,32,40$.}
%	\caption{Numerical solution of single wave for $\beta=1$ and $c = 0.9$ at times $t=0,8,16,24,32,40$.}
%	\label{fig:oneWaveA}
%\end{figure}
\end{center}
Numerical solution of single wave for $\beta=1$ and $c = 0.9$ at times $t=0,8,16,24,32,40$.
\end{frame}

%------------------------------------------------------------------

\begin{frame}
\frametitle{XXXXXXXXXXXX}

\end{frame}


\end{document}


\documentclass{article}

\usepackage{amsthm,amsfonts,amsmath,amscd,amssymb}
\newcommand{\RR}{\mathbb{R}}
\newcommand{\be}{\begin{equation}}
\newcommand{\ee}{\end{equation}}
\newcommand{\rf}[1]{(\ref{#1})}

\begin{document}

\title{Comparison between two numerical methods for solution of 2D Boussinesq paradigm equation}

\author{Krassimir Angelow\thanks{angelow@math.bas.bg}, Natalia Kolkovska\thanks{natali@math.bas.bg}}

\maketitle





In this paper we evaluate propagating wave solutions to the two dimensional Boussinesq Paradigm Equation (BPE)
\begin{align} 
&u_{tt} - \Delta u -\beta_1  \Delta u_{tt} +\beta_2 \Delta ^2 u + \Delta f(u)=0,~~     (x,y) \in \RR^2, \, t\in\RR^+, \label{eq1}
\\ &u(x,y,0)=u_0(x,y), \, u_t(x,y,0)=u_1(x,y),   (x,y) \in \RR^2,~~\label{eq11}
%\\  &u(x,y) \rightarrow 0,  \Delta u(x,y) \rightarrow 0 , ~~ \sqrt{x^2 + y^2} \rightarrow \infty, \label{eq11}
\end{align}
where $f(u)=\alpha u^2$,  $\alpha>0$, $\beta_1>0$, $\beta_2>0$  are dispersion parameters, and $\Delta$ is the two-dimensional Laplace operator. 

Two numerical methods are used to obtain solution for equation \rf{eq1}--\rf{eq11}: Energy Saving method, which uses conservative finite difference scheme, and Taylor method, which uses Taylor series (TS) expansions around the time variable $t$. Furthermore, for each method the energy and the mass of the solution are calculated. The results, i.e. the solution, energy and mass, from the two methods are compared.
 The solution and energy are calculated over three nested meshes to examine the convergence of both methods. 

The energy and mass are  calculated at each iteration step. In the case of Taylor method, the energy is calculated using trapezoidal, Simpson's and Boole's rules with $O(h^{2} + \tau^2 )$, $O(h^{4} + \tau^4 )$ and $O(h^{6} + \tau^6 )$ errors, respectively. In the case of Energy Saving method, the energy is calculated using trapezoidal rule. 

The main tool for testing the convergence rate $\xi$ of all examined finite difference schemes and TS expansions is the Runge's Method. 
At first convergence rates of the obtained Taylor solutions are calculated, followed by the convergence rate of the energy. Then the same type of results are displayed for the Energy Saving method. Some additional calculations are made for the mass to assure its proper behavior. At the end, the difference between Energy Saving and Taylor methods is presented. 
 
The results of the comparison are very good. The maximum difference between the two approaches (among calculated solutions using different parameter sets i.e. $\beta = 3$,  $h = 0.05, 0.1, 0.2$ and $\beta = 1$ and  $h = 0.1, 0.2, 0.4$) in $L_2$ and Infinity norms is $D_{L2} = 0.009981$ and $D_{Inf} = 0.004560$ respectively (see Table DI). This compared to the solution maximum is
 $$\frac{ D_{L2} }{u_{max}} = 0.015198 \text{\%    and,   } \frac{D_{Inf} }{u_{max}} = 0.006943\text{\%.}$$

The goal is to justify the Taylor series approach by showing that both methods produce similar results. Furthermore the Taylor method could be used with high approximation order which produces finer results.





{\bf{Acknowledgment}}
The work of the second author has been partially supported by
the Bulgarian Science Fund under grant K$\Pi$-06-H22/2.



\end{document}
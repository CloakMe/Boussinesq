%\iffalse
\documentclass[%
 aip,
% jmp,
% bmf,
% sd,
% rsi,
cp,  % Conference Proceedings
 amsmath,amssymb,%nobibnotes,
% preprint,%
 reprint,%
%author-year,%
%author-numerical,%
]{revtex4-2}
%\fi

\iffalse
\documentclass{article} 
\usepackage{amsfonts,amsmath}
\usepackage{aip}
\usepackage{cp}
\usepackage{reprint}
\fi

\usepackage{graphicx}% Include figure files
\usepackage{dcolumn}% Align table columns on decimal point
\usepackage{bm}% bold math
%\usepackage[mathlines]{lineno}% Enable numbering of text and display math
%\linenumbers\relax % Commence numbering lines

\usepackage[utf8]{inputenc}
\usepackage[T1]{fontenc}
%% Loads a Times-like font. You can also load
%% {newtxtext,newtxtmath}, but not {times}, 
%% {txfonts} nor {mathtpm} as these packages
%% are obsolete and have been known to cause problems.
\usepackage{mathptmx} 
\usepackage{multirow}
\usepackage{float}

\newcommand{\be}{\begin{equation}}
\newcommand{\ee}{\end{equation}}
\newcommand{\rf}[1]{(\ref{#1})}
\newcommand{\RR}{\mathbb{R}}
\newtheorem{thm}{Theorem}
\newtheorem{lm}{Lemma}

\begin{document}

\title{Numerical Study of the classical Boussinesq Equation}% Force line breaks with \\

\author{...} % Write as First name Surname
 \email[Corresponding author: ]{angelow@math.bas.bg}
\affiliation{
Institute of Mathematics and Informatics\\
Bulgarian Academy of Sciences, Acad.\\
G.~Bonchev Bl.8, 1113, Sofia,
Bulgaria
}

\date{\today} % It is always \today, today, but any date may be explicitly specified
              % Not printed for conference proceedings

\begin{abstract}
In this paper we evaluate propagating wave solutions to the one-dimensional classical Boussinesq Equation (BE).
Two numerical methods are used to obtain solution for the equation. The first is a conservative finite difference scheme, and the second exploits Taylor series expansions around the time variable $t$. 
The solutions are computed over nested meshes to examine the convergence of both approaches. The main tool for testing the convergence rate of all examined finite difference schemes and Taylor expansions is the Runge's method.  It is shown that for a fixed time interval the numerical methods preserve the shape, integral and maximum of the solution. The waves obtained by the two methods are compared.
Both methods produce similar results in case of $O(h^{2} + \tau^2 )$ approximation order. The outcome of the comparison is very good. The maximum difference between the two approaches (among calculated solutions using different parameter sets) in $L_2$ and infinity norms is $???$ and $???$ respectively.

\end{abstract}

\maketitle

\section{\label{sec:level1}Introduction}

The Boussinesq type equations are famous with the approximation of shallow water waves or also weakly non--linear long waves. It is often used for simulation of various physical processes e.g. turbulence in fluid mechanics, vibrations in acoustics, etc. For the numerical interaction of 2D Boussinesq traveling waves (TW) one needs the shape of a stationary wave in order to build the Initial Condition (IC) $u_0$, $u_1$.

The shallow water model developed by Boussinesq and the equaitons he derived in this conjuction were first described in \cite{ref0}. The equation is known to have an explicit solution in the case of one moving solitary wave. It is also known that it has a solution with two solitary waves that results in their temporary combination and subsequent reemergence in their original forms after the interaction. An improvement of the Boussinesq model was done by Christov's "energy-consistent approximation" \cite{ref1} a.k.a the Boussinesq Paradigm Equation (BPE). Christov states that the original model with the initial value problem defined by Boussinesq is incorect in the sense of Hadamard (see \cite{ref1}), i.e. even small perturbations in the initial or boundary conditions could lead to significant differences in the behavior of the solution over time. Unfortunately the BPE is not know to posses an explicit solution in the case of two solitary waves. We would like to compare the numerical and the exact solutions in the case of two solitary waves. This could be done in the case of the original Boussinesq model:


\begin{align}\label{orgBsq}
&\frac{\partial^2 }{\partial t^2}u(x,t)= \frac{\partial^2}{\partial x^2}u(x,t) -  \beta_2 \frac{\partial^4}{\partial x^4}u(x,t) - \alpha\frac{\partial^2}{\partial x^2} (u(x,t)^2)
\\
&u(x,0) = u_0(x), \quad \frac{\partial }{\partial t}u(x,0)=u_1(x), \nonumber
\\
&u(x,t) \rightarrow 0, \quad \frac{\partial }{\partial t} u(x, t) \rightarrow 0 \quad \text{for} \quad |x| \rightarrow \infty \nonumber
\end{align}
where $\alpha>0$ is the amplitude and $\beta_2$  is the dispersion parameter. The domain of the unknown function $u$ is defined by:
\be
 u:\Omega \times [0, T] \rightarrow \RR,
\ee
where $\Omega \equiv (-\infty, \infty)$ and $T>0$. Equation \rf{orgBsq} is equivalent to the BPE if the mixed time and space derivative is omited. E.g. if we set $\beta_1 = 0$ in equation (2) from article \cite{ref21} (Chertock, Christov, Kurganov) then it is equivalent to \rf{orgBsq} defined above. The other parameters $\beta_2 = 1$ and $\alpha = 3$ are chosen in such way so that \rf{orgBsq} obtains the following two wave soliton solution:
\begin{align}\label{orgBsqSol}
u(x,t) &= -2 \frac{\partial ln F(x,t)}{\partial x^2},
\\
 F(x,t) = a_0 + a_1 e^{k_1 x + \omega_1 t + b_1} &+ a_2 e^{k_2 x + \omega_2 t + b_2}  + a_{12} e^{(k_1 + k_2) x + (\omega_1 + \omega_2)  t + b_1 + b_2}, \nonumber
\\
|k_i| < 1, \; \omega_i &= \sqrt{k^2_i(1-k^2_i) }, \; k_i \neq 0, \quad i = 1,2 \nonumber
\end{align}
The soliton is localized wave, which moves at constant velocity ($\omega_i, \: i=1,2$ in case of \rf{orgBsqSol}) and maintains its shape. When a soliton interacts with another soliton, it emerges from the "collision" unchanged, i.e. its amplitude, shape, and velocity are conserved. The results and observations in the conducted numerical analysis for the classical Boussinesq equation \rf{orgBsq} could be further used in order to find a two wave solitary solution for the 1D and 2D BPE.

\iffalse
The goal of this work is to first investigate high order accurate solutions to the BPE with new type of IC obtained in \cite{ref16} where the approximation order of the IC corresponds to the Taylor method applied here. Second, evaluate numerically the Mass, Energy, shape and maximum of the resulting waves for velocities, which are close to the maximum $c \approx c_{max}$, $c < c_{max}$ where $c_{max} = min\{1, \sqrt{ \frac{\beta_2}{\beta_1} } \}$. Third, show that the enumerated properties of the solution remain constant for some fixed time interval. Two numerical methods are applied to the BPE \rf{orgBsq}, namely, a Conservative Finite Difference Scheme (FDS) and Taylor Series (TS) approach with Method of lines. There already exist results obtained by the Conservative FDS \cite{ref20}. The last is used as a referent mechanism. The Taylor method, on the other hand, is a novel approach when applied to the BPE. Results from both techniques are compared. For series of calculations, it is shown that the difference between results (solution shape, Energy and Mass) obtaned by both methods is negligible. Furthermore, the numerical tests show that the Energy, Mass, shape and maximum of the solution are derived with high accuracy FDS and their quantities are perserved.

The paper is organized as follows. Section 2 discusses a variable change applied to \rf{orgBsq} and describes the domain discretization. Section 3 defines a conservation law for the Energy of the solution. Section 4 reviews the numerical tools used to calculate the discrete Energy with different approximation errors. Section 5 introduces the Conservative FDS with $O(|h|^2 + \tau^2)$ approximation errors and the Taylor method with $O(|h|^2 + \tau^2)$, $O(|h|^4 + \tau^4)$ and $O(|h|^6 + \tau^6)$ approximation errors. Section 6 is devoted to numerical results and more precisely the convergence rate of the solution and Energy and the evolution of Mass, Energy, shape and maximum of the solution. The last section gives an overview of the achieved goals and summarize the results.
\fi


\section{Numerical Methods}

Two numerical methods are used to obtain solution for equation \rf{orgBsq}: Method with Conservative FDS and Taylor Method which uses TS expansions around the time variable $t$. Furthermore, for each method the Mass of the solution is calculated. The solution and its properties from the two methods are compared. The numerical calculations are done over two nested meshes to examine the convergence of both methods. The goal is to justify the TS approach by showing that both methods exhibit similar results. Furthermore the TS method could be used with higher approximation order which produces finer solution.

\subsection{ Conservative Finite Difference Scheme }

The approximation of the time derivative and the Laplace operator is defined as:
\be\label{difft}
\frac{\partial^2 u}{\partial t^2}(x_i, y_j, t_k ) = \frac{ u^{(k+1)}_{i, j} - 2u^{(k)}_{i,j} + u^{(k-1)}_{i,j} }{\tau^2} + O(\tau^2) 
\ee

\be\label{diffD}
\Delta u(x_i, y_j, t_k )  = \frac{ u^{(k)}_{i+1, j} - 2u^{(k)}_{i,j} + u^{(k)}_{i-1,j} }{h_1^2} + \frac{ u^{(k)}_{i, j+1} - 2u^{(k)}_{i,j} + u^{(k)}_{i,j-1} }{h_2^2} + O(h_1^2 + h_2^2) 
\ee
Substitute the discrete differential operators \rf{difft} and \rf{diffD} in \rf{problemVC} to obtain the grid function
\be\label{consFDS}
\beta (I-\Delta_h)\frac{ u^{(k+1)}_{i, j} - 2u^{(k)}_{i,j} + u^{(k-1)}_{i,j} }{\tau^2} = (\Delta_h - \Delta_h^2)u^{(k)}_{i,j} + \Delta_h(g(u^{(k)}_{i,j}))
\ee
%
where the non-linear term $g$ is defined as:
\begin{align}
g(u^{(k)}_{i,j})=& -\frac{\alpha \beta} { 3 } \left( (u^{(k+1)}_{i,j})^2 + (u^{(k-1)}_{i,j})(u^{(k+1)}_{i,j}) + (u^{(k-1)}_{i,j})^2 \right) + \nonumber\\
+&\frac{ (\beta - 1 )}{ 2 }\left( u^{(k+1)}_{i,j} + u^{(k-1)}_{i,j} \right).
\end{align}
This is a non-trivial approximation of $g$ so that the discrete Energy is a constant function of the time variable $t$ (see \cite{ref20}). Note that the Conservative FDS \rf{consFDS} is implicit as the non-linear term depends on the solution on the upper time layer. Thus on each time layer Picard Iteration is used to resolve the discrete unknown function $u^{(k+1)}_{i,j}$.

\subsection{ Taylor Series Approach with Method of Lines }
Here we proceed as in \cite{refHyp} and use similar approach as the one defined for the BPE.
The finite differences along space and time discretization require TS expansions of $u(x,t)$. Therefore it is assumed that the solution is $p+1$ times infinitely differentiable with respect to $x$ and $t$, i.e. $u \in C^{p+1,p+1}(\Omega \times T)$.
For the Taylor method two different approximations of the spatial differential operator are used. The following central finite differences along the $x$ asis are applied:
\begin{equation}\label{fd}
u_{\widehat{xx},p}(x,y) :=  \frac{1}{h^2} \sum\limits_{i=-p/2}^{p/2} d_i u(x+ih, t).
\end{equation}
The weights $d_i$ taken from  \cite{forn} are described in Table \ref{table:A00}. 
\begin{table}[ht]
\centering
\small
		\begin{tabular}{||c|l|l|l|l|l|l|l||}
			\hline
			\hline
            $p=2$          &          &                                 &     1      &   -2   &    1    &    &        \\
   			\hline 
			\hline 
           $p=4$          &                            &   $-\frac{1}{12}$     &     $\frac{4}{3}$      &   $-\frac{5}{2} $     &    $\frac{4}{3}$    &  $-\frac{1}{12}$   &        \\
	   \hline
			\hline 
		\end{tabular}
	\caption{ Finite differences used for the approximation of the Laplace operator.}
	\label{table:A00}
\end{table}
The approximation error of formulas \rf{fd} is $O(h^p)$. Let $u_{i}(t)$ and $u_{i, \widehat{xx}, p}(t)$ be the approximations of the unknown function $u$ and its second derivative $u_{xx}$ at mesh point $(x_i)$ for arbitrary time $t$. Then one obtains a system of ODEs:
\be \label{DiscreteEq}
\frac{\partial^2 }{\partial t^2}u(x_i, t) =
(u_{i} - u_{i, \widehat{xx}, p} + 3u^2_{i})_{\widehat{xx}, p}(t) 
\ee
for all mesh points $i = 0..N_x$. For each ODE in the system we do TS expansion along the time variable:
\begin{align} \label{TSe}
u(x_i, t+\tau) = u(x_i, t) + \tau \frac{ \partial u }{ \partial t }(x_i,t)  + ... 
%\nonumber
%\\
\frac{ \tau^p }{ p! } \frac{ \partial^p u }{ \partial t^p }(x_i, t) + O(\tau^{p+1})
\end{align}
for some natural number $p \ge 2$. The approximation order of the time discretization depends on p, i.e. the number of terms included in the TS expansion. Each point on the mesh represents a starting point of a line and the line itself is described by the TS expansion \rf{TSe}. Evaluation of formula \rf{TSe} is done by evaluating each term separately. E.g. for $t=0$ the first two terms are known from the IC ($u_0$, $u_1$). The third term is evaluated from the discrete equation \rf{DiscreteEq}. With subsequent differentiaton of equation \rf{DiscreteEq} one could obtain higher time derivatives $\frac{\partial^3 u}{\partial t^3}$, $\frac{\partial^4 u}{\partial t^4}$, etc. This is an iterative procedure where e.g. the fourth time derivative requires 2nd, 1st and 0th derivatives, 3rd time derivative requires 1st and 0th derivatives. After all necessary time derivatives are calculated one could substitute those in \rf{TSe} and gets the following approximations: for $p=2$ it is $O(|h|^2 + \tau^2)$ and for $p=4$ it is $O(|h|^4 + \tau^4)$. Note that another TS expansion must be calculated for the first time derivative $u_t(x_i, t+\tau)$ which is analogous to \rf{TSe} with the same approximation order. The last is required in order to calculate the solution on the next time layer because the pair ($u$, $u_t$) serve as basis and all higher time derivatives could be expressed only by that pair.

The complexity of the algorithm is
$$ O( N_x  N_t ) $$
where $N_x$ is the number of points in $\Omega_h$ and $N_t = T/\tau$. Current choices of $p$ affect complexity by a constant which resutls in a linear time graph depending on the number of points in the discrete domain.

\section{Numerical Results}

The convergence rates for the Conservative FDS and Taylor method are calculated. At the end, the properties of the solution that are obtained numerically by the two different mechanisms are compared. The Mass is a vector of size $N_t$ and is calculated for each iteration step. The tool for testing the convergence rate $\xi$ of all examined finite difference schemes and TS expansions is the Runge's Method
\begin{equation}\label{Runge}
\xi = ln  \frac{\Vert u_{h} - u^*_{h/2} \Vert_\kappa } {\Vert  u_{h/2} - u^*_{h/4} \Vert_\kappa  } | / ln(2).
\end{equation}
For the finer meshes $u^*_{h/4}$ and $u^*_{h/2}$, the solution is calculated directly using formula \rf{orgBsqSol}.

%0
\begin{table}[H]
\centering
\small
		\begin{tabular}{||c|l|l|l|l|l|l|l||}
			\hline
			\hline
                                                                            & $O(|h|^p + \tau^p)$   &      $h$                                & $L_x$,$L_y$                              &  Methods & $T$      &  Bnd. Cond.   \\
   			\hline 
					\hline
           Test 1                                         &      $p=2, 4$             &    $h=0.4, 0.2, 0.1$      & $L_x = 30$,$L_y=27$                & Taylor, Cons. FDS &                10    &    Zero  \\
	   \hline
			\hline 
           Test 2                                       &      $p=2, 4$             &     $h=0.4, 0.2, 0.1$       & $L_x = 128$,$L_y=58$              & Taylor, Cons. FDS  &               10    &   Zero  \\
	   \hline
			\hline 
		\end{tabular}
\caption{Parameter table for the numerical tests}
\label{tableP}
\end{table}

\subsection{Convergence Rate for the Conservative FDS}

Table \ref{tableC} measures the convergence speed based on the results of the numerical solution $u_h$. The Conservative FDS applies only second approximation order, thus $p=2$ for Test 1 and Test 2 in the first column. Three nested meshes are used with different step sizes which are present in the second column. The next two columns show the errors $\Vert u_{h,\tau} - u^*_{(h,\tau)/2} \Vert_\kappa$, $\Vert  u_{(h,\tau)/2} - u_{(h,\tau)/4} \Vert_\kappa$ and convergence speed $\xi$ from \rf{Runge} which are measured in $L_2$ and infinity norms. 
%C
\begin{table}[ht]
\centering
\small
		\begin{tabular}{||c|l|ll|ll||}
			\hline
			\hline
      \multirow{2  }{*}{FDS}        & \multirow{2  }{*}{$h$, $\tau$}  & \multirow{2  }{*}{errors $E_i$in$L_2$}  &Conv.& \multirow{2  }{*}{errors $E_i$in$L_\infty$}  &Conv.  \\
	                                        &                                                     &                                                                 &  Rate &                                                                       & Rate \\
   			\hline 
					\hline 
  $\beta=3$                &0.2, 0.001         &                    &                &                  &                   \\
   c=0.45                     &0.1, 0.0005         & 0.989422   &                & 1.043649  &                   \\
     $O(h^2 + \tau^ 2)$ &0.05, 0.00025  &0.344818    & 1.52       & 0.355517   &   1.55   \\
	   \hline
			\hline 
       $\beta=1$           & 0.4, 0.002       &                   &           &                 &   \\
                  c=0.9       & 0.2, 0.001        & 0.200424   &          &0.072726  &   \\
  $O(h^2+ \tau^2)$  & 0.1, 0.0005       & 0.047899   & 2.06  &0.021451  & 1.76 \\
	   \hline
			\hline 
		\end{tabular}
		\caption{Convergence speed for the solution obtained by the Conservative FDS with zero boundary conditions and approximation errors $O(h^{2} + \tau^2 )$. Errors $E_i$ are measured in $L_2$ and $L_\infty$ norms}
\label{tableC}
\end{table}

%D
\begin{table}[ht]
\centering
\small
		\begin{tabular}{||c|l|ll|ll||}
			\hline
			\hline
      \multirow{2  }{*}{FDS}        & \multirow{2  }{*}{$h$, $\tau$}  & \multirow{2  }{*}{errors $E_i$in$L_2$}  &Conv.& \multirow{2  }{*}{errors $E_i$in$L_\infty$}  &Conv.  \\
	                                        &                                                     &                                                                 &  Rate &                                                                       & Rate \\
   			\hline 
					\hline 
  $\beta=3$                &0.2, 0.005         &                    &                &                  &                   \\
   c=0.45                     &0.1, 0.0025         & 0.044442   &                & 0.444423  &                   \\
     $O(h^2 + \tau^ 2)$ &0.05, 0.00025  & 0.007831   & 2.50       & 0.110750  & 2.00   \\
	   \hline
			\hline 
       $\beta=1$           & 0.4, 0.002       &                   &           &                 &   \\
                  c=0.9       & 0.2, 0.001        & 0.051409   &          &0.363515  &   \\
  $O(h^2+ \tau^2)$  & 0.1, 0.0005       & 0.008939   & 2.52  &0.089393  & 2.02  \\
	   \hline
			\hline 
		\end{tabular}
		\caption{ Convergence speed for the discrete Energy using the Conservative FDS with zero boundary conditions and approximation errors $O(h^{2} + \tau^2 )$. Errors $E_i$ are measured in $L_2$ and $L_\infty$ norms. }
\label{tableD}
\end{table}

Table \ref{tableD} is analogous to Table \ref{tableC} and measures the convergence speed based on the results of the discrete Energy \rf{ex-en}. The convergence rate of both solution and Energy is found to be in correspondence with the second order approximation that is used.

\subsection{Convergence Rate for the TS Approach with Method of Lines}

Table \ref{tableA} measures the convergence rate based on the results of the numerical solution $u_h$. The Taylor method applies second and fourth approximation order, thus $p=2,4$ for Test 1, Test 2 and Test 3 in the first column. The parameters in \rf{orgBsqSol} for the three tests are defined as:
\be
        k_1 = 1/3,  \;\; k_2 = -1/2,  \;\; b_1 = b_2 = 0,  \;\; a_1 = a_2 = a_{12} = 1.
\ee
 Three nested meshes are used with different step sizes which are present in the second column. The exact solution \rf{orgBsqSol} is calculated directly over the finer grids in order to obtain $u^*_{h/2}$ and $u^*_{h/4}$ from \rf{Runge}. The next two columns show the errors $\Vert u_{h} - u^*_{h/2} \Vert_\kappa$, $\Vert  u_{h/2} - u^*_{h/4} \Vert_\kappa$ and convergence speed $\xi$ from \rf{Runge} which are measured in $L_2$ and infinity norms. The convergence results for the Taylor solution correspond to the approximation order that is used. 
%A
\begin{table}[ht]
\centering
\small
		\begin{tabular}{||c|l|ll|ll||}
			\hline
			\hline
      \multirow{2  }{*}{FDS}        & \multirow{2  }{*}{$h$, $\tau$}  & \multirow{2  }{*}{errors $E_i$in$L_2$}  &Conv.& \multirow{2  }{*}{errors $E_i$in$L_\infty$}  &Conv.  \\
	         &                    &                               & Rate   &                                        & Rate \\
   			\hline 
					\hline 
                    &0.4, 2.5e-04          &              &              &                     &      \\
                    &0.2, 2.5e-04          &0.0017022652 &            &0.0012462726    &       \\
     $O(h^2 + \tau^ 2)$ &0.1, 2.5e-04   & 0.0002766992 & 2.62    &0.0002926296    &  2.09      \\
			\hline 
            &0.4, 1.0e-06        &             &            &           &   \\
                      &0.2, 1.0e-06       &  0.0000091188  &            &0.0000070642 &   \\
  $O(h^4+ \tau^4)$ &0.1, 1.0e-06  &0.0000003715   &4.61  &0.0000003563  & 4.31 \\
			\hline
                   &0.8, 6.25e-05    &            &               &             &    \\
                        &0.4, 6.25e-05     & 0.0001962184   &        &  0.0001058293   &   \\
       $O(h^4+ \tau^4)$ &0.2, 6.25e-05   &0.0000104438 & 4.23  & 0.0000068085  & 3.96  \\
    \hline
			\hline 
		\end{tabular}
		\caption{Convergence rate for the solution obtained by Taylor method with zero boundary conditions and different approximation errors $O(h^{2} + \tau^2 )$ and $O(h^{4} + \tau^4 )$. Errors $E_i$ are measured in $L_2$ and $L_\infty$ norms}
\label{tableA}
\end{table}

%------------------------------------------------------------------------------------------------------

\begin{table}[ht]
\centering
\small
		\begin{tabular}{||c|l|ll|ll||}
			\hline
			\hline
      \multirow{2  }{*}{FDS}        & \multirow{2  }{*}{$h$, $\tau$}  & \multirow{2  }{*}{errors $E_i$in$L_2$}  &Conv.& \multirow{2  }{*}{errors $E_i$in$L_\infty$}  &Conv.  \\
	         &                    &                               & Rate   &                                        & Rate \\
   			\hline 
					\hline 
                    &0.8, 1e-02          &              &              &                     &      \\
                    &0.4, 1e-02          &0.002374647 &            &0.0010459783     &       \\
     $O(h^4 + \tau^ 4)$ &0.2, 1e-02  & 100.80978075  & none    &55.062999033 &      none      \\
			\hline 
                    &0.8, 1e-03          &              &              &                     &      \\
                    &0.4, 1e-03          &0.221571623 &            & 0.084116     &       \\
     $O(h^4 + \tau^ 4)$ &0.2, 1e-03  & 0.000169619  & none   &0.000104696 &     none      \\
			\hline
                    &0.8, 1e-04          &              &              &                     &      \\
                    &0.4, 1e-04          &0.0001926500 &            & 0.0001021745    &       \\
     $O(h^4 + \tau^ 4)$ &0.2, 1e-04  & 0.0000153906  & 3.65   &0.0000105484 &     3.28      \\
    \hline
                    &0.8, 1e-05          &              &              &                     &      \\
                    &0.4, 1e-05          &0.0002018152 &            & 0.0001109456    &       \\
     $O(h^4 + \tau^ 4)$ &0.2, 1e-05  & 0.0000085260 & 4.57   &0.000006186 &      4.16      \\
    \hline
                    &0.8, 1e-06          &              &              &                     &      \\
                    &0.4, 1e-06          &0.0002028415 &            & 0.0001118226    &       \\
     $O(h^4 + \tau^ 4)$ &0.2, 1e-06  & 0.0000091188 & 4.48  &0.0000070642 &      3.98    \\
    \hline
                    &0.8, 1e-07          &              &              &                     &      \\
                    &0.4, 1e-07          &0.0002029452 &            & 0.0001119103    &       \\
     $O(h^4 + \tau^ 4)$ &0.2, 1e-07  & 0.0000091906 & 4.46  &0.0000071520 &       3.97    \\
    \hline
			\hline 
		\end{tabular}
		\caption{Effect of decreasing the time step for $O(h^{4} + \tau^4 )$. Errors $E_i$ are measured in $L_2$ and $L_\infty$ norms}
\label{tableConvSeq}
\end{table}
Table \ref{tableConvSeq} shows the effect of modifing the time step on the solution convergence. By increasing it $\tau>1e-4$, the stability of the Taylor Series method breaks as the $L_2$ and $L_\infty$ errors become much higher compared to the maximum of the solution wave. The solution diverges and becomes non-smooth (e.g. the graph is jagged for T=30) and there is no convergence. On the other hand, if the step $\tau<1e-5$ decreases, the accumulation of errors from numerical calculations amplifies, and thus the convergence decreases.

Thus the  has analogous structure as Table \ref{tableA} and measures the convergence rate based on the results of the discrete Energy \rf{ex-en}. The convergence rate of the Energy corresponds to the approximation order that is used. Only for Test 2 and $O(h^6 + \tau^6)$ the convergence results $4.45$ and $4.68$ for $L_2$ and infinity norms are slightly less than expected. Analogously to the previous Table \ref{tableA} where the $L_2$ is out of expected order, this result could be due to the zero boundary conditions. The evaluation of the Energy functional \rf{ex-en} involves quadrature formulas which implies similar behavour to the $L_2$ norm. Thus, the infinity norm results in a similar convergence rate $4.68$ for the problematic case.
% When using zero boundary conditions all points on the finite difference stencil that are outside the numerical domain $\Omega_h$ are zeros. This creates a jagged solution surface on the boundary.
For the $O(h^2 + \tau^2)$ case ( both Test 1 and Test 2), much smaller time step  $\tau = h/200$ is used. This results in solutions which are similar in shape and form for the three nested meshes. If the step is chosen bigger (e.g. $\tau = h/10$), then the difference in the solution shape between $O(h^2 + \tau^2)$ and the other $O(h^4 + \tau^4)$ and $O(h^6 + \tau^6)$ approximations is much larger. 

%B
\begin{table}[ht]
\centering
\small
		\begin{tabular}{||c|l|ll|ll||}
			\hline
			\hline
      \multirow{2  }{*}{FDS}        & \multirow{2  }{*}{$h$, $\tau$}  & \multirow{ 2 }{*}{errors $E_i$in$L_2$}  &Conv.& \multirow{2  }{*}{errors $E_i$in$L_\infty$}  &Conv.  \\
	                                        &                                                     &                                                                 &  Rate &                                                                       & Rate \\
   			\hline 
					\hline 
  $\beta=3$                &0.2, 0.001         &              &            &                     &      \\
   c=0.45                     &0.1, 0.0005         &0.044442  &            &0.444425 &       \\
     $O(h^2 + \tau^ 2)$ &0.05, 0.00025  & 0.007831 & 2.50      & 0.110750     & 2.00      \\
			\hline 
  $\beta=3$               &0.2, 0.02       &                &            &                     &      \\
   c=0.45                    &0.1, 0.01      &0.018288 &            &0.072718   &       \\
     $O(h^4+ \tau^4)$ &0.05, 0.005  &0.000945 &4.27    &0.002997   &4.60      \\
			\hline 
  $\beta=3$               &0.2, 0.02       &                &            &                      &            \\
     c=0.45                 &0.1, 0.01        &0.027425 &            &  0.122934    &           \\
     $O(h^6+ \tau^6)$ &0.05, 0.005 &0.000318 & 6.42     & 0.001467     &6.38   \\
	   \hline
			\hline 
       $\beta=1$       &0.4, 0.002        &             &            &           &   \\
                  c=0.9    &0.2, 0.001       &  0.046343   &            &0.352955 &   \\
  $O(h^2+ \tau^2)$ &0.1, 0.0005   &0.007430   &2.64  &0.086470  & 2.02 \\
			\hline
      $\beta=1$               &0.4, 0.04    &            &               &             &    \\
       c=0.9                     &0.2, 0.02     & 0.023067   &        &  0.040550   &   \\
       $O(h^4+ \tau^4)$ &0.1, 0.01   &0.001411 & 4.03   & 0.003203  & 3.66  \\
    \hline
  $\beta=1$     &0.4, 0.04   &            &          &                  &      \\
      c=0.9                    &0.2, 0.02   &0.010898 &           & 0.032597      &       \\
     $O(h^6+ \tau^6)$ &0.1, 0.01 & 0.000496 &4.45 & 0.001266  & 4.68        \\
	   \hline
			\hline 
		\end{tabular}
		\caption{Convergence speed for the Energy using Taylor method with zero boundary conditions and different approximation errors $O(h^{2} + \tau^2 )$, $O(h^{4} + \tau^4 )$ and $O(h^{6} + \tau^6 )$. Errors $E_i$ are measured in $L_2$ and $L_\infty$ norms.}
\label{tableB}
\end{table}

\subsection{Numerical Results for the Mass and Energy}

Figure \ref{Test1En} shows the discrete Mass and Energy for Test 1 calculated using solutions from the Conservative FDS and Taylor method. Figure \ref{Test2En} shows analogous results for Test 2. In case of $O(|h|^2 +\tau^2)$ the Mass and Energy realized by both methods for Test 1 and Test 2 overlap (dark line and blue circles). The Energy graphs are constant functions over the time interval.
\begin{figure}[ht]\vspace{0.2cm}
	\begin{minipage}[b]{0.4\linewidth}
		 \includegraphics[width=\linewidth]{Mass_bt3_c045_h005_Taylor_Conservative.png}
	\end{minipage}	
	\begin{minipage}[b]{0.4\linewidth}
		\includegraphics[width=\linewidth]{Energy_bt3_c045_h005_Taylor_Conservative.png}	
	\end{minipage}
\caption{The Mass (left) and Energy (right) of the solution for Test 1, $O(|h|^2 + \tau^2)$ and $T=10$.}
\label{Test1En}
\end{figure}

The Mass increases slightly over the time interval but the gain is neglectable compared to the initial value. For Test 1 the increase with respect to the initial Mass is $0.33\%$ and for Test 2 is $1.8\%$.

\begin{figure}[ht]\vspace{0.2cm}
	\begin{minipage}[b]{0.4\linewidth}
		 \includegraphics[width=\linewidth]{Mass_bt1_c090_h010_Taylor_Conservative.png}
	\end{minipage}	
	\begin{minipage}[b]{0.4\linewidth}
		\includegraphics[width=\linewidth]{Energy_bt1_c090_h010_Taylor_Conservative.png}
		
	\end{minipage}
\caption{The Mass (left) and Energy (right) of the solution for Test 2, $O(|h|^2 + \tau^2)$ and $T=10$.}
\label{Test2En}
\end{figure}

For the Taylor Method, the discrete Mass and Energy in Figure \ref{Test1En} and \ref{Test2En} are calculated with three different approximations, namely, trapezoidal formula \rf{quadr2} with $O(h^2)$, Simpson's Rule \rf{quadr4} with $O(h^4)$ and Boole's Rule \rf{quadr6-2D} with $O(h^6)$. It is observed that increasing the order of approximation does not affect the gain for the Mass. It is also confirmed that decreasing the step size $h$ produces similar result, i.e. the graph is slightly shifted upwards or downwards. In the next computations, in order to validate the proper behaviour of the Mass, only the size of the domain is changed whereas the approximation order and discrete steps are fixed.
%----------------------------------------------------------------------------------------------------------------------------------------------
\iffalse
\begin{figure}[ht]\vspace{0.4cm}
	\begin{minipage}[b]{0.33\linewidth}
		 \includegraphics[width=\linewidth]{figures/Mass_bt1_c090_h010_x3O.png}
	\end{minipage}	
	\begin{minipage}[b]{0.33\linewidth}
		\includegraphics[width=\linewidth]{figures/Energy_bt1_c090_h010_x3O.png}
		
	\end{minipage}
\caption{The Mass (left) and Energy (right) of the Taylor solution for Test 1, $O(|h|^2 + \tau^2)$, $O(|h|^4 + \tau^4)$, $O(|h|^6 + \tau^6)$ and $T=10$.}
\label{Test1TEn}
\end{figure}
\begin{figure}[ht]\vspace{0.4cm}
	\begin{minipage}[b]{0.33\linewidth}
		 \includegraphics[width=\linewidth]{figures/Mass_bt3_c045_h005_x3O.png}
	\end{minipage}	
	\begin{minipage}[b]{0.33\linewidth}
		\includegraphics[width=\linewidth]{figures/Energy_bt1_c090_h010_x3O.png}
		
	\end{minipage}
\caption{The Mass (left) and Energy (right) of the Taylor solution for Test 2, $O(|h|^2 + \tau^2)$, $O(|h|^4 + \tau^4)$, $O(|h|^6 + \tau^6)$ and $T=10$.}
\label{Test2TEn}
\end{figure}
\fi
%----------------------------------------------------------------------------------------------------------------------------------------------


\subsection{Numerical Results for the shape and maximum of the solution}

The following paragraph discusses the shape of the solution obtained by the Conservative FDS and Taylor method. The set up for the calculations is described in Table \ref{tableP} when $p=2$, i.e. both Test 1 and Test 2 are done on three nested meshes. For each Test and mesh both solution techniques are applied with second approximation order. Let us denote with $uC$ and $uT$ the solutions obtained by the Conservative scheme \rf{consFDS} and the TS expansion \rf{TSe}. Figures \ref{Test1_Diff} and \ref{Test2_Diff} show the solution difference for Test 1 and Test 2 respectively. The pictures cover only the center of the domain $\Omega_h$ where the values of the difference are higher. This is one nineth of all mesh points.
\iffalse
\begin{figure}[ht]\vspace{0.4cm}
	\begin{minipage}[b]{0.32\linewidth}
		 \includegraphics[width=\linewidth]{figures/compare_30_bt3_c045_h020.png}
	\end{minipage}	
	\begin{minipage}[b]{0.32\linewidth}
		\includegraphics[width=\linewidth]{figures/compare_30_bt3_c045_h010.png}
	\end{minipage}	
	\begin{minipage}[b]{0.32\linewidth}		
		\includegraphics[width=\linewidth]{figures/compare_30_bt3_c045_h005.png}
	\end{minipage}
\caption{Difference $uC - uT$ between solutions from Conservative Scheme and TS approach at time $t=10$, $O(|h|^2 + \tau^2)$ for Test 1. From Left to right $h=0.2, 0.1, 0.05$.}
\label{Test1_Diff}
\end{figure}

\begin{figure}[ht]\vspace{0.4cm}
	\begin{minipage}[b]{0.32\linewidth}
		\includegraphics[width=\linewidth]{figures/compare_128_bt1_c09_h040.png}
	\end{minipage}	
	\begin{minipage}[b]{0.32\linewidth}
		\includegraphics[width=\linewidth]{figures/compare_128_bt1_c09_h020.png}
	\end{minipage}	
	\begin{minipage}[b]{0.32\linewidth}
		\includegraphics[width=\linewidth]{figures/compare_128_bt1_c09_h010.png}
	\end{minipage}
\caption{Difference $uC - uT$ between solutions from Conservative Scheme and TS approach at time $t=10$, $O(|h|^2 + \tau^2)$ for Test 2. From Left to right $h=0.4, 0.2, 0.1$.}
\label{Test2_Diff}
\end{figure}

\begin{figure}[ht]\vspace{0.2cm}
\centering
	\begin{minipage}[b]{0.30\linewidth}
		\includegraphics[width=\linewidth]{figures/solution_30x45_bt3_c045_T0.png}
	\end{minipage}	
	\begin{minipage}[b]{0.30\linewidth}
		\includegraphics[width=\linewidth]{figures/solution_30x45_bt3_c045_T6.png}
	\end{minipage}	
	\begin{minipage}[b]{0.30\linewidth}
		 \includegraphics[width=\linewidth]{figures/solution_30x45_bt3_c045_T12.png}
	\end{minipage}
	\begin{minipage}[b]{0.30\linewidth}
		\includegraphics[width=\linewidth]{figures/solution_30x45_bt3_c045_T18.png}
	\end{minipage}	
	\begin{minipage}[b]{0.30\linewidth}
		 \includegraphics[width=\linewidth]{figures/solution_30x45_bt3_c045_T24.png}
	\end{minipage}
	\begin{minipage}[b]{0.30\linewidth}
		 \includegraphics[width=\linewidth]{figures/solution_30x45_bt3_c045_T30.png}
	\end{minipage}
\caption{Numerical solution of single wave for $\beta=3$ and $c = 0.45$ at times $t=0,6,12,18,24,30$.}
\label{Wave1}
\end{figure}

\begin{figure}[ht]\vspace{0.2cm}
\centering
	\begin{minipage}[b]{0.30\linewidth}
		\includegraphics[width=\linewidth]{figures/solution_128x90_bt1_c090_T0.png}
	\end{minipage}	
	\begin{minipage}[b]{0.30\linewidth}
		\includegraphics[width=\linewidth]{figures/solution_128x90_bt1_c090_T6.png}
	\end{minipage}	
	\begin{minipage}[b]{0.30\linewidth}
		 \includegraphics[width=\linewidth]{figures/solution_128x90_bt1_c090_T12.png}
	\end{minipage}
	\begin{minipage}[b]{0.30\linewidth}
		\includegraphics[width=\linewidth]{figures/solution_128x90_bt1_c090_T18.png}
	\end{minipage}	
	\begin{minipage}[b]{0.30\linewidth}
		 \includegraphics[width=\linewidth]{figures/solution_128x90_bt1_c090_T24.png}
	\end{minipage}
	\begin{minipage}[b]{0.30\linewidth}
		 \includegraphics[width=\linewidth]{figures/solution_128x90_bt1_c090_T30.png}
	\end{minipage}
\caption{Numerical solution of single wave for $\beta=1$ and $c = 0.9$ at times $t=0,6,12,18,24,30$.}
\label{Wave2}
\end{figure}
\fi
The figures are for visual representation. On the contrary, Table \ref{tableF} shows the difference $||uC - uT||_\kappa$ using $L_2$ and infinity norms. The last column of the Table is for the infinity norm of the solution $||uC||_{L_\infty}$ (which is the wave maximum) measured by the Conservative FDS. It is observed that decreasing the step sizes $h$ and $\tau$ results in a smaller difference.  Furthermore the percentage difference when using the infinity norm
$$\frac{ ||uC - uT||_{L_\infty}} { ||uC||_{L_\infty} } \times 100$$
varies in the inteval $[0.0001\%, 0.69\%]$ for all rows in the table. Analogous results are obtained for the percentage difference when using the $L_2$ norm and thus are omitted. The difference between the wave shapes obtained by Conservative FDS and the Taylor Series approach with Method of lines is negligible. It is expected that $||uC - uT||_\kappa$ goes to zero when $h$ and $\tau$ are infinitely small.

%F
\begin{table}[ht]
\centering
\small
		\begin{tabular}{||c|l|l|l|l||}
			\hline
			\hline
      FDS        &$h$, $\tau$  &   $||uC - uT||$  in $L_2$     &  $||uC - uT||$ in $L_\infty$ & $||uC||$ in $L_\infty$ \\
   			\hline 
					\hline 
  $\beta=3$                   &0.2, 0.001         &  1.749e-05      &  1.965e-05  & 1.315448     \\
   c=0.45                        &0.1, 0.0005        &  8.109e-06       & 8.274e-06 &  1.862688     \\
     $O(h^2 + \tau^ 2)$ &0.05, 0.00025     & 2.460e-06         &2.502e-06  &   2.013184   \\
			\hline 
			\hline 
       $\beta=1$          &0.4, 0.002        & 0.009981     & 0.004560 & 0.656747   \\
                  c=0.9      &0.2, 0.001        & 0.005047      & 0.002373  & 0.673901   \\
  $O(h^2+ \tau^2)$ &0.1, 0.0005         & 0.002521      &0.001117 & 0.672231   \\
			\hline
	   \hline
			\hline 
		\end{tabular}
		\caption{Difference $uC - uT$ between solutions from Conservative Scheme and TS approach at time $t=10$ with $O(|h|^2 + \tau^2)$ approximation errors. Differences are measured in $L_2$ and $L_\infty$ norms.}
\label{tableF}
\end{table}

The following paragraph discusses the preservation of the solution's shape over the time interval $[0, 10]$. The set up for the calculations is described in Table \ref{tableP} when using the Taylor method, i.e. $p=2, 4, 6$ and for each approximation order, three nested meshes are used. For more details refer to Table \ref{tableG}. Here, the first column is for the Test case. The second describes the spatial step size $h$. The third and fourth columns present the solution difference at times $t=0$ and $t=10$ in $L_2$ and $L_\infty$ norms. The wave travels a distance which is equal to the speed $c$ multiplied by the end time $T$. Thus, the maximum of the solution at time $t=10$ is located on $(0, 10 c) \in \Omega$. The maximum at time $t=0$ is located on $(0, 0) \in \Omega$. Unfortunately, for Test 1, $h=0.2$ and Test 2, $h=0.4$ the maximum for $t=10$ is not inside the mesh. Further calculations are done to shift the wave along the $y$ axis so that the maximum fall under the closest available meshpoint. For Test 1 ($c=0.45$) the wave travels additional distance of 0.1 for time 2/9 which shifts the maximum to a position of $(0, 4.6) \in \Omega_h$. For Test 2 ($c=0.9$) the wave travels additional distance of 0.2 for time 2/9 which shifts the maximum to a position of $(0, 9.2) \in \Omega_h$. The subtraction of two solutions $u^{(0)} - u^{(N_t)}$ results in a matrix with the following coefficients:
$$ \delta_{i,j} = u_{i,j}^{(0)} - u_{i,j+cT/h}^{(N_t)},$$
where $0 < i < N_x$ and $0 < j < N_y - cT/h$. 

%G
\begin{table}[ht]
\centering
\small
		\begin{tabular}{||c|l|l|l||}
			\hline
			\hline
      FDS        & $h$, $\tau$  & $||u^{(0)} - u^{(N_t)}||$ in $L_2$  & $||u^{(0)} - u^{(N_t)}||$ in $L_\infty$   \\
   		\hline 
			\hline
  $\beta=3$                &0.2, 0.001\footnote{Position of maximum is further adjusted to fit inside $\Omega_h$.}            & 1.494351 & 1.533173    \\
   c=0.45                     &0.1, 0.0005          & 0.466991 & 0.484011       \\
     $O(h^2 + \tau^ 2)$ &0.05, 0.00025   & 0.127641 & 0.132504      \\
			\hline 
  $\beta=3$               &0.2, 0.02 $^{\text{a}}$      &0.220560 & 0.230486       \\
   c=0.45                    &0.1, 0.01      &0.013762 & 0.014391        \\
     $O(h^4+ \tau^4)$ &0.05, 0.005&0.000877 & 0.000917         \\
			\hline 
  $\beta=3$               &0.2, 0.02 $^{\text{a}}$       &  0.035965 & 0.038039        \\
     c=0.45                 &0.1, 0.01        &0.000600 & 0.000633       \\
     $O(h^6+ \tau^6)$ &0.05, 0.005 &0.000010 & 0.000010          \\
	   \hline
			\hline 
       $\beta=1$       &0.4, 0.002 $^{\text{a}}$       & 0.244208 & 0.103833 \\
                  c=0.9    &0.2, 0.001       &  0.057175 & 0.026919  \\
  $O(h^2+ \tau^2)$ &0.1, 0.0005   & 0.013938 & 0.006622  \\
			\hline
      $\beta=1$               &0.4, 0.04 $^{\text{a}}$    &0.028546 & 0.012203 \\
       c=0.9                     &0.2, 0.02     & 0.001757 & 0.000958     \\
       $O(h^4+ \tau^4)$ &0.1, 0.01   & 0.000112 & 0.000061   \\
    \hline
  $\beta=1$                  &0.4, 0.04 $^{\text{a}}$   &0.006415 & 0.002792  \\
      c=0.9                    &0.2, 0.02   &0.000112 & 0.000065     \\
     $O(h^6+ \tau^6)$ &0.1, 0.01 & 0.000002 & 0.000001         \\
	   \hline
			\hline 
		\end{tabular}
		\caption{Difference $||u^{(0)} - u^{(N_t)}||_\kappa$ in $L_2$ and $L_\infty$ norms between solutions at time $t=0$ and $t=10$. TS approach is used for all measurements in the table.}
\label{tableG}
\end{table}

Table \ref{tableG} shows that if the step size $h$ (and $\tau$ respectively) decreases, then the difference also decreases. Furthermore, the shape of the wave is preserved better while the approximation order $p$ increases. Thus, it is expected that $||u^{(0)} - u^{(N_t)}||_\kappa$ goes to zero when $h$ and $\tau$ are infinitely small. 

\begin{figure}[H]\vspace{0.2cm}
	\centering
	\begin{minipage}[b]{0.40\linewidth}
		\includegraphics[width=\linewidth]{maximum_30_T30_bt3_c045_h005.png}
	\end{minipage}	
	\begin{minipage}[b]{0.40\linewidth}
		 \includegraphics[width=\linewidth]{maximum_30_T30_bt1_c090_h020.png}
	\end{minipage}
\caption{Evolution of the maximum over a larger time interval $[0, 30]$ for Test 1 (left panel) and Test 2 (right panel).}
\label{Maximum}
\end{figure}

The following paragraph discusses the maximum of the solution over a larger time interval $[0, 30]$. The set up for the two calculations is described in Table \ref{tableP} where 
$p=6$, $T=30$ and Taylor method is applied with $h=0.05$ and $\tau = 0.001$ for Test 1 and $h=0.2$ and $\tau=0.02$ for Test 2. Also the size of the computational box $\Omega_h$ is extended appropriately along the $y$ axis to compensate for the wave shift. Figure \ref{Maximum} shows that the maximum is stable for both tests over a larger time interval $[0, 30]$. For each iteration step, the exact position of the maximum is not always located on a mesh point in $\Omega_h$. This produces a jagged graph which is well expressed on the right panel and also present on the left panel in Figure \ref{Maximum}. Solution shapes are shown in Figures \ref{Wave1} and \ref{Wave2}.

\section{Conclusion}
The 2D BPE is solved using Taylor method with high approximation orders $O(|h|^p + \tau^p)$, $p=2, 4, 6$. The results are compared with the Conservative FDS for $O(|h|^2 + \tau^2)$. Solutions from both methods are qualitatively and quantitatively very similar. Runge's Rule show that the discrete solution and Energy converge for all numerical calculations described in Table \ref{tableP}. The Mass and Energy of the Taylor method are saved with high accuracy over the time interval $[0, 10]$. The discrete Energy is a constant function of the time variable. It is hard to measure the Mass which is defined as an infinite integral of the solution. Computational tests show that the discrete Mass slightly increases as the wave moves. Nevertheless, it is obtained that the gain for the discrete Mass diminishes when the domain $\Omega_h$ grows. The numerical solutions for wave speeds near the upper limit $c_{max} = min(1, \sqrt{\beta_2/\beta_1})$ are stable in form over the time interval $[0, 10]$. It is seen that the change in the shape decreases when the step sizes $h$ and $\tau$ contract or the approximation order $p$ increases. The behaviour of the maximum over a long period of time $[0, 30]$ is preserved with small errors.

%\begin{acknowledgments}
%The work of the second author has been partially supported by the Bulgarian Science Fund under grant K$\Pi$-06-H22/2.
%\end{acknowledgments}

%\nocite{*}
%\bibliography{aipsamp}% Produces the bibliography via BibTeX.

\begin{thebibliography}{99} \normalsize

\bibitem{refHyp} Angelow K., Comparison between two numerical methods for solution of 2D Boussinesq paradigm equation, \emph{AIP Conference Proceedings}, \textbf{2522}, (2022), 090001

\bibitem{ref16} Angelow, K., Kolkovksa, N., Numercal Study of Traveling Wave Solutions to 2D Boussinesq Equation, {\it Serdica J. Computing}, \textbf{13} (2019), 1-16.

\bibitem{ref0} Boussinesq, J.V., Theorie des ondes et des remous qui se propagent le long d'un canal rectangulaire horizontal, en communiquant au liquide contenu dans ce canal des vitesses sensiblement pareilles de la surface au fond.  {\it Journal de Mathematiques Pures et Appliquees}, \textbf{17} (1872), 55-108.

\bibitem{ref21} Chertok, A., Christov, C.I., Kurganov, A., Central-Upwind Schemes for the Boussinesq Paradigm Equations,
{\it Computational Science and High Performance Computing IV, Notes Numer. Fluid Mech.}, \textbf{113} (2011), 267-281.

\bibitem{ref13}  Christou M. , Christov C.I.,
Galerkin spectral method for the 2D solitary waves of Boussinesq paradigm equation,
In: {\it Applications of Mathematics in Technical and Natural Sciences, Sozopol (Bulgaria)},
\emph{AIP Conference Proceedings}, \textbf{1186}, Issue 1 (2009), 217-225.

\bibitem{ref14}  Christou M. , Christov C.I.,
Fourier Galerkin method for 2D solitons of Boussinesq equation,
{\it Mathematics and Computers in Simulation} \textbf{74} (2007), 82-92.

\bibitem{ref1} Christov, C.I., An energy-consistent dispersive shallow-water model,  {\it Wave Motion}, \textbf{34} (2001), 161-174.

\bibitem{ref15} Christov, C.I., Choudhury, J., Perturbation solution for the 2D Boussinesq equation, {\it Mech. Res. Commun.}, \textbf{38} (2011), 274-281.

\bibitem{ref4} Christov, I., Christov, C.I., Physical dynamics of quasi-particles in nonlinear wave equations,
{\it Physics Letters A}, \textbf{372}, Issue 4 (2008),  841-848.

\bibitem{ref20} Christov, C.I., Kolkovska, N., Vasileva, D., On the Numerical Simulation of Un-
steady Solutions for the 2D Boussinesq Paragigm Equation,
{\it In: I. Dimov, S. Dimova, N. Kolkovska (Eds.), Numerical Methods and Applications 2010},
\emph{Conference Proceedings}, \textbf{6046} (2010), 386–394.

\bibitem{ref23} Dimova M., Vasileva D., Comparison of Two Numerical Approaches to Boussinesq Paradigm Equation, 
{\it Lect. Notes Comput. Sci.}, \textbf{8236} (2013), 255-262.

\bibitem{forn}
Fornberg, B., Generation of Finite Difference Formulas on Arbitrarily Spaced Grids, 
Math. Comput., 51(1988),  699 -- 706.

\bibitem{ref25} Kolkovska N., Two families of finite difference schemes for multidimensional Boussinesq paradigm equation, In:
{\it Applications of Mathematics in Technical and Natural Sciences,  Sozopol (Bulgaria)},
\emph{AIP Conference Proceedings}, \textbf{1301} (2010), 395.

\bibitem{ref22} Kolkovska, N., Angelow K., A Multicomponent Alternating Direction Method for Numerical Solving of Boussinesq Paradigm Equation,
In: {\it  I. Dimov, I., Farago, I., Vulkov, L. (eds.) NAA 2012},
\emph{Conference Proceedings}, \textbf{8236} (2013), 371–378.

\bibitem{samarski} Samarskii, A., The Theory of Difference Schemes, Marcel Dekker Inc., New York, 2001.
%
\end{thebibliography}
\end{document}
%
% ****** End of file aipsamp.tex ******
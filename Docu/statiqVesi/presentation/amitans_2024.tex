\documentclass{beamer}


%\beamertemplateshadingbackground{yellow!100}{white}
\usepackage[english]{babel}
\usepackage{amsfonts,amsmath}
\usepackage{graphicx}
\usepackage{sansmathaccent}
\pdfmapfile{+sansmathaccent.map}
\usepackage{multirow}
%\usetheme{Warsaw}
\usetheme[secheader]{Madrid}

\usepackage{multimedia}
\logo{\includegraphics[height=0.5cm]{imi.jpg}}

\newcommand{\be}{\begin{equation}}
\newcommand{\ee}{\end{equation}}
\newcommand{\rf}[1]{(\ref{#1})}
\newcommand{\RR}{\mathbb{R}}
\newtheorem{thm}{Theorem}
\newtheorem{lm}{Lemma}

%\def\ra#1\{\renewcommand{\arraystretch}{#1}
\begin{document}
 %[propagating wave solutions to the 2D BPE]
\title{Numerical Study of the good Boussinesq Equation}

\author[angelow@math.bas.bg]{{\underline{Krassimir Angelow}}, Veselina Vucheva}
\institute[IMI -- BAS]{Institute of Mathematics and Informatics\\ Bulgarian Academy of Sciences, Sofia, Bulgaria,\\ e-mail: angelow@math.bas.bg}
\date[2021]{AMITANS, June 21-26  2024,  Albena, Bulgaria}


%---------- frame 01 ----------------
\begin{frame}
 \titlepage

\end{frame}

%---------- frame 02 ----------------
\begin{frame}
\tableofcontents 
\setbeamertemplate{table of contents shaded}[default]
\section{Good Boussinesq Equation}
\section{Numerical Methods}
\subsection{Conservative Finite Difference Scheme (FDS) }
\subsection{Taylor Series (TS) Approach with Method of Lines }


\section{Convergence}

\section{Comparison Between numerical solution (TS or Conservative Scheme) with the exact solution}

\section{Discrete energy and mass}

%\tableofcontents 
\end{frame}

%---------- frame 03 ----------------
\begin{frame}
\frametitle{Boussinesq Paradigm Equation}


The one dimensional good Boussinesq equation is

\begin{align}\label{problem}
 \frac{\partial^2 u}{\partial t^2}= \Delta u -  \Delta^2 u +  3 \Delta (u^2)
\\
u(x,0) = u_0(x), \quad \frac{\partial u}{\partial t}(x,0)=u_1(x), \nonumber
\\
u(x,t) \rightarrow 0, \quad \Delta u(x, t) \rightarrow 0 \quad \text{for} \quad x \rightarrow \infty \nonumber
\end{align}
where $\Delta$ is the second space derivative. The domain of the unknown function $u$ is defined by:
\be
 u:\Omega \times [0, T] \rightarrow R
\ee
where $\Omega \in R$ and $T>0$.
\end{frame}

%---------- frame 04 ----------------


\begin{frame}
\frametitle{Good Boussinesq Equation}
%The results and observations for the good Boussinesq equation \rf{orgBsq} could be further used in order to find a two wave solitary solution for the 1D and 2D BPE.
The goal of this talk is to show that:
\begin{itemize}
 \item one could obtain a numerical solution for the good BE in the frame of two solitary waves moving towards each other;
 \item by increasing the approximation order or decreasing the space step, the numerical methods produce  solutions with smaller error norms;
 \item the mass and the energy (in case of the Conservative Finite Difference Scheme) norms are preserved with the same accuracy as the numerical solution.
\end{itemize}

\end{frame}


%---------- frame 05 ----------------
\begin{frame}
\frametitle{Conservation of the mass and energy}
The mass:

\begin{equation}\label{int}
M(u(t))=M(u(0))=\int_{R} u(x)dx 
\end{equation}

and energy:
\begin{align}\label{ex-en}
E(u(t)) = E(u(0)) =&\frac{1}{2} \left\|(-\partial^2_x)^{-1/2} \frac{\partial u}{\partial t}(x,t)\right\|^2 + \frac{1}{2}  \left\|u (x,t)\right\|^2 
 \nonumber\\
+& \frac{1}{2}\left\| \nabla u(x,t) \right\|^2+ \int _{R} u(x,t)^3  dx = const
\end{align}
of the continuous problem, where $Au=-\Delta u$.
\end{frame}



%---------- frame ----------------

\begin{frame}
\frametitle{Conservative FDS}
The approximation of the differential operators is defined as:
\begin{equation}\label{secDer}
y_{\bar{t}t,i}^k=\dfrac{y_i^{k+1}-2y_i^k+y_i^{k-1}}{\tau^2},
\end{equation}

\begin{equation}\label{secDerfd}
\Delta_h y_i^k=\dfrac{y_{i+1}^k-2y_i^k+y_{i-1}^k}{h^2}
\end{equation}

\begin{equation}\label{forDerfd}
\Delta_h^2 y_i^k=\dfrac{y_{i+2}^k-4y_{i+1}^k+6y_{i}^k-4y_{i-1}^k+y_{i-2}^k}{h^4}.
\end{equation}

\end{frame}

%------------------------------------------------------------------

\begin{frame}
\frametitle{Conservative FDS}
Replacing the derivatives \rf{secDer}, \rf{secDerfd} and \rf{forDerfd} in the good BE leads to:
\begin{equation}\label{scheme1}
y_{\bar{t}t,i}^k=\Delta_h y_i^{\sigma,k}-\Delta_h^2 y_i^{\sigma,k} -3\Delta_h g(y_i^k).
\end{equation}
Here $ y_i^{\sigma,k}= y_i^{k}+\sigma \tau^2 y_{\bar{t}t,i}^k$, where the nonlinear term is approximated with:
\begin{equation}\label{nonLin}
g(y_i^k) = \frac{1}{3} \frac{(y_i^{k+1})^3 - (y_i^{k-1})^3}{y_i^{k+1} - y_i^{k-1}}.
\end{equation}


\end{frame}


%------------------------------------------------------------------

\begin{frame}
\frametitle{TS Approach with Method of Lines}

Let $u_{i, \widehat{xx}, p}$ be the approximation of $\frac{\partial^2 }{\partial x^2} u$ with $O(|h|^p)$, $p=2,4$.
\\
Let $u_{i}(t)$ be the approximation of $u(x_i, t)$ at grid point $(x_i)$.
\\
Then, inserting the above approximations into the good BE, one obtains a system of ODEs:
\be \label{DiscreteEq}
\frac{\partial^2 }{\partial t^2}u(x_i, t)  =
(u_{i} - u_{i, \widehat{xx}, p} - 3u^2_{i})_{\widehat{xx}, p}(t)
\ee
for $i = 0..N$. For each ODE in the system we do TS expansion:
\begin{align} \label{TSe}
u(x_i, t+\tau) = u(x_i, t) + \tau \frac{ \partial u }{ \partial t }(x_i, t)  + ... 
%\nonumber
%\\
\frac{ \tau^p }{ p! } \frac{ \partial^p u }{ \partial t^p }(x_i, t) + O(\tau^{p+1})
\end{align}
for some natural number $p \ge 2$.
\end{frame}

%------------------------------------------------------------------

\begin{frame}
\frametitle{TS Approach with Method of Lines}
Evaluating formula [\ref{TSe}]
\begin{itemize}
 \item IC $\rightarrow$ $u(x_i, t=0)$, $\frac{ \partial u }{ \partial t }(x_i,  t=0)$,
 \item $[$\ref{DiscreteEq}$]$ $\rightarrow$ $\frac{ \partial^2 u }{ \partial t^2 }(x_i, t=0)$ ,
 \item Differentiating equation $\frac{ \partial^{p-2} u }{ \partial t^{p-2} }$ [\ref{DiscreteEq}] $\rightarrow$  $\frac{ \partial^p u }{ \partial t^p }(x_i, t)$,
 \item Substitute the evaluated derivatives in formula $[$ \ref{TSe} $]$.
\end{itemize}


Thus one gets the following approximations
\begin{description}
 \item[$p=2$] $O(|h|^2 + \tau^2)$,
 \item[$p=4$] $O(|h|^4 + \tau^4)$.
\end{description}
%The complexity of the algorithm is
%$$ O( N_1 N_2 N_3 p ) $$
%where $N_1 N_2$ is the number of points in $\Omega_h$ and $N_3 = T/\tau$.
\end{frame}

%------------------------------------------------------------------
\begin{frame}
\frametitle{Exact Solution of the good BE}
\begin{align}\label{orgBsqSol}
u(x,t) =& -2 \frac{\partial ln F(x,t)}{\partial x^2},
\\
 F(x,t) =& a_0 + a_1 e^{k_1 x + \omega_1 t + b_1} + a_2 e^{k_2 x + \omega_2 t + b_2}  + a_{12} e^{(k_1 + k_2) x + (\omega_1 + \omega_2)  t + b_1 + b_2}, \nonumber
\\
|k_i| <& 1, \; \omega_i = \sqrt{k^2_i(1-k^2_i) }, \; k_i \neq 0, \quad i = 1,2, \nonumber
\\
a_0 =& \frac{a_1 a_2}{a_{12}}\frac{4k_1^4 + 4k_2^4 - 3k_1^2 - 3k_2^2 - 2k_1^2 k_2^2 + 6\sqrt{k_1^2-k_1^4}\sqrt{k_2^2-k_2^4} }{(k_1 + k_2)^2 (4(k_1^2 + k_1 k_2 + k_2^2) - 3)}.\nonumber
\end{align}
The parameters used for the numerical tests are defined as:
\be\label{params}
        k_1 = 1/3,  \; k_2 = -1/2,  \; b_i = -20\sqrt{(k_i^ 2  (1 - k_i ^ 2))}, i=1,2,  \;\; a_1 = a_2 = a_{12} = 1.
\ee

\end{frame}

%------------------------------------------------------------------

\begin{frame}
\frametitle{Exact Solution of the good BE}
\begin{figure}[ht]
	\centering
	\includegraphics[width=0.84\linewidth]{../IC.png}

Initial condition
\end{figure}
\end{frame}
%---------------------------------------

\begin{frame}
\frametitle{Validation and Results}
The space and time domains are defined as follows:
\begin{description}
 \item[-] $[L_1,L_2]$ with $-L_1 = L_2 = 60$;
 \item[-] $[T_0,T]$ with $T_0 = 0$ and $T = 10$ or $T = 35$.
\end{description}

Numerical methods applied for the tests:
\begin{enumerate}
  \item Conservative FDS with $O(|h|^2 + \tau^2)$
  \item TS method with $O(|h|^p + \tau^p)$ and $p = 2, 4$
\end{enumerate}

Numerical tests use zero boundary condition, i.e. values of the finite difference stencil outside the space domain $[L_1,L_2]$ are zeros.
\end{frame}

\begin{frame}
%C
\frametitle{Convergence for Conservative FDS}
$T = 10$
\begin{table}[ht]
\centering
\small
		\begin{tabular}{||c|l|ll|ll||}
			\hline
			\hline
      \multirow{2  }{*}{FDS}        & \multirow{2  }{*}{$h$, $\tau$}  & \multirow{2  }{*}{errors $E_i$in$L_2$}  &Conv.& \multirow{2  }{*}{errors $E_i$in$L_\infty$}  &Conv.  \\
	                                        &                                                     &                                                                 &  Rate &                                                                       & Rate \\
   			\hline 
					\hline 
  $\beta=3$                &0.2, 0.001         &                    &                &                  &                   \\
   c=0.45                     &0.1, 0.0005         & 0.989422   &                & 1.043649  &                   \\
     $O(h^2 + \tau^ 2)$ &0.05, 0.00025  &0.344818    & 1.52       & 0.355517   &   1.55   \\
	   \hline
			\hline 
       $\beta=1$           & 0.4, 0.002       &                   &           &                 &   \\
                  c=0.9       & 0.2, 0.001        & 0.200424   &          &0.072726  &   \\
  $O(h^2+ \tau^2)$  & 0.1, 0.0005       & 0.047899   & 2.06  &0.021451  & 1.76 \\
	   \hline
			\hline 
		\end{tabular}
		\caption{Convergence tests for the Conservative FDS method with zero boundary and approximation errors $O(h^{2} + \tau^2 )$. Errors $E_i$ are measured in $L_2$ and $L_\infty$ norms}
\label{tableC}
\end{table}

\end{frame}


%------------------------------------------------------------------

\begin{frame}
\frametitle{Convergence for TS Approach}
%A
$T = 10$
\begin{table}[ht]
\centering
\small
\resizebox{0.6\linewidth}{!}{%
		\begin{tabular}{||c|l|ll|ll||}
			\hline
			\hline
      \multirow{2  }{*}{FDS}        & \multirow{2  }{*}{$h$, $\tau$}  & \multirow{2  }{*}{errors $E_i$in$L_2$}  &Conv.& \multirow{2  }{*}{errors $E_i$in$L_\infty$}  &Conv.  \\
	         &                    &                               & Rate   &                                        & Rate \\
   			\hline 
					\hline 
  $\beta=3$                &0.2, 0.001          &              &              &                     &      \\
   c=0.45                     &0.1, 0.0005          &0.989414 &            &1.043641    &       \\
     $O(h^2 + \tau^ 2)$ &0.05, 0.00025   & 0.344813 & 1.52    &0.355511    &  1.55      \\
			\hline 
  $\beta=3$               &0.2, 0.02       &              &            &                     &      \\
   c=0.45                    &0.1, 0.01      &0.191224 &            &0.193874    &       \\
     $O(h^4+ \tau^4)$ &0.05, 0.005&0.013029 & 3.87   &0.013656     &3.82       \\
			\hline 
  $\beta=3$               &0.2, 0.02       &                &            &                     &      \\
     c=0.45                 &0.1, 0.01        &0.032671 &            &  0.033626    &       \\
     $O(h^6+ \tau^6)$ &0.05, 0.005 &0.000598 &5.77     & 0.000635    & 5.72       \\
	   \hline
			\hline 
       $\beta=1$       &0.4, 0.002        &             &            &           &   \\
                  c=0.9    &0.2, 0.001       &  0.20366   &            &0.075854 &   \\
  $O(h^2+ \tau^2)$ &0.1, 0.0005   &0.048320   &2.07  &0.022307  & 1.77 \\
			\hline
      $\beta=1$               &0.4, 0.04    &            &               &             &    \\
       c=0.9                     &0.2, 0.02     & 0.028275   &        &  0.013518   &   \\
       $O(h^4+ \tau^4)$ &0.1, 0.01   &0.001812 & 3.96  & 0.000971  & 3.80  \\
    \hline
  $\beta=1$     &0.4, 0.04   &            &          &                  &      \\
      c=0.9                    &0.2, 0.02   &0.006734 &           & 0.003338      &       \\
     $O(h^6+ \tau^6)$ &0.1, 0.01 & 0.000232 &4.86 & 0.000069  & 5.60        \\
	   \hline
			\hline 
		\end{tabular}
		}%
		\caption{Convergence tests for Taylor method with zero boundary and different approximation errors $O(h^{2} + \tau^2 )$, $O(h^{4} + \tau^4 )$ and $O(h^{6} + \tau^6 )$. Errors $E_i$ are measured in $L_2$ and $L_\infty$ norms}
\label{table:A}
\end{table}

\end{frame}

%------------------------------------------------------------------


\begin{frame}
\frametitle{Results}
\framesubtitle{Solution Shape, Taylor Series, smallest step sizes, $O(|h|^6+\tau^6)$}

%\begin{figure}[ht]
%	\centering
%	\begin{minipage}[b]{0.49\linewidth}
%		\includegraphics[width=\linewidth]{figures/compare_start_end_bt3_c045_T10.png}
%	\end{minipage}	
%	\begin{minipage}[b]{0.49\linewidth}
%		 \includegraphics[width=\linewidth]{figures/compare_start_end_bt1_c090_T10.png}
%	\end{minipage}

Difference between localized wave at time $T=0$ and $T=10$ for Test Wave 1 with $\beta =3$  (left) and Test Wave 2 with $\beta=1$ (right). 
%\end{figure}

It is obtained that:
\begin{description}
 \item[$\beta = 3$, $c = 0.45$] $||u_h(t=10)-u_h(t=0)|_{L_2} =  0.00000981$
 \item[$\beta = 1$, $c = 0.9$] $||u_h(t=10)-u_h(t=0)|_{L_2} = 0.00000186$
\end{description}
\end{frame}

%------------------------------------------------------------------

\begin{frame}
\frametitle{Conclusion}

\begin{description}
 \item[-] 2D BPE is solved using TS method with high approximation orders $O(|h|^k+\tau^k)$, $k=2,4,6$
 \item[-] the results are compared with the Conservative Scheme for $O(|h|^2+\tau^2)$
 \item[-] the Energy of the TS solution is saved with high accuracy over the time interval $[0, 10]$
 \item[-] the numerical solutions for wave speeds near the upper limit $c_{max} = min\{1, \sqrt{\beta_2/\beta_1} \}$ are stable in form and their maximums change with small errors over the time interval $[0, 10]$.
\item[-] the obtained solutions show soliton behavior!
\end{description}

\end{frame}

%------------------------------------------------------------------



\end{document}


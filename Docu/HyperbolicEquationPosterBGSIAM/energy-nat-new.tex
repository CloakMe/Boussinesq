\documentclass{article} 

%\usepackage{amsmath,amsthm,amsfonts}
\usepackage[english]{babel}
\usepackage[numbers]{natbib}
\usepackage{rotating}
\usepackage{amsfonts,amsmath}
\usepackage{graphicx}
\usepackage{color} 
\usepackage[notref,notcite]{showkeys}
\usepackage{multirow}

\newcommand{\be}{\begin{equation}}
\newcommand{\ee}{\end{equation}}
\newcommand{\rf}[1]{(\ref{#1})}
\newcommand{\RR}{\mathbb{R}}
\newtheorem{thm}{Theorem}
\newtheorem{lm}{Lemma}

\begin{document}
We study the problem

\be\label{problem}
\beta(E-\Delta) \frac{\partial^2 u}{\partial t^2}=
 \beta \Delta u -\Delta^2 u -\alpha \beta \Delta (u^2)
\ee


Define the operator $A$ as $Av=-\Delta_h v=-v_{\bar{x}x} - v_{\bar{y}y}$ and consider the FDS

\be\label{FDS1}
\beta (E+A)v_{\bar{t}t}^k +\beta Av^k+A^2 v^k -\alpha \beta A\left(\frac{(v^{k+1})^3-(v^{k-1})^3}{3(v^{k+1}-v^{k-1})} \right)=0
\ee

We multiply \rf{FDS1} by $A^{-1}$ and get 
\be\label{FDS2}
\beta (E+A^{-1})v_{\bar{t}t}^k +\beta v^k+A v^k -\alpha \beta \frac{(v^{k+1})^3-(v^{k-1})^3}{3(v^{k+1}-v^{k-1})} =0
\ee

We substitute $v^{k}=0.5(v^{k+1}+v^{k-1})-\frac{\tau^2}{2}v_{\bar{t}t}^k$ into \rf{FDS2}
and obtain
\begin{align*}
&\left( \beta (E+A^{-1})- \frac{\tau^2}{2}(\beta E+A ) \right)v_{\bar{t}t}^k  + \frac{1}{2} (\beta E +A )(v^{k+1}+v^{k-1}) \\
&~~~~~-\alpha \beta \frac{(v^{k+1})^3-(v^{k-1})^3}{3(v^{k+1}-v^{k-1})} =0
\end{align*}
We multiply the previous equation by $(v^{k+1}-v^{k-1})=\tau (v_{\bar{t}}^k + v_{t}^k)$ and sum over the spatial mesh points.
We reorganize the expressions and get the following formula 
\be\label{num_en}
E_h(v^k) =E_h(v^{k-1}),
\ee
where
\begin{align*}
E_h(v^k)=\left( \left( \beta (E+A^{-1})- \frac{\tau^2}{4}(\beta E+A ) \right)v_{t}^k ,v_{t}^k \right)+\frac{1}{4} \beta \left(  v^{k+1}+v^{k}, v^{k+1}+v^{k} \right) \\
+\frac{1}{4}  \left(  A(v^{k+1}+v^{k}), v^{k+1}+v^{k} \right)
-\alpha \beta \frac{((v^{k+1})^3,1)+((v^{k})^3,1)}{3}.
\end{align*}

In this way we prove the following theorem
\begin{thm}
The solution to the FDS \rf{FDS1} conserves the discrete energy
 $E_h(v^0)$, i.e.  $E_h(v^k) =E_h(v^{0})$ for each $k=1,2,...K$.
\end{thm}

\begin{thm}
The linear FDS corresponding to \rf{FDS1} is conditionally stable if
the following restriction is satisfied
$\tau^2 < \frac{\beta}{2}(1-\frac{\tau^2}{4}) h^2$.

\end{thm}
The proof is a consequence of the stability results from the book of
 Samarskii,  The theory of difference schemes, Marcel Dekker Inc., New York, 2001.

\newpage

the energy of the continuous problem \rf{problem}:

Define$Au=-\Delta u$
Then
\begin{align}\label{ex-en}
E(u):=&\int_{R^2} u_t \left((A^{-1}+E)u_t\right) dxdy+
\beta \int_{R^2} u^2 dxdy \nonumber\\
+& \int_{R^2}u \left(A u\right) dxdy
-\frac{2 \alpha \beta}{3} \int_{R^2} u^3 dxdy =const
\end{align}
Here $E(u)$ is the exact energy of problem \rf{problem}

===============================================


Let us replace the operator $Au=-
\Delta u$ with discrete operators $-\Delta_h u$ with different approximation errors - $O(h^2)$, $O(h^4)$, $O(h^6)$.

Suppose we know the discrete approximation of the derivative $u_t$
with approximation errors $O(\tau^2)$, $O(\tau^4)$, $O(\tau^6)$.

then one can apply quadrature formulas to evaluate numerically
the energy \rf{ex-en}!

When the numerical solution is found with $O(h^2+\tau^2)$ error,
we apply \rf{quadr2} with $O(h^2)$ error;


When the numerical solution is found with $O(h^4+\tau^4)$ error,
we apply 2D Simpson's rule \rf{quadr4} with $O(h^4)$ error;

When the numerical solution is found with $O(h^6+\tau^6)$ error,
we apply 2D Boole's rule \rf{quadr6-2D} with $O(h^6)$ error;

===========================================================

Quadrature formulas in 2D case for evaluation of 

\begin{equation}\label{int}
D(u)=\int_{a_1}^{b_1} \int_{a_2}^{b_2} u(x,y)dx dy
\end{equation}

$x_i, ~i=0,1,...,N_1$; $x_0=a_1,~x_{N_1}=b_1$, $h_1=(b_1-a_1)/N_1$


$y_j, ~j=0,1,...,N_2$; $y_0=a_2,~y_{N_2}=b_2$,  $h_2=(b_2-a_2)/N_2$

===========================================================================================

2D trapezoidal formula (with global $O(h^2)$ error):

\begin{align}\label{quadr2}
D_h(u_{i,j}) =& \sum_{i=1}^{N_1-1} \sum_{j=1}^{N_2-1} h_1 h_2 u_{i,j}
+\frac{h_1}{2}\sum_{i=0} \sum_{j=1}^{N_2-1} h_2 u_{i,j}
+\frac{h_1}{2}\sum_{i=N_1} \sum_{j=1}^{N_2-1} h_2 u_{i,j} \nonumber\\
+&\frac{h_2}{2}\sum_{j=0} \sum_{i=1}^{N_1-1} h_1 u_{i,j}
+\frac{h_2}{2}\sum_{j=N_2} \sum_{i=1}^{N_1-1} h_1 u_{i,j}
\nonumber\\
+&\frac{1}{4}h_1 h_2 \left(u_{0,0}+u_{N_1,0}+u_{N_1,N_2}+u_{0,N_2}
\right)
\end{align}
=================================================

2D Simpson's rule (with global $O(h_1^4+h_2^4)$ error), assuming that $N_1=2k$, $N_2=2 l$:

for every $m=0,1,2,\cdots N_2$ we compute 
$$D_m= \frac{h_1 }{3} 
\left\{ u_{0,m}+u_{N_1,m}+ 4 \sum_{i=1}^{\frac{N_1}{2}}   u_{2i-1,m}
 +2 \sum_{i=1}^{\frac{N_1}{2}-1} u_{2i,m} \right\}$$


Then 

\begin{equation}\label{quadr4}
D_h(u)=\frac{h_2 }{3} 
\left\{ D_{0}+D_{N_2}+ 4 \sum_{j=1}^{\frac{N_2}{2}}   D_{2j-1}
 +2 \sum_{j=1}^{{\frac{N_2}{2}}-1} D_{2j} \right\}
\end{equation}
is the approximation of the integral \eqref{int} with global $O(h_1^4+h_2^4)$ error

%
==============================================

Let $N_1=4k$, $x_i, ~i=0,1,...,N_1$, $h_1=(b_1-a_1)/N_1$ 

1D Boole's rule (with global $O(h_1^6)$ error):

\begin{align}\label{quadr6-1D}
D_h(u) =& \frac{2h_1}{45} 
\left\{
7u_0+7u_{N_1}+32 \sum_{k=1}^{{\frac{N_1}{2}}}u_{2k-1}
+12\sum_{k=1}^{{\frac{N_1}{4}}}u_{4k-2}
+14 \sum_{k=1}^{\frac{N_1}{4}-1}u_{4k}
\right\}
\end{align}

===========================================

2D Boole's rule (with global $O(h_1^6+h_2^6)$ error), where $N_1=4k$, $N_2=4 l$

for every $m=0,1,2,\cdots N_2$ we compute

\begin{align*}
D_m =& \frac{2h_1}{45} 
\left\{
7u_{0,m}+7u_{N_1,m}+32 \sum_{i=1}^{\frac{N_1}{2}}u_{2i-1,m}
+12\sum_{i=1}^{\frac{N_1}{4}}u_{4i-2,m}
+14 \sum_{i=1}^{\frac{N_1}{4}-1}u_{4i,m}
\right\}
\end{align*}


then \eqref{quadr6-2D} is the approximation to the integral \eqref{int} with global $O(h_1^6+h_2^6)$ error



\begin{align}\label{quadr6-2D}
&D_h(u) =
\frac{2h_2}{45} 
\left\{
7D_{0}+7D_{N_2}+32 \sum_{j=1}^{\frac{N_2}{2}}D_{2j-1}
+12\sum_{j=1}^{\frac{N_2}{4}}D_{4j-2}
+14 \sum_{j=1}^{\frac{N_2}{4}-1}D_{4j}
\right\}
\end{align}

%A
\begin{table}[ht]
\centering
\small
		\begin{tabular}{||c|l|ll|ll||}
			\hline
			\hline
      \multirow{2  }{*}{FDS}        & \multirow{2  }{*}{$h$, $\tau$}  & \multirow{2  }{*}{errors $E_i$in$L_2$}  &Conv.& \multirow{2  }{*}{errors $E_i$in$L_\infty$}  &Conv.  \\
	         &                    &                               & Rate   &                                        & Rate \\
   			\hline 
					\hline 
  $\beta=3$       &??, ??   &            &        &                  &      \\
      c=0.52   &??, ??   &?? &           &??      &       \\
     $O(h^2 + \tau^ 2)$ &??, ??5 &?? &??  &?? & ??       \\
			\hline 
   $\beta=3$        &0.2, 0.08   &            &        &                  &      \\
   c=0.52  &??, ??   &?? &           &??      &       \\
     $O(h^4+ \tau^4)$ &??, ??5 &?? &??  &?? & ??       \\
			\hline 
  $\beta=3$               &0.2, 0.08   &            &        &                  &      \\
   c=0.52                  &??, ??   &?? &           &??      &       \\
     $O(h^6+ \tau^6)$ &??, ??5 &?? &??  &?? & ??       \\
	   \hline
			\hline 
       $\beta=1$       &0.4, 0.04        &             &            &           &   \\
                  c=0.9    &0.2, 0.02       &0.326105   &            &0.152517&   \\
  $O(h^2+ \tau^2)$ &0.1, 0.01   &0.081954  &1.99 &0.041693  & 1.87 \\
			\hline
      $\beta=1$               &0.4, 0.08    &            &               &             &    \\
       c=0.9                     &0.2, 0.04     & 0.029515   &        &  0.014333   &   \\
       $O(h^4+ \tau^4)$ &0.1, 0.02   &0.002057 & 3.84   & 0.001041  & 3.78  \\
    \hline
  $\beta=1$     &0.4, 0.08   &            &          &                  &      \\
      c=0.9                    &0.2, 0.04   &0.007320 &           & 0.003625      &       \\
     $O(h^6+ \tau^6)$ &0.1, 0.02 & 0.000815 &3.16 & 0.000092  & 5.30        \\
	   \hline
			\hline 
		\end{tabular}
		\caption{Convergence test for SOLUTION with ZERO BOUNDARY and different approximation errors $O(h^{2} + \tau^2 )$, $O(h^{4} + \tau^4 )$ and $O(h^{6} + \tau^6 )$. Errors $E_i$ are measured in $L_2$ and $L_\infty$ norms}
\label{tableA}
\end{table}

%B
\begin{table}[ht]
\centering
\small
		\begin{tabular}{||c|l|ll|ll||}
			\hline
			\hline
      \multirow{2  }{*}{FDS}        & \multirow{2  }{*}{$h$, $\tau$}  & \multirow{2  }{*}{errors $E_i$in$L_2$}  &Conv.& \multirow{2  }{*}{errors $E_i$in$L_\infty$}  &Conv.  \\
	         &                    &                               & Rate   &                                        & Rate \\
   			\hline 
					\hline 
  $\beta=3$       &??, ??   &            &        &                  &      \\
      c=0.52   &??, ??   &?? &           &??      &       \\
     $O(h^2 + \tau^ 2)$ &??, ??5 &?? &??  &?? & ??       \\
			\hline 
   $\beta=3$        &0.2, 0.08   &            &        &                  &      \\
   c=0.52  &??, ??   &?? &           &??      &       \\
     $O(h^4+ \tau^4)$ &??, ??5 &?? &??  &?? & ??       \\
			\hline 
  $\beta=3$               &0.2, 0.08   &            &        &                  &      \\
   c=0.52                  &??, ??   &?? &           &??      &       \\
     $O(h^6+ \tau^6)$ &??, ??5 &?? &??  &?? & ??       \\
	   \hline
			\hline 
       $\beta=1$       &0.4, 0.04        &             &            &           &   \\
                  c=0.9    &0.2, 0.02       & 0.399714   &            &1.561666 &   \\
  $O(h^2+ \tau^2)$ &0.1, 0.01   &0.117543  &1.76  &0.473093  & 1.72 \\
			\hline
      $\beta=1$             &0.4, 0.08     &            &            &             &    \\
       c=0.9                   &0.2, 0.04      & 0.048286   &       &  0.060691  &   \\
       $O(h^4+ \tau^4)$ &0.1, 0.02   & 0.00459& 3.39    &0.007596  & 2.99  \\
    \hline
  $\beta=1$     &0.4, 0.08   &                      &           &                   &      \\
      c=0.9    &0.2, 0.04        & 0.025302     &           & 0.04921     &       \\
     $O(h^6+ \tau^6)$         &  0.1, 0.02    & 0.00308 & 3.03  &0.005102 & 3.26        \\
	   \hline
			\hline 
		\end{tabular}
		\caption{Convergence test for ENERGY with ZERO BOUNDARY and different approximation errors $O(h^{2} + \tau^2 )$, $O(h^{4} + \tau^4 )$ and $O(h^{6} + \tau^6 )$. Errors $E_i$ are measured in $L_2$ and $L_\infty$ norms}
\label{tableB}
\end{table}


\end{document}
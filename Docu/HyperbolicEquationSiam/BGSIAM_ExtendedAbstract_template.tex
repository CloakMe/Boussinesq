\documentclass{article}
\usepackage{times}
\usepackage{amsthm,amsfonts,amsmath,amscd,amssymb}


\textwidth 130mm
\textheight 188mm
\footskip 8mm
\parindent 0in
\newcommand{\RR}{\mathbb{R}}
\newcommand{\writetitle}[2]{
\addcontentsline{toc}{part}{\normalsize{{\it #2}\\#1}\vspace{-17pt}}\vskip 2em
\begin{center}{\Large {\bf #1} \par}\vskip 1em{\large\lineskip .5em{\bf #2}\par}
\end{center}\vskip .5em}


\begin{document}


\writetitle{Numerical Study of Soliton Solutions to the Two Dimensional Boussinesq  Equation}
{K. Angelow, N. Kolkovska}


In this paper we evaluate propagating wave solutions to the two dimensional Boussinesq Equation (BE)
\begin{align} \label{eq1}
&u_{tt} - \Delta u -\beta_1  \Delta u_{tt} +\beta_2 \Delta ^2 u + \Delta f(u)=0   \quad \text{for}  (x,y) \in \RR^2, \, t\in\RR^+, 
\\ \nonumber &u(x,y,0)=u_0(x,y), \, u_t(x,y,0)=u_1(x,y)   \quad\text{for} \, (x,y) \in \RR^2,
\\  &u(x,y) \rightarrow 0,  \Delta u(x,y) \rightarrow 0 ,  \quad \text{for}  \sqrt{x^2 + y^2} \rightarrow \infty, \label{eq11}
\end{align}
where $f(u)=\alpha u^2$,  $\alpha>0$, $\beta_1>0$, $\beta_2>0$  are dispersion parameters, and $\Delta$ is the two-dimensional Laplace operator. We suppose that the solution of \eqref{eq1} is sufficiently smooth function.

The BE is famous with the approximation of shallow water waves or also weakly non--linear long waves. It is often used for simulation of various physical processes e.g. turbulence in fluid mechanics, vibrations in acoustics, geotechnical engineering, atmosphere dynamics, etc. 

Boussinesq equation \eqref{eq1} admits localized stationary propagating solutions of type $u(x,y,t)=U(x,y-ct)$. They are obtained by different approximation techniques in \cite{chr-chr-07,Ch2012,Ch2011}. The time behavior and structural stability of the solutions to the hyperbolic problem with initial data from \cite{chr-chr-07,Ch2012,Ch2011} are presented in 
\cite{cher,dani,mil-dani} using numerical methods with second order of approximation in space and time.

In \cite{EllipticProblem} the stationary propagating solutions to \eqref{eq1} are evaluated with high accuracy of fourth and sixth order of approximation. The resulting solutions differ from the solution developed in \cite{chr-chr-07,Ch2012,Ch2011} especially for velocities $c$ close to $\min(1,\frac{\beta_2}{\beta_1})$.  

The aim of this paper is to present and apply a high accuracy numerical method for solving \eqref{eq1} with initial data obtained in \cite{EllipticProblem}.

To construct the numerical method, we first introduce a uniform mesh $\Omega_h$ on a computational domain $\Omega$ and apply finite differences of second, fourth and sixth order for approximation $\Delta_h$ to the Laplace operator. In this way we obtain the following system of second order ODE's with respect to unknowns
%$u_{i,j}(t)$, 
 $u_{i,j}(t)= u(x_i,y_j,t)$ 
\begin{align}\label{eq2}
%(E-\beta_1 \Delta_h)
(u_{i,j})_{tt} = (E-\beta_1 \Delta_h)^{-1} \left(\Delta_h u_{i,j} - \beta_2 \Delta_h^2 u_{i,j} -\Delta_h f(u_{i,j})\right),
\end{align}
where $E$ is the identity operator.

Then the values of the solutions on the next time $u_{i,j}(t+\tau)$ are evaluated using Taylor series expansion of $u_{i,j}$ at time $t$ with degrees corresponding to the order of space approximation. All time derivatives in the Taylor series expansion are evaluated iteratively by differentiating \eqref{eq2} with respect to the time variable. This is possible since the highest time derivative in \eqref{eq2} is isolated on one side of equation \eqref{eq2}. %one could obtain higher time derivative terms in the TS expansion.

An explicit formula for the numerical solution is applied on the computational boundary $\partial \Omega$. The formula is taken from \cite{BoundaryProblem} and adjusted for the hyperbolic equation (\ref{eq1})
\begin{equation}\label{eqBCH}
u_B(x, y, t) = \mu_u \frac{ (1 - c^2) x^2 - (y-ct)^2}{( (1 - c^2) x^2 + (y-ct)^2)^2}.
\end{equation}
%The values \eqref{eqBCH} are used also at mesh points included in the cross stencils which lie outside the computational domain.

Near the computational boundariy $\partial \Omega$ we do not change any of the cross stencils used in the approximation of the discrete Laplacian - 
%. There, the discrete Laplacian is defined by using the values of the discrete solution (\ref{eqBCH})
we use the values \eqref{eqBCH} at points outside the computational domain. In the implementation of the numerical method Fast Poisson Solver (FPS) technique  is used at each time  step. The band matrices produced at each time level are inverted by Thomas or similar technique.

In this way the numerical solution to \eqref{eq1} is computed by Taylor series method with high accuracy of second, fourth and sixth order both in space and time on relatively coarse grid.

The performed numerical tests validate the numerical method for solving BE. The results show that the  Taylor series method achieves the prescribed high accuracy. 

We solve BE with two sets of initial parameters and initial data obtained in \cite{EllipticProblem}. 
%Furthermore, the propagating wave preserves its maximum and more importantly, its shape.
The resulting two numerical solutions are stable in form and their maxima are changed for
long period of time $T = 40$ with small errors. Thus the obtained solutions show soliton behavior.

\begin{thebibliography}{99}

\bibitem{BoundaryProblem}
K. Angelow, 
\emph{New Boundary Condition for the Two Dimensional Stationary Boussinesq Paradigm Equation},
International Journal of Applied Mathematics, \textbf{32, 1} (2019), 141--154.

%%5
\bibitem{EllipticProblem}
K. Angelow and N. Kolkovska, 
\emph{Numerical Study of Traveling Wave Solutions to 2D Boussinesq Equation},
Serdica J. Computing,  Accepted for publication.
%
%8
\bibitem{cher}
A. Chertock, C. Christov and A. Kurganov, 
\emph{Central--upwind schemes for the  Boussinesq paradigm equation},
Comp. Sci. High Performance Comp. IV, NNFM, \textbf{113} (2011), 267--281.
%
%%3
%\bibitem{chr-chr}
%M. Christou and C. Christov, 
%\emph{Galerkin Spectral Method for the 2D Solitary Waves of Boussinesq Paradigm Equation}
%AIP Conference Proceedings \textbf{1186, 217} (2009) 217--225.
%
%%4
\bibitem{chr-chr-07}
M. Christou and C. Christov, 
\emph{Fourier–Galerkin method for 2D solitons of Boussinesq equation}, 
Mathematics and Computers in Simulation, \textbf{74} (2007), 82--92.
%
%%1
%\bibitem{ChChr}
%C. Christov, 
%\emph{An energy-consistent dispersive shallow-water model}, 
%{\it Wave Motion}, \textbf{34} (2001), 161--174.
%
%%6
\bibitem{Ch2012}
C. Christov, 
\emph{Numerical implementation of the asymptotic boundary conditions
for steadily propagating 2D solitons of Boussinesq type equation},       
Math. Computers  Simul., \textbf{82} (2012),  1079--1092.
%
%%2
\bibitem{Ch2011}
C. Christov and J. Choudhury, 
\emph{Perturbation solution  for the 2D Boussinesq equation},       
Mech. Res. Commun., \textbf{38} (2011),  274--281.
%
%%9
\bibitem{dani}
C. Christov, N. Kolkovska and D. Vasileva, 
\emph{On the numerical simulation of unsteady solutions for the 2D Boussinesq paradigm equation},
LNCS, \textbf{6046} (2011), 386--394.
%
\bibitem{mil-dani}
M. Dimova, D. Vasileva, Comparison of two numerical approaches to
Boussinesq paradigm equation, LNCS, \textbf{8236} 
(2013), 255--262.

%%12
%\bibitem{forn}
%B. Fornberg, 
%\emph{Generation of Finite Difference Formulas on Arbitrarily Spaced Grids}, 
%Math. Comput., \textbf{51} (1988),  699--706.
%
%%11
%\bibitem{FPS}
%T. Lyche,
%\emph{Fast Poisson Solvers and FFT}, 
%Lecture Notes, University of Oslo, Norway
%
%%10
%\bibitem{Tref}
%L. Trefethen and D. Bau,
%\emph{Numerical linear algebra},
%1$^{st}$ ed., SIAM, Philadelphia, 1997.

%\bibitem{critEn}
%N. Kolkovska and K. Angelow,
%\emph{Numerical computation of the critical energy constant for two-dimensional Boussinesq equations}
%AIP Conference Proceedings 1684, 080007 (2015)

%%\bibitem{sam}
%%A.~Samarskii, The theory of difference schemes, M. Dekker,  2001.

\end{thebibliography}

\end{document}

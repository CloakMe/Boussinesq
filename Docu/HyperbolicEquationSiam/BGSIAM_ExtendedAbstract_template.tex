\documentclass{article}
\usepackage{times}
\usepackage{amsthm,amsfonts,amsmath,amscd,amssymb}


\textwidth 130mm
\textheight 188mm
\footskip 8mm
\parindent 0in
\newcommand{\RR}{\mathbb{R}}
\newcommand{\writetitle}[2]{
\addcontentsline{toc}{part}{\normalsize{{\it #2}\\#1}\vspace{-17pt}}\vskip 2em
\begin{center}{\Large {\bf #1} \par}\vskip 1em{\large\lineskip .5em{\bf #2}\par}
\end{center}\vskip .5em}


\begin{document}


\writetitle{Numerical Study of Soliton Solutions to the Two Dimensional Boussinesq  Equation}
{K. Angelow, N. Kolkovska}

The length of the extended abstract should be between 1 and 2 pages.


This paper evaluates propagating wave solutions to the two dimensional Boussinesq Equation (BE)
\begin{align} \label{eq1}
&u_{tt} - \Delta u -\beta_1  \Delta u_{tt} +\beta_2 \Delta ^2 u + \Delta f(u)=0   \quad \text{for}  (x,y) \in \RR^2, \, t\in\RR^+, 
\\ \nonumber &u(x,y,0)=u_0(x,y), \, u_t(x,y,0)=u_1(x,y)   \quad\text{for} \, (x,y) \in \RR^2,
\\  &u(x,y) \rightarrow 0,  \Delta u(x,y) \rightarrow 0 ,  \quad \text{for}  \sqrt{x^2 + y^2} \rightarrow \infty, \label{eq11}
\end{align}
where $f(u)=\alpha u^2$,  $\alpha>0$, $\beta_1>0$, $\beta_2>0$  are dispersion parameters, and $\Delta$ is the Laplace operator. Taylor Series (TS) expansion method about time is used to move the solution forward. The equation allows to isolate the highest time derivative on one side. Thus, by differentiating with respect to time variable one could obtain higher time derivative terms in the TS expansion. Here, the solver could be adjusted to use second, fourth or sixth degree of the Taylor polynomial.

High order finite difference schemes for the spatial derivatives are implemented to solve this nonlinear hyperbolic problem. The two dimensional laplace operator $\Delta$ is approximated by three different cross stencils with five, nine and thirteen points each:
\[
S_5 = 
\begin{bmatrix}
            &  1  &  \\
    1     &  -4    & 1\\
            &  1  &  
\end{bmatrix},
S_9 =
\begin{bmatrix}
          &       &  -1/12  & &  \\
          &       &  4/3     &   & \\
-1/12 & 4/3 &  -5       & 4/3 & -1/12\\
          &       &   4/3    &    &\\
          &       &  -1/12  &    &
\end{bmatrix},
\]
\[
S_{13} = 
\begin{bmatrix}
          &       &       &   1/90  & & &  \\
          &       &       &  -3/20  & & &  \\
          &       &       &   3/2     & &   & \\
1/90 & -3/20 &  3/2 & -49/18 & 3/2 & -3/20 & 1/90 \\
          &       &       &   3/2     & &   & \\
          &       &       &  -3/20  & & &  \\
          &       &       &   1/90  & & &
\end{bmatrix}
\]
Thus, the numerical solution is computed on relatively coarse grid with high accuracy of second, fourth and sixth order both in space and time. This means that the BE equation solver could be adjusted to use among second, fourth or sixth approximation order.

An explicit formula for Boundary Condition (BC) is applied on the computational boundary. The formula is taken from \cite{BoundaryProblem} and adjusted for the hyperbolic equation (\ref{eq1})
\begin{equation}\label{eqBCH}
u_B(x, y, t) = \mu_u \frac{ (1 - c^2) x^2 - (y-ct)^2}{( (1 - c^2) x^2 + (y-ct)^2)^2}.
\end{equation}
Near the computational boundaries $\partial \Omega$ we do not change any of the cross stencils defined above. There, the discrete Laplacian is defined by using the values of the discrete solution (\ref{eqBCH}) at points outside the computational domain.

Fast Poisson Solver technique is used, in order to isolate the highest time derivative on one side of the equation. This is done at each time iteration $t_k = k\tau$, where $\tau$ is the time step. After the two dimensional laplace operator is discretized numerically by any of the cross stencils $S_5, S_9, S_{13}$,  it can be inverted using Fast Poisson Solvers (\cite{FPS} and \cite{Tref}). The FPS produces series of band matrices which need to be inverted. Thomas algorithm is used for three diagonal matrices with the $S_5$ stencil.  Five and seven diagonal matrices (using $S_9$ and $S_{13}$ stencils, respectively) use similar to Thomas technique which is again a simplified form of Gauss elimination.

The performed numerical tests exhibit good convergence and confirm the validity of the TS method. Furthermore, the propagating wave preserves its maximum and more importantly, its shape.



\begin{thebibliography}{99}
%\bibitem{A11}
%FirstNameInitial.  MiddleNameInitial. LastName, % first name middle initial. and then last name.  Only the first character in the paper title is capitalized.
%\emph{Title of the paper,}
%Name of the Journal \textbf{Volume} (Year), StaringPage--EndingPage.

%7
\bibitem{BoundaryProblem}
K. Angelow, 
\emph{New Boundary Condition for the Two Dimensional Stationary Boussinesq Paradigm Equation},
International Journal of Applied Mathematics, \textbf{32, 1} (2019), 141--154.

%5
\bibitem{EllipticProblem}
K. Angelow and N. Kolkovska, 
\emph{Numerical Study of Traveling Wave Solutions to 2D Boussinesq Equation},
Serdica J. Computing,  Accepted for publication.

%8
\bibitem{cher}
A. Chertock, C. Christov and A. Kurganov, 
\emph{Central--upwind schemes for the  Boussinesq paradigm equation},
Comp. Sci. High Performance Comp. IV, NNFM, \textbf{113} (2011), 267--281.

%3
\bibitem{chr-chr}
M. Christou and C. Christov, 
\emph{Galerkin Spectral Method for the 2D Solitary Waves of Boussinesq Paradigm Equation}
AIP Conference Proceedings \textbf{1186, 217} (2009) 217--225.

%4
\bibitem{chr-chr-07}
M. Christou and C. Christov, 
\emph{Fourier–Galerkin method for 2D solitons of Boussinesq equation}, 
Mathematics and Computers in Simulation, \textbf{74} (2007), 82--92.

%1
\bibitem{ChChr}
C. Christov, 
\emph{An energy-consistent dispersive shallow-water model}, 
{\it Wave Motion}, \textbf{34} (2001), 161--174.

%6
\bibitem{Ch2012}
C. Christov, 
\emph{Numerical implementation of the asymptotic boundary conditions
for steadily propagating 2D solitons of Boussinesq type equation},       
Math. Computers  Simul., \textbf{82} (2012),  1079--1092.

%2
\bibitem{Ch2011}
C. Christov and J. Choudhury, 
\emph{Perturbation solution  for the 2D Boussinesq equation},       
Mech. Res. Commun., \textbf{38} (2011),  274--281.

%9
\bibitem{dani}
C. Christov, N. Kolkovska and D. Vasileva, 
\emph{On the numerical simulation of unsteady solutions for the 2D Boussinesq paradigm equation},
LNCS, \textbf{6046} (2011), 386--394.

%12
\bibitem{forn}
B. Fornberg, 
\emph{Generation of Finite Difference Formulas on Arbitrarily Spaced Grids}, 
Math. Comput., \textbf{51} (1988),  699--706.

%11
\bibitem{FPS}
T. Lyche,
\emph{Fast Poisson Solvers and FFT}, 
Lecture Notes, University of Oslo, Norway

%10
\bibitem{Tref}
L. Trefethen and D. Bau,
\emph{Numerical linear algebra},
1$^{st}$ ed., SIAM, Philadelphia, 1997.

%\bibitem{critEn}
%N. Kolkovska and K. Angelow,
%\emph{Numerical computation of the critical energy constant for two-dimensional Boussinesq equations}
%AIP Conference Proceedings 1684, 080007 (2015)

%\bibitem{sam}
%A.~Samarskii, The theory of difference schemes, M. Dekker,  2001.

\end{thebibliography}

\end{document}

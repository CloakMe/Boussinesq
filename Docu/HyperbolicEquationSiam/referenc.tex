%%%%%%%%%%%%%%%%%%%%%%%% referenc.tex %%%%%%%%%%%%%%%%%%%%%%%%%%%%%%
% sample references
% %
% Use this file as a template for your own input.
%
%%%%%%%%%%%%%%%%%%%%%%%% Springer-Verlag %%%%%%%%%%%%%%%%%%%%%%%%%%
%
% BibTeX users please use
% \bibliographystyle{}
% \bibliography{}
%
\biblstarthook{References may be \textit{cited} in the text either by number (preferred) or by author/year.\footnote{Make sure that all references from the list are cited in the text. Those not cited should be moved to a separate \textit{Further Reading} section or chapter.} The reference list should ideally be \textit{sorted} in alphabetical order -- even if reference numbers are used for the their citation in the text. If there are several works by the same author, the following order should be used: 
\begin{enumerate}
\item all works by the author alone, ordered chronologically by year of publication
\item all works by the author with a coauthor, ordered alphabetically by coauthor
\item all works by the author with several coauthors, ordered chronologically by year of publication.
\end{enumerate}
The \textit{styling} of references\footnote{Always use the standard abbreviation of a journal's name according to the ISSN \textit{List of Title Word Abbreviations}, see \url{http://www.issn.org/en/node/344}} depends on the subject of your book:
\begin{itemize}
\item The \textit{two} recommended styles for references in books on \textit{mathematical, physical, statistical and computer sciences} are depicted in ~\cite{science-contrib, science-online, science-mono, science-journal, science-DOI} and ~\cite{phys-online, phys-mono, phys-journal, phys-DOI, phys-contrib}.
\item Examples of the most commonly used reference style in books on \textit{Psychology, Social Sciences} are~\cite{psysoc-mono, psysoc-online,psysoc-journal, psysoc-contrib, psysoc-DOI}.
\item Examples for references in books on \textit{Humanities, Linguistics, Philosophy} are~\cite{humlinphil-journal, humlinphil-contrib, humlinphil-mono, humlinphil-online, humlinphil-DOI}.
\item Examples of the basic Springer style used in publications on a wide range of subjects such as \textit{Computer Science, Economics, Engineering, Geosciences, Life Sciences, Medicine, Biomedicine} are ~\cite{basic-contrib, basic-online, basic-journal, basic-DOI, basic-mono}. 
\end{itemize}
}

\begin{thebibliography}{99.}%

\bibitem{BoundaryProblem}
K. Angelow, 
New Boundary Condition for the Two Dimensional Stationary Boussinesq Paradigm Equation,
International Journal of Applied Mathematics, \textbf{32, 1} (2019), 141--154.

%5
\bibitem{EllipticProblem}
K. Angelow and N. Kolkovska, 
Numerical Study of Traveling Wave Solutions to 2D Boussinesq Equation,
Serdica J. Computing,  Accepted for publication.

%3
\bibitem{chr-chr}
M. Christou and C. Christov, 
Galerkin Spectral Method for the 2D Solitary Waves of Boussinesq Paradigm Equation
AIP Conference Proceedings \textbf{1186, 217} (2009) 217--225.

%8
\bibitem{cher}
A. Chertock, C. Christov and A. Kurganov, 
Central--upwind schemes for the  Boussinesq paradigm equation,
Comp. Sci. High Performance Comp. IV, NNFM, \textbf{113} (2011), 267--281.

%4
\bibitem{chr-chr-07}
M. Christou and C. Christov, 
Fourier–Galerkin method for 2D solitons of Boussinesq equation, 
Mathematics and Computers in Simulation, \textbf{74} (2007), 82--92.

%1
\bibitem{ChChr}
C. Christov, 
An energy-consistent dispersive shallow-water model, 
{\it Wave Motion}, \textbf{34} (2001), 161--174.

%6
\bibitem{Ch2012}
C. Christov, 
Numerical implementation of the asymptotic boundary conditions
for steadily propagating 2D solitons of Boussinesq type equation,       
Math. Computers  Simul., \textbf{82} (2012),  1079--1092.

%2
\bibitem{Ch2011}
C. Christov and J. Choudhury, 
Perturbation solution  for the 2D Boussinesq equation,       
Mech. Res. Commun., \textbf{38} (2011),  274--281.

%9
\bibitem{dani}
C. Christov, N. Kolkovska and D. Vasileva, 
On the numerical simulation of unsteady solutions for the 2D Boussinesq paradigm equation,
LNCS, \textbf{6046} (2011), 386--394.

%12
\bibitem{forn}
B. Fornberg, 
Generation of Finite Difference Formulas on Arbitrarily Spaced Grids, 
Math. Comput., \textbf{51} (1988),  699--706.

%11
\bibitem{FPS}
T. Lyche,
Fast Poisson Solvers and FFT, 
Lecture Notes, University of Oslo, Norway,
\url{https://www.uio.no/studier/emner/matnat/ifi/nedlagte-emner/INF-MAT4350/h08/undervisningsmateriale/chap10slides.pdf}

%10
\bibitem{Tref}
L. Trefethen and D. Bau,
Numerical linear algebra,
1$^{st}$ ed., SIAM, Philadelphia, 1997.

\end{thebibliography}

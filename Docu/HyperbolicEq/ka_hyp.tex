\documentclass[leqno,11pt]{book}
\usepackage{amsfonts,amssymb,amsmath,longtable,array,bbm,url}
\usepackage[utf8]{inputenc}
\usepackage[T2A]{fontenc}
\usepackage[english]{babel}
\usepackage{amssymb}
\usepackage{graphicx}
\usepackage{epstopdf}
\usepackage{float}
\usepackage{wrapfig}

\newenvironment{proof}{{P\,r\,o\,o\,f. }}{~~~{$\Box$}}
%\usepackage{serd_inf}
%\usepackage{fancyhea}
\usepackage{enumerate}
\usepackage[dvips]{epsfig}

\newcommand{\rf}[1]{(\ref{#1})}
\newcommand{\RR}{\mathbb{R}}

\renewcommand{\thesubsection}{}
\makeatletter
\def\@seccntformat#1{\@ifundefined{#1@cntformat}%
   {\csname the#1\endcsname\quad}%    default
   {\csname #1@cntformat\endcsname}}% enable individual control
\newcommand\section@cntformat{}     % section level 
\makeatother

\newtheorem{definition}{Definition}
\newtheorem{construction}{Construction}
\newtheorem{example}{Example}
\newtheorem{proposition}{Proposition}
\newtheorem{lemma}{Lemma}
\newtheorem{theorem}{Theorem}

\def\p{\phantom{0}}
\everymath{\displaystyle}
\allowdisplaybreaks

\def\theequation{\arabic{equation}}

\def\hlst{\setlength{\topsep}{4pt}\setlength{\partopsep}{4pt}%
\setlength{\parsep}{4pt}\setlength{\itemsep}{\parskip}}
%\begin{list}{{$\bullet$}}{\hlst}

%\setlength{\textfloatsep}{3pt}
%\setlength{\floatsep}{3pt}
%\setlength{\intextsep}{3pt}
\newcommand{\eeth}{\rm}
\newcommand{\dO}{\partial\Omega_{h}}
%%%%%
\textwidth 13.5cm
\textheight 18.5cm
\topmargin 0in
\parindent 0.5in
%\headsep 0in
\oddsidemargin 0.5in
\evensidemargin 0.5in

\makeatletter
\def\ps@serdica{%
    \let\@oddfoot\@empty\let\@evenfoot\@empty
    \def\@evenhead{\thepage\hfil{\sl \leftmark} \hfil}%
    \def\@oddhead{\hfil{\sl \rightmark}\hfil\thepage}%

    }
\makeatother

\pagestyle{serdica}

\newcommand\kwams[2]
{
\def\thefootnote{}
\footnote{{\it$\rm{}$}
#1

{}~~{\it Key words:} #2
}
}

\def\abstractname{} %{\sc Abstract.}} % <---------
\def\abstract{\if@twocolumn
\section*{\abstractname}
\else \small
%\begin{center}
%{\abstractname} %\vspace{-.5em}\vspace{0pt}}
%\end{center}
%\noindent
\quotation  \noindent
\fi}
\def\endabstract{\if@twocolumn\else\endquotation\fi}

\def\bibname{\rm R\,E\,F\,E\,R\,E\,N\,C\,E\,S} % <----------
\def\thebibliography#1{\addvspace{3em}\noindent
\hfil \bibname \hfill \list
 {[\arabic{enumi}]}{\settowidth\labelwidth{[#1]}\leftmargin\labelwidth
 \advance\leftmargin\labelsep
 \usecounter{enumi}}
 \def\newblock{\hskip .11em plus .33em minus .07em}
 \sloppy\clubpenalty4000\widowpenalty4000
 \sfcode`\.=1000\relax}
\let\endthebibliography=\endlist

\newcommand{\head}[8]
{ % 1-page, 2-title, 3-author, 4-short title,
  % 5-author, 6-vol, 7-year, 8-number
\thispagestyle{empty}

\noindent {\footnotesize{\rm Serdica} J. Computing {\bf #6} (#7),
No #8, #1} \hfill
%\begin{minipage}{42mm}
\setlength{\unitlength}{1mm}
\begin{picture}(46,14.13)
%%%\put(0,2.5){\special{em:graph serd-inf.bmp}} %3
%\put(54.6,1.9){\special{em:graph serd1o.pcx}}
\end{picture}
%\end{minipage}

\vspace*{2cm}

\markboth{#5}{#4} \vspace*{2cm}
\begin{center}
{\large\bf #2}  %%%\uppercase{#2}}

\bigskip

{\large #3}

\end{center}

\bigskip

\bigskip
}

\newcommand{\sect}[1]{\bigskip \par {\large\bf #1}}

%%%%%


\begin{document}
\setcounter{page}{3}

\def\thefootnote{}
\footnote{{\it ACM Computing Classification System $\rm(1998){ G.1.8, G.1.10}$}
%#1

{}~~{\it Key words:} two dimensional Boussinesq equation, traveling wave solutions (TWS), high order finite difference schemes, asymptotic boundary conditions
}

\head{3--4} {
Numerical Study of Soliton Solutions to the Two Dimensional Boussinesq  Equation
}
{K. Angelow, N. Kolkovska}
{
}
{K. Angelow, N. Kolkovska} {8}{2018}{2}

{\centering
\small \sl

}

\bigskip

\begin{abstract}\noindent{\sc Abstract.}
The aim of this paper is to evaluate propagating  wave solutions to the two dimensional Boussinesq Equation (BE). To solve this nonlinear fourth order hyperbolic  problem we use high order finite difference schemes for the spatial derivatives. Taylor Series (TS) expansion is used to describe the time derivatives. A Boundary Condition (BC) is applied on the computational boundary. Results from the TS approach are compared with results from another known method in the literature. The performed numerical tests exhibit good convergence and confirm the validity of the TS method.

\footnotetext{*This work is supported by the Institute of Mathematics and Informatics, Bulgarian
Academy of Sciences. The second author is partially supported by the Bulgarian Science Fund under  Grant~H~22/2 from 2018.}
\end{abstract}
\bigskip
\setcounter{footnote}{0}
\def\thefootnote{\arabic{footnote}}

\sect{1.~Introduction.}\label{introduction}

In this paper we  consider the two dimensional Boussinesq  Equation (BE)
\begin{align}
&u_{tt} - \Delta u -\beta_1  \Delta u_{tt} +\beta_2 \Delta ^2 u + \Delta f(u)=0   \quad \text{for}  (x,y) \in \RR^2, \, t\in\RR^+,\label{eq1}
\\ \nonumber &u(x,y,0)=u_0(x,y), \, u_t(x,y,0)=u_1(x,y)   \quad\text{for} \, (x,y) \in \RR^2,
\\  &u(x,y) \rightarrow 0,  \Delta u(x,y) \rightarrow 0 ,  \quad \text{for}  \sqrt{x^2 + y^2} \rightarrow \infty, \label{eq11}
\end{align}
where   $f(u)=\alpha u^2$,  $\alpha>0$, $\beta_1>0$, $\beta_2>0$  are dispersion parameters, and $\Delta$ is the Laplace operator. The BE is famous with the approximation of shallow water waves or also weakly non--linear long waves. It is often used for simulation of various physical processes e.g. turbulence in fluid mechanics, vibrations in acoustics etc. A derivation of the BE from the original Boussinesq system can be found in \cite{ChChr}.

The goal of the article is to seek for soliton solutions to (\ref{eq1}), which 
are traveling  in $y$ direction with velocity $c$. The most suitable Initial Condition (IC) for the numerical solver of the hyperbolic BE \rf{eq1} -- \rf{eq11} that was developed here, is found in \cite{EllipticProblem}. TS expansion is used to calculate the next time layer with respect to the time step $\tau$. The time derivatives
are obtained from the main equation by an iterative procedure. This is possible because the equation allows to isolate the highest time derivative on one side of the equation. By differentiating with respect to time variable one could obtain higher time derivatives terms in the TS expansion. The approximations used in the TS expansion could be adjusted to second, fourth or sixth order. To increase the precision of the numerical method, finite difference schemes (FDS) with local approximation of forth $O(h^4)$ and sixth $O(h^6)$ order are applied to the spatial derivatives i.e. the second derivative along space are approximated with high order finite differences. These FDS allow one to evaluate the numerical solution with high accuracy on relatively coarse grid.
In order to compare the results from TS method, a conservative finite difference scheme with weights is used, which applies second order of approximation along
time and space.
Both methods are tested with zero boundary and also with BC found and developed in \cite{BoundaryProblem}.


\frenchspacing
\def\bibname{\rm R\,E\,F\,E\,R\,E\,N\,C\,E\,S}
\begin{thebibliography}{30}

\bibitem{ChChr} C.I. Christov, An energy-consistent dispersive shallow-water model,  {\it Wave Motion}, \textbf{34} (2001), 161-174.


\bibitem{EllipticProblem}
K. Angelow, N. Kolkovska, Numerical Study of Traveling Wave Solutions to 2D Boussinesq Equation.
Serdica J. Computing,  8, (2018) 3-4.

\bibitem{BoundaryProblem}
K. Angelow, New Boundary Condition for the Two Dimensional Stationary Boussinesq Paradigm Equation.
International Journal of Applied Mathematics, (2018).


\bibitem{chr-chr-07}
M.~Christou, C.~I.~Christov, Fourier–Galerkin method for 2D solitons of Boussinesq equation, 
Mathematics and Computers in Simulation, 74, (2007) 82 -- 92.

\bibitem{chr-chr}
M.~Christou, C.~I.~Christov, Galerkin Spectral Method for the 2D Solitary
Waves of Boussinesq Paradigm Equation, CP 1186, (2009) 217 -- 225.

\bibitem{Ch2012}
C.~I.~Christov,  Numerical implementation of the asymptotic boundary conditions
for steadily propagating 2D solitons of   Boussinesq type equation,       
Math. Computers  Simul., 82 (2012),  1079 -- 1092.

\bibitem{Ch2011}
C.~I.~Christov, J. Choudhury, Perturbation solution  for the 2D Boussinesq equation,       
Mech. Res. Commun., 38 (2011),  274 -- 281.

\bibitem{cher}
A.~Chertock, C.~Christov, A.~Kurganov, Central--upwind schemes for the  Boussinesq paradigm equation, Comp. Sci. High Performance Comp. IV, NNFM, 113, (2011), 267 -- 281.

\bibitem{dani}
C.~Christov, N.~Kolkovska, D.~Vasileva, On the numerical simulation of unsteady solutions for the 2D Boussinesq paradigm equation, LNCS, 6046  (2011), 386 -- 394.

\bibitem{chd-chr}
J.~Choudhury, C.~Christov, 2D  Solitary waves of  Boussinesq equation, CP75, (2005), 85 -- 90.
 
\bibitem{bnd}
K.~Angelow, New Boundary Condition for the Two Dimensional Stationary Boussinesq Paradigm Equation, 
International Journal of Applied Mathematics, Vol. 32, No 1, (2019), 141 -- 154.

\bibitem{forn}
B.~Fornberg, Generation of Finite Difference Formulas on Arbitrarily Spaced Grids, 
Math. Comput., 51(1988),  699 -- 706.

\bibitem{sam}
A.~Samarskii, The theory of difference schemes, M. Dekker,  2001.

\end{thebibliography}

\bigskip
\noindent\sl
\begin{tabular}[b]{l}
Krassimir Angelow\\
Institute of Mathematics and Informatics\\
Bulgarian Academy of Sciences, Acad.\\
G.~Bonchev Bl.8, 1113, Sofia,
Bulgaria\\
e-mail: \texttt{angelow@math.bas.bg}\\
\\
Natalia Kolkovska\\
Institute of Mathematics and Informatics\\
Bulgarian Academy of Sciences, Acad.\\
G.~Bonchev Bl.8, 1113, Sofia,
Bulgaria\\
e-mail: \texttt{n.kolkovska@gmail.com}
\end{tabular}
\hfill
\begin{tabular}[b]{l}
Received ...\\
Final Accepted ...
\end{tabular}

\end{document}

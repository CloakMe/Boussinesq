%%%%%%%%%%%%%%%%%%%%%%%%%%%%%%

%%% SAMPLE_IJAM file
%%% For LaTeX users:


\documentclass[12pt]{article}
%{article}
\usepackage{amssymb}
\usepackage{amsmath, amsthm}
\usepackage{amstext, amsfonts}
\usepackage{graphicx}
\usepackage{epstopdf}
\usepackage{wrapfig}

\newcommand{\dO}{\partial\Omega_{h}}

\textwidth 12cm \textheight 18cm

%%% Theorem Like Envirouments

\newtheoremstyle{theorem}%name
{10pt} % space above
{10pt} % space below
{\sl} % bofy font
{\parindent} % ident - empty=no indent, \parindent= paragraph indent
{\bf} % thm head font
{. } % punctuation after thm head
{ } % space after thm head: `` ``=normal \newline=linebreak
{} % thm head specification
\theoremstyle{theorem}
\newtheorem{theorem}{Theorem}
\newtheorem{corollary}[theorem]{Corollary}

\newtheoremstyle{defi}%name
{10pt} % space above
{10pt} % space below
{\rm} % bofy font
{\parindent} % ident - empty=no indent, \parindent= paragraph indent
{\bf} % thm head font
{. } % punctuation after thm head
{ } % space after thm head: `` ``=normal \newline=linebreak
{} % thm head specification
\theoremstyle{defi}
\newtheorem{definition}[theorem]{Definition}

\def\proofname{\indent {\sl Proof.}}

%%%% Author's Definitions start here
% ..............
%%%% End of Author's Definitions

\begin{document} %%%%%%%%%%%%%%%%%%%%%%%%%%

\title{New Boundary Condition for the Two Dimensional Stationary Boussinesq Paradigm Equation}

\author{K. Angelow $^1$ \\[6pt]
$^1$ angelow@math.bas.bg\\ 
Institute of Mathematics and Informatics\\
Bulgarian Academy of Sciences, Acad. G.~Bonchev Bl.8\\
Sofia - 1113, Bulgaria\\
e-mail: angelow@math.bas.bg\\[6pt] }

\maketitle

\begin{abstract}

In this paper the stationary propagating wave solution to the two dimensional Boussinesq equation is examined, which is a solution to a nonlinear fourth order elliptic equation.  We propose and apply a new boundary condition (BC) on the computational boundary. The numerical algorithm for computation of stationary propagating waves is based on high order accurate finite difference schemes. The performed numerical tests confirm the validity of the new BC. A comparison with the known in the literature formulas is also given. 

\medskip

%{\bf Math. Subject Classification:} ...

{\bf Key Words and Phrases:} stationary wave, Boussinesq equation, boundary, soliton, uniform grid, false transient method

\end{abstract}

%%%%%%%% Section 1 %%%%%%%%%%%%%%%%%%%%
\section{Introduction}
In this paper we consider stationary solutions (solutions of type  $u(x,y,t)=U(x,y - ct)$) to the two dimensional Boussinesq Paradigm Equation (BE) 

\begin{align}
&u_{tt} - \Delta u -\beta_1  \Delta u_{tt} +\beta_2 \Delta ^2 u + \Delta f(u)=0 \quad \text{for} (x,y) \in R^{+} ,\label{eq1}
\\ \nonumber &u(x,y,0)=u_0(x,y), \, u_t(x,y,0)=u_1(x,y)   \quad\text{for} \, (x,y) \in R^2,
\\  &u(x,y) \rightarrow 0,  \Delta u(x,y) \rightarrow 0 ,  \quad \text{for}  \sqrt{x^2 + y^2} \rightarrow \infty, \label{eq11}
\end{align}
where   $f(u)=\alpha u^2$,  $\alpha>0$, $\beta_1>0$, $\beta_2>0$  are dispersion parameters, and $\Delta$ is the Laplace operator. A derivation of the BE from the original Boussinesq system with discussion on the different mechanical properties could be found e.g. in \cite{ref1}. 
The one-dimensional (1D) BE is famous with its approximation for long waves propagating in shallow water \cite{ref2, ref3}. Furthermore, 1D BE admits localized wave solutions (called ‘solitons’), 
\begin{align}
&u_{tt} - u_{xx} -\beta_1  u_{xxtt} +\beta_2 u_{xxxx} + f_{xx}(u) =0, \label{eq2}
\end{align}
which maintain shape and emerge unchanged from collisions with other traveling waves, appear to be a very suitable model for particles \cite{ref4, ref5}.
We try to find a stationary, traveling in y direction with phase velocity c, wave solution to the 2D BE, i.e. a solution to \ref{eq1, eq11} of type $u(x,y,t)=U(x,y - ct)$. The waves U satisfy the nonlinear fourth order elliptic equation:
\begin{equation}
c^2 (E-\beta_1 \Delta) U_{yy} = \Delta U -\beta_2 \Delta^2 U - \Delta f(U), \label{eq3}
\end{equation}
where $E$ is the identity operator. If the condition  $c<min(1/\sqrt{\beta}, 1), \beta = \beta_1/\beta_2$,   holds, then \ref{eq3} is an elliptic equation of fourth order and the linear second order derivatives in \ref{eq3} form a second order elliptic equation. In this paper we consider velocities $c$ which fulfill this inequality.
 The aim of this paper is to evaluate numerically the stationary soliton solution U to \ref{eq1}, i.e. the solution to \ref{eq3}. In the future, it is planned to investigate this solution as a candidate for a two dimensional ‘soliton-like’ solution of the nonstationary problem \ref{eq1} (to study the evolution in time of the shape of this solution and the collision of two solutions).
Different solution techniques have been applied through the investigation of the elliptic problem \ref{eq3}. The “False Transient Method” and the “Galerkin Spectral Method” are used in \cite{ref6,ref9} ; the “Fourier Galerkin Method” is implemented in \cite{ref8,ref9} and the “The Perturbation Solution” - in \cite{ref10}.
We emphasize that both problems, Eq. \ref{eq1} and Eq. \ref{eq3}, are posed on unbounded domain – the plane $R^2$. Thus we have to limit the domain of computation numerically so that the numerical solution will approximate the exact solution for the unbounded domain and, moreover, to keep the overall computational cost reasonable.
So we have to state an artificial boundary $Ω$ and artificial boundary conditions (BC), known in the literature as ‘absorbing’ boundary conditions (BC) or ‘nonreflecting’ BC (see \cite{ref11} for a wave equation, \cite{ref12} for a Helmholtz type equation, \cite{ref13} for elliptic second order equation, etc.). 
\\
The problem for posing artificial BC for BE is studied in \cite{ref6}, where first the following asymptotic of U is found
\begin{equation}
U(r) \sim  C_u/r^2, \text{for} >> 1\label{eq4}
\end{equation}
Text ... (see \cite{gasrah}, \cite{rosbl}, \cite{Moak})...


\begin{definition}
Text of Definition 1.
\end{definition}

\begin{theorem}
Text of Theorem 1.
\end{theorem}

\begin{proof}
The proof of the theorem.
\end{proof}

\begin{corollary}
Text of Corollary 2.
\end{corollary}

%%%%%%%% Section 2 %%%%%%%%%%%%%%%%%%%%
\section{Second Section of the Paper}

Text ...

%%%%%%%%%%% REFERENCES %%%%%%%%%%%%%%%%
\begin{thebibliography}{99}

%% example for a book
\bibitem{gasrah} G. Gasper, M. Rahman,
{\it Basic Hypergeometric Series}, Cambridge University Press, Cambridge (1990).

%% example for paper in journal
\bibitem{Moak} D.S. Moak,
The $q$-analogue of the Laguerre polynomials, {\it J. Math. Anal. Appl.}, {\bf 81} (1981), 20-47.

\bibitem{ref1} C. I. Christov, An energy-consistent dispersive shallow-water model,  {\it Wave Motion }, 34, (2001) 161 - 174
\bibitem{ref2} Boussinesq, J. (1871). “Theorie de l’intumescence liquide, applelee onde solitaire ou de translation, se propageant dans un canal rectangulaire”,  {\it Comptes Rendus de l’Academie des Sciences } 72: 755-759

\bibitem{ref3} Boussinesq, J. (1872). “Theorie des ondes et des remous qui se propagent le long d’un canal rectangulaire horizontal, en communiquant au liquide contenu dans ce canal des vitesses sensiblement pareilles de la surface au fond”, {\it Journal de Mathematiques Pures et Aplliquees}, Deuxieme Serie 17: 55-108

\bibitem{ref4} I. Christov, C.I.Christov, Physical dynamics of quasi-particles in nonlinear wave equations, arxiv.org/pdf/nlin/0612005

\bibitem{ref5} J. K. Perring, T. H. R. Skyrme, A model unified field equation, {\it Nuclear Physics},  31 (1962) p. 550-555 

\bibitem{ref6}  C. I. Christov, Numerical implementation of the asymptotic boundary conditions for steadily propagating 2D solitons of Boussinesq type equations, {\it Mathematics and Computers in simulation}, 82 (2012), p. 1079-1092

\bibitem{ref7}  J. Choudhury, C. Christov, 2D Solitary waves of Boussinesq equation, AIP Conference Proceedings (2005), Volume 755, Issue 1, p. 85 - 90

\bibitem{ref8}  M. Christou, C. I. Christov, Fourier Galerkin method for 2D solitons of Boussinesq equation,  {\it Mathematics and Computers in Simulation} 74 (2007) p. 82 - 92

\bibitem{ref9}   M. Christou, C. I. Christov, Galerkin Spectral Method for the 2D Solitary Waves of Boussinesq Paradigm Equation, AIP Conference Proceedings (2009), Volume 1186, Issue 1, p. 217 - 225

\bibitem{ref10} C. I. Christov, J. Choudhury, Perturbation solution for the 2D Boussinesq equation,{\it Mech. Res. Commun. }, 38 (2011) p. 274 - 281

\bibitem{ref11} B. Engquist and A. Majda, Absorbing boundary conditions for the numerical simulation of waves, {\it Math. Comp.}, 31 (1977), p. 629–651.

\bibitem{ref12}  C. I. Goldstein, A finite element method for solving Helmholtz type equations in waveguides and other unbounded domains, {\it Math. Comp.}, 39 (1982), p. 309–324.

\bibitem{ref13}  H.Han, W. Bao, Error estimates for the finite element approximation of problems in unbounded domains,  {\it SIAM J. NUMER. ANAL.} 37, No. 4, p. 1101–1119

\bibitem{ref14} N. Kolkovska, Two Families of Finite Difference Schemes for Multidimensional Boussinesq Paradigm Equation, AIP Conference Proceedings (2010), Volume 1301, p. 395 http://dx.doi.org/10.1063/1.3526638

\bibitem{ref15} N. Kolkovska, K. Angelow, Numerical Computation of the Critical Energy Constant for Two-dimensional Boussinesq Equations,  AIP Conference Proceedings, Volume 1684, 080007-1–080007-6; doi: 10.1063/1.4934318

\bibitem{ref16} P.Northrop, P.A.Ramachandran, W.Schiesser, V. R.Subramanian, A robust false transient method of lines for elliptic partial differential equations, Chemical Engineering Science 90 (2013) 32–39


%% example for paper in Proc. or Collected Works
\bibitem{rosbl} M. Rosenblum,
Generalized Hermite polynomials and the Bose-like oscillator
calculus, In: {\it Operator Theory: Advances and Applications},
Birkh\"auser, Basel (1994), 369-396.

\end{thebibliography}

\end{document}
%%%%%%%%%%%%%%%%%%%%%%%%%55
